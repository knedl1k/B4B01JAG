% --------------------------------------------------------------
% This is all preamble stuff that you don't have to worry about.
% Head down to where it says "Start here"
% --------------------------------------------------------------
 
\documentclass[11pt]{article}
 
\usepackage[margin=1in]{geometry}
\usepackage{amsmath,amsthm,amssymb}
\usepackage[czech]{babel}

\usepackage{tikz}
\usetikzlibrary{automata, positioning, arrows}


\newcommand{\N}{\mathbb{N}}
\newcommand{\Z}{\mathbb{Z}}
 
\newenvironment{theorem}[2][Theorem]{\begin{trivlist}
\item[\hskip \labelsep {\bfseries #1}\hskip \labelsep {\bfseries #2.}]}{\end{trivlist}}
\newenvironment{lemma}[2][Lemma]{\begin{trivlist}
\item[\hskip \labelsep {\bfseries #1}\hskip \labelsep {\bfseries #2.}]}{\end{trivlist}}
\newenvironment{exercise}[2][Exercise]{\begin{trivlist}
\item[\hskip \labelsep {\bfseries #1}\hskip \labelsep {\bfseries #2.}]}{\end{trivlist}}
\newenvironment{problem}[2][Problem]{\begin{trivlist}
\item[\hskip \labelsep {\bfseries #1}\hskip \labelsep {\bfseries #2.}]}{\end{trivlist}}
\newenvironment{question}[2][Question]{\begin{trivlist}
\item[\hskip \labelsep {\bfseries #1}\hskip \labelsep {\bfseries #2.}]}{\end{trivlist}}
\newenvironment{corollary}[2][Corollary]{\begin{trivlist}
\item[\hskip \labelsep {\bfseries #1}\hskip \labelsep {\bfseries #2.}]}{\end{trivlist}}

\newenvironment{solution}{\begin{proof}[Solution]}{\end{proof}}
 
\begin{document}
 
% --------------------------------------------------------------
%                         Start here
% --------------------------------------------------------------
 
\title{\textbf{Druhá samostatná práce}}%replace X with the appropriate number
\author{Jakub Adamec\\ %replace with your name
B4B01JAG} %if necessary, replace with your course title
 
\maketitle

\textbf{Příklad 3.6.} Navrhněte deterministický konečný automat (DFA), který přijímá jazyk \textit{L} 
abecedou $\{a, b\}$, kde \textit{L} obsahuje právě všechna slova $w$ taková, že
    \begin{itemize}
        \item $w$ začíná $a$
        \item $w$ obsahuje jako podslovo $abb$
        \item $w$ končí $a$.
    \end{itemize}
Automat redukujte.

\tikzset{
->, % makes the edges directed
>=stealth', % makes the arrow heads bold
node distance=3cm, % specifies the minimum distance between two nodes. Change if necessary.
every state/.style={thick, fill=gray!10}, % sets the properties for each ’state’ node
initial text=$ $, % sets the text that appears on the start arrow
}


\begin{tikzpicture}
    \node[state, initial] (q1) {$q_1$};
    \node[state, right of=q1] (q2) {$q_2$};
    \node[state, right of=q2] (q3) {$q_3$};
    \node[state, right of=q3] (q4) {$q_4$};
    \node[state, accepting, right of=q4] (q5) {$q_5$};
    \node[state, below of=q1] (q6) {chyba};
    \draw 
          (q1) edge[bend left, above] node{a} (q2)
          (q1) edge[left] node{b} (q6)
          (q2) edge[loop above] node{a} (q2)
          (q2) edge[bend left, above] node{b} (q3)
          (q3) edge[bend left, below] node{a} (q2)
          (q3) edge[bend left, above] node{b} (q4)
          (q4) edge[loop above] node{b} (q4)
          (q4) edge[bend left, above] node{a} (q5)
          (q5) edge[bend left, below] node{b} (q4)
          (q5) edge[loop above] node{a} (q5)
          (q6) edge[loop right] node{a, b} (q6)
          ;
\end{tikzpicture}

\[
\begin{tabular}{|r r|cc|c|cc|c|cc|c|cc|c|cc|c|cc|c|}
    \hline
    & & $a$ & $b$ & $\sim_0$ & $a$ & $b$ & $\sim_1$ & $a$ & $b$ & $\sim_2$ & $a$ & $b$ & $\sim_3$ & $a$ & $b$ & $\sim_4$ & $a$ & $b$ & $\sim_5$\\
    \hline
    \hline
    $\rightarrow$& $q_1$& $q_2$& chyba& $O$ & $O$ & $O$ & $O$ & $O$ & $O$ & $O$ & $O$ & $O$ & $O$ & $C$ & $O$ & $D$ & $C$ & $O$ & $D$\\
                 & $q_2$& $q_2$& $q_3$& $O$ & $O$ & $O$ & $O$ & $O$ & $O$ & $O$ & $O$ & $B$ & $C$ & $C$ & $B$ & $C$ & $C$ & $B$ & $C$\\
                 & $q_3$& $q_2$& $q_4$& $O$ & $O$ & $O$ & $O$ & $O$ & $A$ & $B$ & $O$ & $A$ & $B$ & $C$ & $A$ & $B$ & $B$ & $A$ & $B$\\
                 & $q_4$& $q_5$& $q_4$& $O$ & $K$ & $O$ & $A$ & $K$ & $A$ & $A$ & $K$ & $A$ & $A$ & $K$ & $A$ & $A$ & $A$ & $A$ & $A$\\
    $\leftarrow$ & $q_5$& $q_5$& $q_4$& $K$ & $K$ & $O$ & $K$ & $K$ & $A$ & $K$ & $K$ & $A$ & $K$ & $K$ & $A$ & $K$ & $K$ & $A$ & $K$\\
                 & chyba& chyba& chyba& $O$ & $O$ & $O$ & $O$ & $O$ & $O$ & $O$ & $O$ & $O$ & $O$ & $O$ & $O$ & $O$ & $O$ & $O$ & $O$\\ 
    \hline
\end{tabular}
\]
\noindent
$\sim_4 = \sim_5$, a protože každý řádek má jinou třídu, původní automat je již redukovaný.


\pagebreak 

\textbf{Příklad 3.7.} Pomocí Nerodovy věty a pomocí pumping lemmatu dokažte, že jazyk $L \subseteq \{a, b\}^*$, kde $L=\{u; |u|_a = |u|_b\}$ není regulární.

\section{Důkaz Nerodovou větou} 
L je regulární $\iff$ existují ekvivalence $T$ na $\sum^*$ taková, že:
\begin{enumerate}
    \item L je sjednocení některých tříd $T$.
    \item pokud $uTv$, tak $uwTvw$ pro každé $w \in \sum^*$.
    \item $T$ má konečný počet tříd.
\end{enumerate}
Kdyby existovala $T$ na $\{a, b\}^*.$\\
Mějme $a^n b^{n-1} = u_1 \not\in L$ a $a^n b^{n-10}= u_2 \not\in L$.\\
A protože předpokládáme regulérnost $L$, tak musí platit 2. bod Nerodovy věty.\\
Zvolme $w=b^1$, a tedy $u_1 w T u_2 w$ musí platit. Po dosazení vyjde $a^n b^n$ $T$ $a^n b^{n-9}$, kde $a^n b^n \in L$, ale $a^n b^{n-9} \not\in L$.\\
Což je ve sporu s 2. bodem Nerodovy věty, protože platí $u_1Tu_2$, ale $u_1 w T u_2 w$ již ne. A tedy $L$ není regulární.


\section{Důkaz Pumping lemmatem}
Je-li $L$ regulární, existuje $n \geq 1$ tak, že každé $u \in L$, $|u|>n$, lze rozdělit $u=xwy$ tak, že:
\begin{enumerate}
    \item $|xw| \leq n$.
    \item $w \neq \varepsilon$.
    \item $xw^iy \in L$, $i=0, 1, ...$
\end{enumerate}
Kdyby $L$ byl regulární, tak existuje $n$ z Pumping lemma.\\
Zvolíme konkrétní slovo $u = a^n b^n$, $u \in L$.\\
Podle Pumping lemmatu lze toto slovo $u$ rozdělit na $u=xwy$ tak, že $|xw| \leq n$. Z toho plyne, že:
    \begin{itemize}
        \item $xw$ se skládá pouze z písmen $a$, protože prvních $n$ symbolů ve slově $u=a^n b^n$ jsou pouze $a$.\\
        Tedy $xw = a^n$.
        \item Dále $w \not\in \varepsilon$, takže $w=a^k$, kde $1 \leq k \leq n$.
    \end{itemize}
Teď napumpujeme $w$, tedy například $i=2$, a dostaneme nové slovo $xw^2y = a^{n+k} b^n$.\\
Pro slovo $a^{n+k} b^n$ platí $|u|_a > |u|_b$, protože má $n+k$ písmen $a$ a $n$ písmen $b$. A tedy $u \not\in L$. \\
A protože Pumping lemma vyžaduje, aby  $\forall i \geq 0$ platilo $xw^iy \in L$, tak lze říct, že $L$ není regulární.


% --------------------------------------------------------------
%     You don't have to mess with anything below this line.
% --------------------------------------------------------------
 
\end{document}

