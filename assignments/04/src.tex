% --------------------------------------------------------------
% This is all preamble stuff that you don't have to worry about.
% Head down to where it says "Start here"
% --------------------------------------------------------------
 
\documentclass[11pt]{article}
 
\usepackage[margin=1in]{geometry}
\usepackage{amsmath,amsthm,amssymb}
\usepackage[czech]{babel}

\usepackage{tikz}
\usetikzlibrary{automata, positioning, arrows, calc}

\usepackage{tabularray}
\usepackage{bm}
 
\begin{document}


\tikzset{
->, % makes the edges directed
>=stealth', % makes the arrow heads bold
node distance=3cm, % specifies the minimum distance between two nodes. Change if necessary.
every state/.style={thick, fill=gray!10}, % sets the properties for each ’state’ node
initial text=$ $, % sets the text that appears on the start arrow
between/.style args={#1 and #2}{
         at = ($(#1)!0.5!(#2)$)
    },
}

\newcommand\splitpage[2]{
      \begin{minipage}[t]{0.45\textwidth}#1
      \end{minipage}%
      \hfill
      \begin{minipage}[t]{0.45\textwidth}#2
      \end{minipage}
}
 
% --------------------------------------------------------------
%                         Start here
% --------------------------------------------------------------
 
\title{\textbf{Čtvrtá samostatná práce}}%replace X with the appropriate number
\author{Jakub Adamec\\ %replace with your name
B4B01JAG} %if necessary, replace with your course title

\maketitle

\textbf{Příklad 7.3.} Je dán regulární výraz \textbf{a(ab$^\star$ + b)$^\star$ba}. K danému regulárnímu výrazu sestrojte redukovaný DFA $M$, který přijímá jazyk reprezentovaný regulárním výrazem.
\\

Konstrukce konečného automatu k regulárnímu výrazu přímou metodou.
$r = a_1 (a_2 b_3^\star + b_4)^\star b_5 a_6$.

1. může začínat: $a_1$.

2. můžou po sobě následovat: $\bm{a_1} \text{: } a_2, b_4, b_5; \bm{a_2}\text{: } a_2, b_3, b_4, b_5; \bm{b_3}\text{: } a_2, b_3, b_4, b_5; \bm{b_4}\text{: } a_2, b_4, b_5; \bm{b_5}\text{: } a_6$.

3. slovo z $L_r$ končí: $a_6$.

4. $\varepsilon$ neleží v $L_r$.

\splitpage{
    a) NFA:

    \begin{tikzpicture}
        \node[state, initial] (q1) {$S$};
        \node[state, right of=q1] (q2) {$a_1$};
        \node[state, right of=q2] (q3) {$a_2$};
        \node[state, below of=q3] (q4) {$b_3$};
        \node[state, below of=q2] (q5) {$b_4$};
        \node[state, below of=q5] (q6) {$b_5$};
        \node[state, accepting, left of=q6] (q7) {$a_6$};
        
        \draw 
            (q1) edge[bend left, above] node{$a$} (q2)
            (q2) edge[bend right] node[right]{$b$} (q5)
            (q2) edge[bend left, above] node{$a$} (q3)
            (q2) edge[bend right, below] node[left]{$b$}(q6)
            (q3) edge[loop above] node{$a$} (q3)
            (q3) edge[bend left, above] node[right]{$b$} (q4)
            (q3) edge[below] node[right]{$b$}(q6)
            (q3) edge[bend right, below] node[left]{$b$}(q5)
            (q4) edge[loop right] node{$b$}(q4)
            (q4) edge[above] node[right]{$a$} (q3)
            (q4) edge[bend right, above] node{$b$}(q5)
            (q4) edge[bend left, above] node{$b$}(q6)
            (q5) edge[loop right] node{$b$}(q5)
            (q5) edge[below] node[left]{$b$}(q6)
            (q5) edge node[left]{$a$}(q3)
            (q6) edge node[above]{$a$}(q7)
            ;
    \end{tikzpicture}
}{
c) redukovaný DFA:

    \begin{tikzpicture}
        \node[state, initial] (q1) {$C$};
        \node[state, below of=q1] (q2) {$O$};
        \node[state, right of=q1] (q3) {$B$};
        \node[state, below of=q3] (q4) {$A$};
        \node[state, right of=q3, accepting] (q5) {$K$};
        
        \draw 
            (q1) edge[bend left, above] node{$a$} (q3)
            (q1) edge[bend left, below] node[right]{$b$} (q2)
            (q2) edge[loop below] node{$a,b$} (q2)
            (q3) edge[loop above] node{$a$} (q3)
            (q3) edge[bend right, below] node[left]{$b$} (q4)
            (q4) edge[loop below] node{$b$} (q4)
            (q4) edge[bend right, above] node{$a$}(q5)
            (q5) edge[bend right, above] node{$a$}(q3)
            (q5) edge[bend right, below] node{$b$}(q4)
            ;
    \end{tikzpicture}
}

b) podmnožinová konstrukce DFA a následná redukce
\[
\begin{tabular}{|r r|c c|c|c c|c|c c|c|c c|c|c c|c|}
    \hline
    & & $a$ & $b$ & $\sim_0$ & $a$ & $b$ & $\sim_1$ & $a$ & $b$ & $\sim_2$ & $a$ & $b$ & $\sim_3$ & $a$ & $b$ & $\sim_4$\\
    \hline
    \hline
    $\rightarrow$& $S$                &$a_1$         &$\emptyset$        & $O$ & $O$ & $O$ & $O$ & $O$ & $O$ & $O$ & $B$ & $O$ & $C$ & $B$ & $O$ & $C$\\
                 & $a_1$              &$a_2$         &$\{b_4, b_5\}$     & $O$ & $O$ & $O$ & $O$ & $O$ & $A$ & $B$ & $B$ & $A$ & $B$ & $B$ & $A$ & $B$\\
                 & $\emptyset$        &$\emptyset$   &$\emptyset$        & $O$ & $O$ & $O$ & $O$ & $O$ & $O$ & $O$ & $O$ & $O$ & $O$ & $O$ & $O$ & $O$\\
                 & $a_2$              &$a_2$         &$\{b_3, b_4, b_5\}$& $O$ & $O$ & $O$ & $O$ & $O$ & $A$ & $B$ & $B$ & $A$ & $B$ & $B$ & $A$ & $B$\\
                 & $\{b_4, b_5\}$     &$\{a_2, a_6\}$&$\{b_4, b_5\}$     & $O$ & $K$ & $O$ & $A$ & $K$ & $A$ & $A$ & $K$ & $A$ & $A$ & $K$ & $A$ & $A$\\
                 & $\{b_3, b_4, b_5\}$&$\{a_2, a_6\}$&$\{b_3, b_4, b_5\}$& $O$ & $K$ & $O$ & $A$ & $K$ & $A$ & $A$ & $K$ & $A$ & $A$ & $K$ & $A$ & $A$\\
   $\leftarrow$ & $\{a_2, a_6\}$     &$a_2$         &$\{b_3, b_4, b_5\}$& $K$ & $O$ & $O$ & $K$ & $O$ & $A$ & $K$ & $B$ & $A$ & $K$ & $B$ & $A$ & $K$\\ 
    \hline
\end{tabular}
\]


\pagebreak 
\textbf{Příklad 7.4.} Pro daný DFA $M$ vytvořte regulární výraz, který reprezentuje jazyk $L(M)$.
\[ M:
    \begin{tabular}{|r|c c|}
        \hline
        & $a$ & $b$\\
        \hline
        \hline
        $\leftrightarrow 1$ & $1$ & $2$\\
        $2$                 & $3$ & $4$\\
        $\leftarrow 3$      & $1$ & $2$\\
        $            4$     & $4$ & $4$\\
        \hline
    \end{tabular}
\]
Postup řešení vysvětlete.
\\

\splitpage{
1. Úvodní DFA

\begin{tikzpicture}
   \node[state, initial, accepting] (q1) {$1$};
   \node[state, right of=q1] (q2) {$2$};
   \node[state, accepting, below of=q1] (q3) {$3$};
   \node[state, below of=q2] (q4) {$4$};
        
   \draw    
        (q1) edge[loop above] node{$a$} (q1)
        (q1) edge[bend left, above] node{$b$} (q2)
        (q2) edge[bend left, above] node{$a$} (q3)
        (q2) edge[bend left, above] node[left]{$b$} (q4)
        (q3) edge[bend left, above] node[left]{$a$} (q1)
        (q3) edge[bend left, above] node[left]{$b$} (q2)
        (q4) edge[loop left] node[above]{$a,b$} (q4)
        ;
\end{tikzpicture}
}{

2. Přidání $S$ a $F$ stavů.

\begin{tikzpicture}
   \node[state, initial] (q5) {$S$};
   \node[state, right of=q5] (q1) {$1$};
   \node[state, right of=q1] (q2) {$2$};
   \node[state, below of=q1] (q3) {$3$};
   \node[state, below of=q2] (q4) {$4$};
   \node[state, below of=q5, accepting] (q6) {$F$};
        
   \draw    
        (q1) edge[loop above] node{$a$} (q1)
        (q1) edge[bend left, above] node{$b$} (q2)
        (q1) edge[bend right, below] node{$\varepsilon$} (q6)
        (q2) edge[bend left, above] node{$a$} (q3)
        (q2) edge[bend left, above] node[left]{$b$} (q4)
        (q3) edge[bend left, above] node[left]{$a$} (q1)
        (q3) edge[bend left, above] node[left]{$b$} (q2)
        (q4) edge[loop left] node[above]{$a,b$} (q4)
        (q5) edge[bend left, above] node{$\varepsilon$} (q1)
        (q3) edge[bend left, above] node{$\varepsilon$} (q6)
        ;
\end{tikzpicture}
}

\splitpage{
3. Odstranění smyček.

\begin{tikzpicture}
   \node[state, initial] (q5) {$S$};
   \node[state, right of=q5] (q1) {$1$};
   \node[state, right of=q1] (q2) {$2$};
   \node[state, below of=q1] (q3) {$3$};
   \node[state, below of=q2] (q4) {$4$};
   \node[state, below of=q5, accepting] (q6) {$F$};
        
   \draw 
        (q1) edge[bend left, above] node{$a^\star b$} (q2)
        (q1) edge[bend right, below] node{$a ^\star$} (q6)
        (q2) edge[bend left, above] node{$a$} (q3)
        (q2) edge[bend left, above] node[left]{$b$} (q4) %FIXME Im not sure about this one (a^\star + b^\star)
        (q3) edge[bend left, above] node[left]{$a$} (q1)
        (q3) edge[bend left, above] node[left]{$b$} (q2)
        (q5) edge[bend left, above] node{$\varepsilon$} (q1)
        (q3) edge[bend left, above] node{$\varepsilon$} (q6)
        ;
\end{tikzpicture}
}{
4. Odstranění vrcholu č. 4.

\begin{tikzpicture}
   \node[state, initial] (q5) {$S$};
   \node[state, right of=q5] (q1) {$1$};
   \node[state, right of=q1] (q2) {$2$};
   \node[state, below of=q1] (q3) {$3$};
   \node[state, below of=q5, accepting] (q6) {$F$};
        
   \draw 
        (q1) edge[bend left, above] node{$a^\star b$} (q2)
        (q1) edge[bend right, below] node{$a ^\star$} (q6)
        (q2) edge[bend left, above] node{$a$} (q3)
        (q3) edge[bend left, above] node[left]{$a$} (q1)
        (q3) edge[bend left, above] node[left]{$b$} (q2)
        (q5) edge[bend left, above] node{$\varepsilon$} (q1)
        (q3) edge[bend left, above] node{$\varepsilon$} (q6)
        ;
\end{tikzpicture}
}

\splitpage{
5. Odstranění vrcholu č. 2.

\begin{tikzpicture}
   \node[state, initial] (q5) {$S$};
   \node[state, right of=q5] (q1) {$1$};
   \node[state, below of=q1] (q3) {$3$};
   \node[state, below of=q5, accepting] (q6) {$F$};
        
   \draw 
        (q1) edge[bend left, above] node[right]{$a^\star b a$} (q3)
        (q1) edge[bend right, below] node{$a ^\star$} (q6)
        (q3) edge[bend left, above] node[left]{$a$} (q1)
        (q3) edge[loop right] node{$ba$} (q3)
        (q5) edge[bend left, above] node{$\varepsilon$} (q1)
        (q3) edge[bend left, above] node{$\varepsilon$} (q6)
        ;
\end{tikzpicture}
}{
6. Odstranění smyček.

\begin{tikzpicture}
   \node[state, initial] (q5) {$S$};
   \node[state, right of=q5] (q1) {$1$};
   \node[state, below of=q1] (q3) {$3$};
   \node[state, below of=q5, accepting] (q6) {$F$};
        
   \draw 
        (q1) edge[bend left, above] node[right]{$a^\star b a$} (q3)
        (q1) edge[bend right, below] node[left]{$a ^\star$} (q6)
        (q3) edge[bend left, above] node[left]{$(ba)^\star a$} (q1)
        (q5) edge[bend left, above] node{$\varepsilon$} (q1)
        (q3) edge[bend left, above] node{$(ba)^\star$} (q6)
        ;
\end{tikzpicture}
}
\[
\text{Pokračování na další straně.}
\]

\splitpage{
7. Odstranění vrcholu č. 3.

\begin{tikzpicture}
   \node[state, initial] (q5) {$S$};
   \node[state, right of=q5] (q1) {$1$};
   \node[state, below of=q5, accepting] (q6) {$F$};
        
   \draw 
        (q1) edge[loop right] node[right]{$a^\star b a (ba)^\star a$} (q1)
        (q1) edge[bend right, below] node[left]{$a ^\star$} (q6)
        (q1) edge[bend left, below] node[right]{$a^\star b a (b a)^\star$} (q6)
        (q5) edge[bend left, above] node{$\varepsilon$} (q1)
        ;
\end{tikzpicture}
}{
8. Odstranění paralelních hran.

\begin{tikzpicture}
   \node[state, initial] (q5) {$S$};
   \node[state, right of=q5] (q1) {$1$};
   \node[state, below of=q5, accepting] (q6) {$F$};
        
   \draw 
        (q1) edge[loop right] node[right]{$a^\star b a (ba)^\star a$} (q1)
        (q1) edge[bend left, below] node[right]{$a ^\star + a^\star b a (b a)^\star$} (q6)
        (q5) edge[bend left, above] node{$\varepsilon$} (q1)
        ;
\end{tikzpicture}
}

\splitpage{
9. Odstranění smyček.

\begin{tikzpicture}
   \node[state, initial] (q5) {$S$};
   \node[state, right of=q5] (q1) {$1$};
   \node[state, below of=q5, accepting] (q6) {$F$};
        
   \draw 
        (q1) edge[bend left, below] node[right]{$(a^\star b a (ba)^\star a)^\star (a ^\star + a^\star b a (b a)^\star)$} (q6)
        (q5) edge[bend left, above] node{$\varepsilon$} (q1)
        ;
\end{tikzpicture}
}{
10. Odstranění vrcholu č. 1.
\\

\begin{tikzpicture}
   \node[state, initial] (q5) {$S$};
   \node[state, right of=q5, accepting] (q6) {$F$};
        
   \draw 
        (q5) edge[bend right, below] node[below]{$(a^\star b a (ba)^\star a)^\star (a ^\star + a^\star b a (b a)^\star)$} (q6)
        ;
\end{tikzpicture}
}

\[
    \text{Výsledný regulární výraz je: } (a^\star b a (ba)^\star a)^\star (a ^\star + a^\star b a (b a)^\star) \text{.}
\]


% --------------------------------------------------------------
%     You don't have to mess with anything below this line.
% --------------------------------------------------------------
 
\end{document}

