% --------------------------------------------------------------
% This is all preamble stuff that you don't have to worry about.
% Head down to where it says "Start here"
% --------------------------------------------------------------
 
\documentclass[11pt]{article}
 
\usepackage[margin=1in]{geometry}
\usepackage{amsmath,amsthm,amssymb,mathtools}
\usepackage[czech]{babel}

\usepackage{tikz}
\usetikzlibrary{automata, positioning, arrows, calc}

\usepackage{tabularray}
\usepackage{bm}
 
\begin{document}


\tikzset{
->, % makes the edges directed
>=stealth', % makes the arrow heads bold
node distance=3cm, % specifies the minimum distance between two nodes. Change if necessary.
every state/.style={thick, fill=gray!10}, % sets the properties for each ’state’ node
initial text=$ $, % sets the text that appears on the start arrow
between/.style args={#1 and #2}{
         at = ($(#1)!0.5!(#2)$)
    },
}

\newcommand\splitpage[2]{
      \begin{minipage}[t]{0.45\textwidth}#1
      \end{minipage}%
      \hfill
      \begin{minipage}[t]{0.45\textwidth}#2
      \end{minipage}
}
 
% --------------------------------------------------------------
%                         Start here
% --------------------------------------------------------------
 
\title{\textbf{Pátá samostatná práce}}
\author{Jakub Adamec\\
B4B01JAG}

\maketitle

\noindent
\textbf{Příklad 8.7.} K automatu $M$ zkonstruujte gramatiku typu 3, která generuje jazyk $L(M)$, kde $M$ je dán tabulkou
\[
    \begin{tabular}{|r|c|c|}
        \hline
        & $a$ & $b$\\
        \hline
        \hline
        $\rightarrow A$     & $\{A,B\}$   & $\{C\}$\\
        $B$                 & $\{B\}$     & $\{C\}$\\
        $\leftrightarrow C$ & $\emptyset$ & $\{D\}$\\
        $\leftarrow D$      & $\{B\}$     & $\{D\}$\\
        \hline
    \end{tabular}
\]
\splitpage{
Protože automat $M$ má dva počátčení stavy, vytvoříme nový automat $M'$ takto:

\[
    \begin{tabular}{|r|c|c|c|}
        \hline
        & $\varepsilon$ & $a$ & $b$\\
        \hline
        \hline
        $\rightarrow S$&$\{A,C\}$  &$\emptyset$  &$\emptyset$\\
        $A$            &$\emptyset$& $\{A,B\}$   & $\{C\}$\\
        $B$            &$\emptyset$& $\{B\}$     & $\{C\}$\\
        $\leftarrow C$ &$\emptyset$& $\emptyset$ & $\{D\}$\\
        $\leftarrow D$ &$\emptyset$& $\{B\}$     & $\{D\}$\\
        \hline
    \end{tabular}
\]


\[
    \begin{tikzpicture}
        \node[state, initial] (q1) {$S$};
        \node[state, accepting, right of=q1] (q4) {$C$};
        \node[state, below of=q4] (q2) {$A$};
        \node[state, right of=q2] (q3) {$B$};
        \node[state, accepting, right of=q4] (q5) {$D$};
            
        \draw 
            (q1) edge[bend right, above] node{$\varepsilon$} (q2)
            (q1) edge node[above]{$\varepsilon$} (q4)
            (q2) edge[loop below] node{$a$} (q2)
            (q2) edge node[above]{$a$} (q3)
            (q2) edge node[left]{$b$} (q4)
            (q3) edge[loop below] node{$a$} (q3)
            (q3) edge node[above]{$b$}(q4)
            (q4) edge node[above]{$b$}(q5)
            (q5) edge[loop right] node{$b$} (q5)
            (q5) edge[bend left] node[right]{$a$} (q3)
            ;
    \end{tikzpicture}
\]
}{
Nyní hledaná gramatika je $\mathcal{G} = (N, \Sigma, S, P)$, kde
$N = \{S, A, B, C, D\}$, $\Sigma = \{a, b\}$ a
\begin{flalign*}
    P:  S & \rightarrow A \mid C&& \\
        A & \rightarrow aA \mid aB \mid bC&& \\
        B & \rightarrow aB \mid bC&& \\
        C & \rightarrow bD \mid \varepsilon&& \\ 
        D & \rightarrow aB \mid bD \mid \varepsilon&&
\end{flalign*}
}

\newpage
\noindent
\textbf{Příklad 8.8.} Navrhněte bezkontextovou gramatiku $\mathcal{G}$, která generuje jazyk ${L = \{0^i1^j; 0 \leq i \leq j\}}$. Zdůvodněte, proč gramatika $\mathcal{G}$ jazyk $L$ generuje.\\

\noindent
Hledaná gramatika $\mathcal{G} = (N, \Sigma, S, P)$, kde $N=\{S\}$, $\Sigma = \{0,1\}$ a $S \rightarrow 0S1 \mid S1 \mid \varepsilon$.\\

\noindent
zdůvodnění:

\begin{enumerate}
    \item $L \subseteq L(\mathcal{G})$, tj. každé slovo $0^i 1^j$, $0 \leq i \leq j$, gramatika $\mathcal{G}$ vygeneruje.

        $S \xRightarrow{S \rightarrow 0S1}^{(i)} 0^i S 1^i \xRightarrow{S \rightarrow S1}^{(j-i)} 0^i S 1^{j-i} 1^i \xRightarrow{S \rightarrow \varepsilon} 0^i 1^j$. \qed
    \item $L(\mathcal{G}) \subseteq L$, tj. $\mathcal{G}$ nevygeneruje nic navíc.

        Uvažujme derivaci $S \Rightarrow^\star u$. Pak poslední použité pravidlo musí být $S \rightarrow \varepsilon$. Indukcí podle počtu kroků dokážeme, že $S \Rightarrow ^{(n)} 0^i S q^j$, kde $i \leq j$.

        \textit{Základní krok.} $n=1$, $S \Rightarrow 0S1$ nebo $S \Rightarrow S1$ a $S1 = 0^0 S 1^1$, kde $i \leq j$.\\
        \textit{Indukční krok.} předpokládejme, že každá derivace o $n$ krocích vygeneruje $S \Rightarrow^{(n)} 0^i S 1^j$, $i \leq j$.
        \noindent
        \[
            \text{Pak derivace o $n+1$ krocích je}
            \begin{cases*}
                S \Rightarrow^{(n)} 0^i S 1^j \xRightarrow{S \rightarrow 0S1} 0^{i+1} S 1^{j+1} \text{ a } i+1 \leq j+1. \\
                S \Rightarrow^{(n)} 0^i S 1^j \xRightarrow{S \rightarrow S1} 0^i S 1^{j+1} \text{ a } i \leq j+1. 
            \end{cases*}
        \]
        Tedy z $S$ je možné vygenerovat pouze $0^i S 1^j$, $0 \leq i \leq j$, tedy $S \Rightarrow^\star 0^i S 1^j \xRightarrow{S \rightarrow \varepsilon} 0^i 1^j$, $0 \leq i \leq j$.
        
        \qed
        
\end{enumerate}


% --------------------------------------------------------------
%     You don't have to mess with anything below this line.
% --------------------------------------------------------------
 
\end{document}
