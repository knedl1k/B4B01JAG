% --------------------------------------------------------------
% This is all preamble stuff that you don't have to worry about.
% Head down to where it says "Start here"
% --------------------------------------------------------------
 
\documentclass[11pt]{article}
 
\usepackage[margin=1in]{geometry}
\usepackage{amsmath,amsthm,amssymb,mathtools}
\usepackage[czech]{babel}

\usepackage{tikz}
\usetikzlibrary{automata, positioning, arrows, calc}

\usepackage{tabularray}
\usepackage{bm}
\usepackage{amsmath}
\usepackage{ragged2e} 
\usepackage{booktabs, tabularx}
\usepackage{makecell}
\usepackage{nicematrix,tikz}
\usepackage{multicol}
\usepackage{tikz-qtree}
 
\begin{document}


\tikzset{
->, % makes the edges directed
>=stealth', % makes the arrow heads bold
node distance=3cm, % specifies the minimum distance between two nodes. Change if necessary.
every state/.style={thick, fill=gray!10}, % sets the properties for each ’state’ node
initial text=$ $, % sets the text that appears on the start arrow
between/.style args={#1 and #2}{
         at = ($(#1)!0.5!(#2)$)
    },
}

\newcommand\splitpage[2]{
      \begin{minipage}[t]{0.45\textwidth}#1
      \end{minipage}%
      \hfill
      \begin{minipage}[t]{0.45\textwidth}#2
      \end{minipage}
}
 
% --------------------------------------------------------------
%                         Start here
% --------------------------------------------------------------
 
\title{\textbf{Sedmá samostatná práce}}
\author{Jakub Adamec\\ %replace with your name
B4B01JAG} %if necessary, replace with your course title

\maketitle

\noindent
\textbf{Příklad 10.5.} Je dána gramatika $\mathcal{G} = (N, \Sigma, S, P)$, kde $N = \{S, A, B, C, D\}$, $\Sigma = \{a,b,c\}$ a pravidla P jsou dána
\begin{align*}
    P:    S & \rightarrow AB \mid CD \mid AC \\
          A & \rightarrow AC \mid a \\
          B & \rightarrow BD \mid b \\ 
          C & \rightarrow AD \mid a \\
          D & \rightarrow BA \mid b \\
\end{align*}
Algoritmem CYK rozhodněte, zda gramatika $\mathcal{G}$ generuje slova $w_1$ a $w_2$, kde ${w_1 = baaba}$ a ${w_2 = abaaa}$. Pokud ano, nakreslete derivační strom a napište jemu odpovídající levou derivaci.
\\

\splitpage{
        CYK pro slovo $w_1$.
        
        \begin{NiceTabular}{|c|c|c|c|c|c|c|c}[hlines, corners = NE] % NE = north east
            $D$   &         &       &       &   \\ 
            $D$   & $S,A,C$ &       &       &   \\ 
            $D$   & $S,A,C$ & $S,C$ &       &   \\ 
            $D$   & $S,A$   & $S,C$ & $D$   &   \\ 
            $B,D$ & $A,C$   & $A,C$ & $B,D$ & $A,C$ \\
            \hline
            $b$   & $a$     & $a$   & $b$   & $a$
        \end{NiceTabular}

        $\implies$ gramatika $\mathcal{G}$ negeneruje slovo $w_1$.
        \\
        \\
        \\
        
        Pomocný přepis pravidel.
        
        \begin{falign*}
            AB  \leftarrow S\\
            AC  \leftarrow S,A \\
            AD  \leftarrow C \\ 
            BA  \leftarrow D \\
            BD  \leftarrow B \\
            CD  \leftarrow S \\
        \end{falign*}
        
}{
        CYK pro slovo $w_2$.
        
        \begin{NiceTabular}{|c|c|c|c|c|c|c|c}[hlines, corners = NE] % NE = north east
            $S,C$ &       &       &       &   \\ 
            $S,C$ & $D$   &       &       &   \\ 
            $S,C$ & $D$   & $S,A$ &       &   \\ 
            $S,C$ & $D$   & $S,A$ & $S,A$ &   \\ 
            $A,C$ & $B,D$ & $A,C$ & $A,C$ & $A,C$ \\
            \hline
            $a$   & $b$   & $a$   & $a$   & $a$
        \end{NiceTabular}
        
        $\implies$ gramatika $\mathcal{G}$ generuje slovo $w_2$.   

        Derivační strom pro $w_2$:
        
        \begin{tikzpicture}
          \Tree [.$S$ [.$C$ $a$ ]
                      [.$D$ [.$B$ $b$ ]
                            [.$A$ [.$A$ [.$A$ $a$ ]  
                                        [.$C$ $a$ ] ]
                                  [.$C$ $a$ ] ] ] ]
        \end{tikzpicture}
        
        Levá derivace: $S \overset{S \rightarrow CD}{\implies} CD \overset{C \rightarrow a }{\implies} aD \overset{D \rightarrow BA }{\implies} {aBA \overset{B \rightarrow b }{\implies} abA \overset{A \rightarrow AC }{\implies} abAC \overset{A \rightarrow AC }{\implies} abACC} \overset{A \rightarrow a }{\implies} abaCC \overset{C \rightarrow a }{\implies} abaaC \overset{C \rightarrow a }{\implies} abaaa$.
}


% --------------------------------------------------------------
%     You don't have to mess with anything below this line.
% --------------------------------------------------------------
 
\end{document}

