\documentclass[11pt]{article}
 
\usepackage[margin=1in]{geometry}
\usepackage{amsmath,amsthm,amssymb,mathtools}
\usepackage[czech]{babel}

\usepackage{tikz}
\usetikzlibrary{automata, positioning, arrows, calc}

\usepackage{tabularray}
\usepackage{bm}
\usepackage{multicol}
 
\begin{document}


\tikzset{
->, % makes the edges directed
>=stealth', % makes the arrow heads bold
node distance=3cm, % specifies the minimum distance between two nodes. Change if necessary.
every state/.style={thick, fill=gray!10}, % sets the properties for each ’state’ node
initial text=$ $, % sets the text that appears on the start arrow
between/.style args={#1 and #2}{
         at = ($(#1)!0.5!(#2)$)
    },
}

\newcommand\splitpage[2]{
      \begin{minipage}[t]{0.45\textwidth}#1
      \end{minipage}%
      \hfill
      \begin{minipage}[t]{0.45\textwidth}#2
      \end{minipage}
}
 
% --------------------------------------------------------------
%                         Start here
% --------------------------------------------------------------
 
\title{\textbf{První samostatná práce}}
\author{Jakub Adamec\\ %replace with your name
B4B01JAG} %if necessary, replace with your course title

\maketitle

\noindent
\textbf{Příklad 1.5.}
Pro uvedený automat nakreslete stavový diagram. Najděte vlastnost $\mathcal{V}$, která charakterizuje slova přijímaná daným automatem. Dokažte, že automat přijímá právě všechna slova s vlastností $\mathcal{V}$.
\begin{multicols}{2} % Start two-column layout
    % Left column content
    \section*{}
        \begin{tikzpicture}
            \node[state, initial, accepting] (q1) {$q_0$};
            \node[state, accepting, right of=q1] (q2) {$q_1$};
            \node[state, accepting, below of=q1] (q3) {$q_2$};
                
            \draw 
                (q1) edge[loop above] node{$0$} (q1)
                (q1) edge[bend left, above] node{$1$} (q2)
                (q2) edge[loop above] node{$1$} (q2)
                (q2) edge[bend right, below] node{$0$} (q3)
                (q3) edge[bend left, above] node[left]{$0$} (q1)
                (q3) edge[bend right, above] node{$1$} (q2)
                ;
        \end{tikzpicture}
        \\
        $F = \{q_1, q_2\}$.\\
        $L(M) = \{w \mid w \text{ končí } 1 \text{ nebo } 10\}$.\\
        Ať $u \in L$.
    
    % Right column content
    \columnbreak % Force a break to the next column
    \section*{}
        \begin{multicols}{2} % Start two-column layout
            % Left column content
            \section*{}
                Důkaz $u0 \not\in F:$\\
                $\delta(q_i, 0) = q_0 \not\in F$\\
                $i = 0, 2$.
                \\
                \\
                \\
                Důkaz $u00 \not\in F:$\\
                $\delta(q_i, 00) = q_0 \not\in F$\\
                $i = 0, 1, 2$.
                \\
                \\
                \\
                Důkaz $\varepsilon \not\in F:$\\
                $\delta(q_1, \varepsilon) = q_1$.\\ 
                
            % Right column content
            \columnbreak % Force a break to the next column
            \section*{}
                Důkaz $u1 \in F:$\\
                $\delta(q_i, 1) = q_1$\\
                $i = 0, 1, 2$.
                \\
                \\
                \\
                Důkaz $u10 \in F:$\\
                $\delta(q_i, 10) = q_2$\\
                $i = 0, 1, 2$.
            
        \end{multicols}
    \end{multicols}
\newpage
\noindent
\textbf{Příklad 1.6.} Jazyk $L$ nad abecedou $\Sigma = \{a, b\}$ je dán induktivně

\begin{gather*}
    \varepsilon \in L \\
    u \in L \implies aua \in L \\
    u \in L \implies bub \in L
\end{gather*}

\noindent
Charakterizujte slova jazyka $L$, tj. najděte vlastnost $\mathcal{V}$ takovou, že $L = \{u \mid \text{slovo } u \text{ má vlastnost } \mathcal{V}\}$. Své tvrzení zdůvodněte.\\

\noindent
$\mathcal{V} = \text{slovo } u \text{ je sudé délky a } u = u^R \text{(tj. je palindrom).}$
Označme $L_1 = \{u \mid u^R = u, |u| \text{ je sudé}\}$. Dokážeme, že $L=L_1$.\\

\noindent
a) $L \subseteq L_1$, indukcí podle definice množiny $L$.

i) $\varepsilon$ je sudé délky a zároveň $\varepsilon^R = \varepsilon$, tedy $\varepsilon \in L_1$. \qed

ii) Mějme slovo $u$, které je sudé délky a platí $u^R = u$. Pak také slova $v_1 = aua$ a $v_2=bub$ mají sudou délku a jsou palindromy, tj. ${v_1^R = (aua)^R = au^Ra=aua=v_1}$ a ${v_2^R = (bub)^R = bu^Rb=bub=v_2}$. \qed

\noindent
b) $L_1 \subseteq L$, každé slovo, které palindrom sudé délky vzniklo dle pravidel, indukcí podle délky slova $u \in L_1$.

i) Nejkratší slovo podle pravidel $\mathcal{V}$ je $\varepsilon$, které patří do L. \qed

ii) Předpokládejme, že všechny palindromy $v$ délky $2n$ vznikly podle pravidel jazyka $L$. \\
Uvažujme libovolný palindrom $u$ délky $2(n+1)$. Pak $u$ nutně začíná buď písménem $a$ nebo písmenem $b$. Jestliže $u$ začíná $a$, pak musí končit $a$, protože je palindromem. Pak lze tedy říct, že $u=ava$, navíc platí $u^R = av^Ra = u$, proto $u^R = v$ a $|v|=2n$. Z indukčního předpokladu víme, že $v \in L$ a tedy i $u \in L$.

A analogicky platí to samé pro případ, že $u$ začíná, a tedy i končí, písmenem $b$.
\qed
\\
\\
Takže platí, že $L = L_1$.

% --------------------------------------------------------------
%     You don't have to mess with anything below this line.
% --------------------------------------------------------------
 
\end{document}

