% --------------------------------------------------------------
% This is all preamble stuff that you don't have to worry about.
% Head down to where it says "Start here"
% --------------------------------------------------------------
 
\documentclass[11pt]{article}
 
\usepackage[margin=1in]{geometry}
\usepackage{amsmath,amsthm,amssymb,mathtools}
\usepackage[czech]{babel}

\usepackage{tikz}
\usetikzlibrary{automata, positioning, arrows, calc}

\usepackage{tabularray}
\usepackage{bm}
 
\begin{document}


\tikzset{
->, % makes the edges directed
>=stealth', % makes the arrow heads bold
node distance=3cm, % specifies the minimum distance between two nodes. Change if necessary.
every state/.style={thick, fill=gray!10}, % sets the properties for each ’state’ node
initial text=$ $, % sets the text that appears on the start arrow
between/.style args={#1 and #2}{
         at = ($(#1)!0.5!(#2)$)
    },
}

\newcommand\splitpage[2]{
      \begin{minipage}[t]{0.45\textwidth}#1
      \end{minipage}%
      \hfill
      \begin{minipage}[t]{0.45\textwidth}#2
      \end{minipage}
}
 
% --------------------------------------------------------------
%                         Start here
% --------------------------------------------------------------
 
\title{\textbf{Šestá samostatná práce}}
\author{Jakub Adamec\\ %replace with your name
B4B01JAG} %if necessary, replace with your course title

\maketitle

\noindent
\textbf{Příklad 9.9.} 

Navrhněte bezkontextovou gramatiku generující následující jazyk ${L = \{a^n b^m a^n \mid m, n \geq 0\}}$.\\
Zdůvodněte, proč zkonstruovaná gramatika jazyk $L$ generuje.\\

\noindent
Gramatika $\mathcal{G}=(N, \Sigma, S, P)$, kde $N=\{S, A\}$ a

\noindent
$P: $
$S \rightarrow aSa \mid b$

$A \rightarrow bA \mid \varepsilon$\\


\noindent
Důkaz:
\\

\textbf{1.} $a^n n^m a^n \in L(y)$.

$S \xRightarrow{S \rightarrow a S a }^{(n)} a^n S a^n \xRightarrow{S \rightarrow A} a^n A a^n \xRightarrow{A \rightarrow bA}^{(m)} a^n b^m A a^n \xRightarrow{A \rightarrow \varepsilon} a^n b^m a^n , m,n \geq 0$.\\

\textbf{2.} $\mathcal{G}$ nevygeneruje nic navíc.

Předpokládejme, že $S \Rightarrow^\star u, u \in \Sigma^\star$. Pak poslední pravidlo derivace musí být $A \rightarrow \varepsilon$. Proto v derivaci musí být použito pravidlo $S \rightarrow A$. Mezi použitím pravidel $S \rightarrow A$ a $A \rightarrow \varepsilon$ může být použito několik (nebo žádné) pravidel $S \rightarrow bA$. Protože derivace začíná $S$, před použitím pravidla $S \rightarrow A$ může být použito několik (nebo žádné) pravidel $S \rightarrow aSa$. Jinak pravidla být použita nemohou. Tedy derivace má tvar $S \xRightarrow{S \rightarrow a S a}^{(k)} a^k S a^k \xRightarrow{S \rightarrow A} a^k A a^k \xRightarrow{A \rightarrow bA}^{(l)} a^k b^l A a^k \xRightarrow{A \rightarrow \varepsilon} a^k b^l a^k , k,l \geq 0$.


\newpage
\noindent
\textbf{Příklad 9.10.} 

\noindent
\splitpage{
Zredukujte gramatiku $\mathcal{G}$, která je dána {pravidly}:
    \begin{flalign*}
        \mathcal{G}:    S & \rightarrow aA \mid bB \mid aSa \mid bSb \mid \varepsilon&& \\
                        A & \rightarrow bCD \mid Dba&& \\
                        B & \rightarrow Bb \mid AC&& \\ 
                        C & \rightarrow aA \mid a&& \\
                        D & \rightarrow DE&& \\
                        E & \rightarrow \varepsilon&&
    \end{flalign*}

}{
\\
\\
\\
\\

Mějme CF gramatiku $\mathcal{G} = \{N, \Sigma, S, P\}$.
\\

1. krok

$V = \{A \mid A \in N, A \Rightarrow_\mathcal{G} ^\star w, w \in \Sigma^\star\}$

$V_1 =\{A \mid A \rightarrow w \in P, w \in \Sigma^\star\} =\{S, C, E\}$

${V_2 = V_1 \cup \{A \mid \alpha \in P, \alpha \in (\Sigma \cup V_1)^\star\} = \{S, C, E\} \cup \emptyset = V_1}$
}
\\
\\
\\

\noindent
2. krok 
    \begin{flalign*}
        \mathcal{G}':   S & \rightarrow aSa \mid bSb \mid \varepsilon&& \\
                        C & \rightarrow a&& \\
                        E & \rightarrow \varepsilon&&
    \end{flalign*}

Pro novou gramatiku $\mathcal{G}' = \{V, \Sigma, S, P'\}$ sestrojíme indukcí množinu

$U = \{A \mid A \in V, \exists \alpha, \beta \in (V \cup \Sigma)^\star \text{ tak, že } S \Rightarrow_\mathcal{G} ^\star \alpha A \beta\}$.

$U_0 = \{S\}$, $S \Rightarrow ^\star S$

$U_1 = U_0 \cup \{X \mid X \text{ se vyskytuje v } \alpha \text{ pro pravidlo } Y \rightarrow \alpha \in P, Y \in U_0\} = \{S\} \cup \emptyset$\\

Redukovaná gramatika je $\mathcal{G}'':  S \rightarrow aSa \mid bSb \mid \varepsilon$.

% --------------------------------------------------------------
%     You don't have to mess with anything below this line.
% --------------------------------------------------------------
 
\end{document}
