% --------------------------------------------------------------
% This is all preamble stuff that you don't have to worry about.
% Head down to where it says "Start here"
% --------------------------------------------------------------
 
\documentclass[11pt]{article}
 
\usepackage[margin=1in]{geometry}
\usepackage{amsmath,amsthm,amssymb,mathtools}
\usepackage[czech]{babel}

\usepackage{tikz}
\usetikzlibrary{automata, positioning, arrows, calc}

\usepackage{tabularray}
\usepackage{bm}
\usepackage{amsmath}
\usepackage{ragged2e} 
\usepackage{booktabs, tabularx}
\usepackage{makecell}
\usepackage{nicematrix,tikz}
\usepackage{multicol}
\usepackage{tikz-qtree}
\usepackage[shortlabels]{enumitem}
 
\begin{document}


\tikzset{
->, % makes the edges directed
>=stealth', % makes the arrow heads bold
node distance=3cm, % specifies the minimum distance between two nodes. Change if necessary.
every state/.style={thick, fill=gray!10}, % sets the properties for each ’state’ node
initial text=$ $, % sets the text that appears on the start arrow
between/.style args={#1 and #2}{
         at = ($(#1)!0.5!(#2)$)
    },
}

\newcommand\splitpage[2]{
      \begin{minipage}[t]{0.45\textwidth}#1
      \end{minipage}%
      \hfill
      \begin{minipage}[t]{0.45\textwidth}#2
      \end{minipage}
}
 
% --------------------------------------------------------------
%                         Start here
% --------------------------------------------------------------
 
\title{\textbf{Osmá samostatná práce}}
\author{Jakub Adamec\\ %replace with your name
B4B01JAG} %if necessary, replace with your course title

\maketitle

\noindent
\textbf{Příklad 12.5.}\\
Je dán jazyk $L$. Sestrojte zásobníkové automaty $A, B$ tak, že $L = N(A)$ a $L = L(B)$, (tj. $A$ přijímá $L$ prázdným 
zásobníkem, $B$ přijímá $L$ koncovým stavem), kde
\[L = \{w \in \{a,b\}^\star \mid |w|_a = |w|_b -1\}\text{.}\]

Přímou metodou:

\begin{multicols}{2}
      \begin{tikzpicture}
            \node[state, initial] (0) {$q_0$};
            \node[state, right of=0] (1) {$q_1$};
            \node[state, accepting, right of=1] (f) {$q_f$};
        
            \draw
                (0) edge[loop above] node[align=center]{$a, Z_0 / A Z_0$ \\ $a, A / AA$} (0)
                (0) edge[bend left, above] node{$b, Z_0 / Z_0$} (1)
                (0) edge[bend left, below] node{$b, A / A$} (1)
                (1) edge[loop above] node[align=center]{$a, Z_0 / Z_0$\\$b, Z_0, Z_0$} (1)
                (1) edge[loop below] node[align=center]{$a, A/AA$ \\ $b, B /BB$ \\ $a,B / \varepsilon$ \\ $b,A/\varepsilon$} (1)
                (1) edge[bend left, above] node{$\varepsilon, Z_0 / \varepsilon$} (f)
      
                ;
      \end{tikzpicture}

      \columnbreak
      Zásobníkový automat: $(Q, \Sigma, \Gamma, \delta, q_0, Z_0, F)$,

      $Q = \{q_0, q_1, q_f\}$,

      $\Sigma = \{a, b\}$,

      $\Gamma = \{Z_0, A, B\}$,

      $F = \{q_f\}$.

\end{multicols}

Tento zásobníkový automat přijímá stejné slovo jak prázdným zásobníkem, tak koncovým stavem, protože k vyprázdnění 
zásobníku dojde pouze v přechodu $q_1 \rightarrow q_f$, a to do koncového stavu.

\begin{enumerate}
      \item ZA přijímá každé slovo z jazyka $L$:
      
      Jestliže slovo $w$ splňuje $|w|_a = |w|_b -1$, pak obsahuje alespoň jeden symbol $b$. Tedy ze stavu $q_0$ přejde
      do $q_1$ a při prvním navštívení $q_1$ bude mít na zásobníku $A^i Z_0$, kde $i$ je počet $a$ předcházejících před
      prvním $b$.

      Proto při prvním navštívení $q_1$ bude zbývat přečíst slovo $u$, kde $w = a^i b u$ a $|a^i u|_a > |a^i u|_b$.

      Tedy ve stavu $q_1$ dojde po přečtení celého slova $u$ k situaci, kdy je ZA ve stavu $q_1$, na vrcholu zásobníku
      je $Z_0$ a slovo je přečtené. Proto automat skončí v koncovém stavu s prázdným zásobníkem.

      Tedy slovo je přijato jak koncovým stavem, tak prázdným zásobníkem.

      \item ZA nepřijme slovo $w \not\in L$:
      
      \begin{enumerate}[a)]
            \item Jestliže $w$ neobsahuje $b$, pak ZA skončí v $q_0$ a na zásobníku bude mít $A^{|w|} Z_0$. Tedy není
            přijato ani koncovým stavem, ani prázdným zásobníkem.
            \item Jestliže $w$ obsahuje $b$ a není z jazyka $L$, pak $|w|_a = |w|_b -1$. ZA přejde do $q_1$ poprvé se
            zásobníkem $A^i Z_0$, kde $w=a^i b u$. Protože $w \not\in L$ neplatí $|a^i u|_a = |a^i u|_b$, proto ZA
            skončí práci (při přečtení slova $w$) ve stavu $q_1$, kde na vrcholu zásobníku je buď $A$ ($|a^i u|_a > |a^i u|_b$)
            nebo B ($|a^i u|_a < |a^i u|_b$). Proto $w$ není přijato.

            I kdyby během práce nad slovem $w$ se automat dostal do situace, kdy je ve stavu $q_1$ a na vrcholu zásobníku
            je $Z_0$, pak i při přechodu do $q_f$, slovo přijato není, protože nebylo celé přečteno. 
      \end{enumerate}
\end{enumerate}

% --------------------------------------------------------------
%     You don't have to mess with anything below this line.
% --------------------------------------------------------------
 
\end{document}
