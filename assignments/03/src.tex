% --------------------------------------------------------------
% This is all preamble stuff that you don't have to worry about.
% Head down to where it says "Start here"
% --------------------------------------------------------------
 
\documentclass[11pt]{article}
 
\usepackage[margin=1in]{geometry}
\usepackage{amsmath,amsthm,amssymb}
\usepackage[czech]{babel}

\usepackage{tikz}
\usetikzlibrary{automata, positioning, arrows, calc}

\usepackage{tabularray}

\newcommand{\N}{\mathbb{N}}
\newcommand{\Z}{\mathbb{Z}}
 
\newenvironment{theorem}[2][Theorem]{\begin{trivlist}
\item[\hskip \labelsep {\bfseries #1}\hskip \labelsep {\bfseries #2.}]}{\end{trivlist}}
\newenvironment{lemma}[2][Lemma]{\begin{trivlist}
\item[\hskip \labelsep {\bfseries #1}\hskip \labelsep {\bfseries #2.}]}{\end{trivlist}}
\newenvironment{exercise}[2][Exercise]{\begin{trivlist}
\item[\hskip \labelsep {\bfseries #1}\hskip \labelsep {\bfseries #2.}]}{\end{trivlist}}
\newenvironment{problem}[2][Problem]{\begin{trivlist}
\item[\hskip \labelsep {\bfseries #1}\hskip \labelsep {\bfseries #2.}]}{\end{trivlist}}
\newenvironment{question}[2][Question]{\begin{trivlist}
\item[\hskip \labelsep {\bfseries #1}\hskip \labelsep {\bfseries #2.}]}{\end{trivlist}}
\newenvironment{corollary}[2][Corollary]{\begin{trivlist}
\item[\hskip \labelsep {\bfseries #1}\hskip \labelsep {\bfseries #2.}]}{\end{trivlist}}

\newenvironment{solution}{\begin{proof}[Solution]}{\end{proof}}
 
\begin{document}


\tikzset{
->, % makes the edges directed
>=stealth', % makes the arrow heads bold
node distance=3cm, % specifies the minimum distance between two nodes. Change if necessary.
every state/.style={thick, fill=gray!10}, % sets the properties for each ’state’ node
initial text=$ $, % sets the text that appears on the start arrow
between/.style args={#1 and #2}{
         at = ($(#1)!0.5!(#2)$)
    },
}

\newcommand\splitpage[2]{
      \begin{minipage}[t]{0.45\textwidth}#1
      \end{minipage}%
      \hfill
      \begin{minipage}[t]{0.45\textwidth}#2
      \end{minipage}
  }
 
% --------------------------------------------------------------
%                         Start here
% --------------------------------------------------------------
 
\title{\textbf{Třetí samostatná práce}}%replace X with the appropriate number
\author{Jakub Adamec\\ %replace with your name
B4B01JAG} %if necessary, replace with your course title

\maketitle

\textbf{Příklad 5.6.} Pro daný NFA sestrojte podmnožinovou konstsrukcí DFA a výsledek redukujte.

\[
    \begin{tabular}{|r|c c|}
        \hline
        & $a$ & $b$\\
        \hline
        \hline
        $\leftrightarrow A$ & $\{A, C\}$ & $\{B\}$\\
        $B$ & $\emptyset$ & $\{B, D\}$\\
        $\leftarrow C$ & $\emptyset$ & $\emptyset$\\
        $\rightarrow D$ & $\{A\}$ & $\{C, D\}$\\
        \hline
    \end{tabular}
\]

\begin{enumerate}
    \item Nakreslete stavový diagram tohoto NFA.
    \item Podmnožinovou konstrukcí zkonstruujte příslušný DFA.
    \item K DFA z bodu 2. sestrojte redukovaný DFA a nakreslete jeho stavový diagram.
\end{enumerate}

\splitpage{
    1.

    \begin{tikzpicture}
        \node[state, initial, accepting] (q1) {$A$};
        \node[state, below of=q1] (q2) {$B$};
        \node[state, initial, right of=q2] (q4) {$D$};
        \node[state, accepting, right of=q1] (q3) {$C$};
        
        \draw 
            (q1) edge[loop above] node{$a$} (q1)
            (q1) edge[bend left, above] node{$a$} (q3)
            (q1) edge[bend right] node[left]{$b$} (q2)
            (q2) edge[loop below] node{$b$} (q2)
            (q2) edge[bend right, below] node{$b$} (q4)
            (q4) edge[loop below] node{$b$} (q4)
            (q4) edge[bend right] node[right]{$b$} (q3)
            (q4) edge node[right]{$a$} (q1)
            ;
    \end{tikzpicture}
}{
3. 

    \begin{tikzpicture}
        \node[state, initial, accepting] (q1) {$K$};
        \node[state, right of=q1] (q2) {$A$};
        \node[state, below of=q2] (q3) {$C$};
        \node[state, left of=q3] (q4) {$B$};
        \node[state, right of=q3] (q5) {$0$};
        
        \draw 
            (q1) edge[loop above] node{$b$} (q1)
            (q1) edge[bend left, above] node{$a$} (q2)
            (q2) edge[loop above] node{$a$} (q2)
            (q2) edge[bend left, above] node[right]{$b$} (q3)
            %(q3) edge[loop below] node{$a$} (q3)
            (q3) edge[bend left, below] node{$b$} (q4)
            (q3) edge node[above]{$a$} (q5)
            (q4) edge[above] node{$a$} (q2)
            (q4) edge[bend left, above] node[left]{$b$} (q1)
            (q5) edge[loop right] node{$a, b$} (q5)
            ;
    \end{tikzpicture}
}

2.
\[
\begin{tabular}{|r|c c|c|c c|c|c c|c|c c|c|}
    \hline
    & $a$ & $b$ & $\sim_0$ & $a$ & $b$ & $\sim_1$ & $a$ & $b$ & $\sim_2$ & $a$ & $b$ & $\sim_3$\\
    \hline
    \hline
    $\leftrightarrow \{A,D\}$& $\{A,C\}$  & $\{B,C,D\}$ & $K$ & $K$ & $K$ & $K$ & $A$ & $K$ & $K$ & $A$ & $K$ & $K$\\
    $\leftarrow \{A,C\}$     & $\{A,C\}$  & $\{B\}$     & $K$ & $K$ & $0$ & $A$ & $A$ & $0$ & $A$ & $A$ & $C$ & $A$\\
    $\leftarrow \{B,C,D\}$   & $\{A\}$    & $\{B,C,D\}$ & $K$ & $K$ & $K$ & $K$ & $A$ & $K$ & $K$ & $A$ & $K$ & $K$\\
    $\{B\}$                  & $\emptyset$& $\{B,D\}$   & $0$ & $0$ & $0$ & $0$ & $0$ & $B$ & $0$ & $0$ & $B$ & $C$\\
    $\leftarrow \{A\}$       & $\{A,C\}$  & $\{B\}$     & $K$ & $K$ & $0$ & $A$ & $A$ & $0$ & $A$ & $A$ & $C$ & $A$\\
    $\{B,D\}$                & $\{A\}$    & $\{B,C,D\}$ & $0$ & $K$ & $K$ & $B$ & $A$ & $K$ & $B$ & $A$ & $K$ & $B$\\
    $\emptyset$              &$\emptyset$ &$\emptyset$  & $0$ & $0$ & $0$ & $0$ & $0$ & $0$ & $0$ & $0$ & $0$ & $0$\\ 
    \hline
\end{tabular}
\]


\pagebreak 
\textbf{Příklad 5.7.} Je dán nedeterministický automat s $\varepsilon$ přechody tabulkou

\[
    \begin{tabular}{|r|c c c|}
        \hline
        & $\varepsilon$ & $a$ & $b$\\
        \hline
        \hline
        $\rightarrow 1$ & $\emptyset$ & $\emptyset$ & $\{4\}$\\
        $\rightarrow 2$ & $\emptyset$ & $\{5\}$ & $\{1, 4\}$\\
        $3$ & $\{4\}$ & $\emptyset$ & $\{6\}$\\
        $\leftarrow 4$ & $\{6\}$ & $\{2, 3\}$ & $\{6\}$\\
        $5$ & $\emptyset$ & $\emptyset$ & $\{3, 6\}$\\
        $\leftarrow 6$ & $\emptyset$ & $\{5\}$ & $\emptyset$\\
        \hline
    \end{tabular}
\]

\begin{enumerate}
    \item Nakreslete stavový diagram tohoto NFA.
    \item Podmnožinovou konstrukcí zkonstruujte příslušný DFA.
    \item K DFA z bodu 2. sestrojte redukovaný DFA a nakreslete jeho stavový diagram.
\end{enumerate}

\splitpage{

1. 

\begin{tikzpicture}
    \node[state, initial] (q1) {$1$};
    \node[state, accepting, right of=q1] (q4) {$4$};
    \node[state, accepting, right of=q4] (q6) {$6$};
    %\node[state, below of=between=q1 and q4] (q2) {$2$};
    \node[state, initial, below of=q1] (q2) {$2$};
    \node[state, below of=q4] (q3) {$3$};
    \node[state, below of=q6] (q5) {$5$};
    
    \draw 
        (q1) edge[bend left, above] node{$b$} (q4)
        (q4) edge[bend left, above] node{$\varepsilon, b$} (q6)
        (q2) edge[bend left] node[left]{$b$} (q1)
        (q2) edge[bend left, above] node{$b$} (q4)
        (q2) edge[bend right, below] node{$a$} (q5)
        (q4) edge[below] node{$a$} (q2)
        (q4) edge[bend right, below] node[right]{$a$} (q3)
        (q3) edge[bend right] node[left]{$\varepsilon$} (q4)
        (q3) edge[above] node{$b$} (q6)
        (q5) edge[above] node{$b$} (q3)
        (q5) edge[bend right, above] node[right]{$b$} (q6)
        (q6) edge[bend right, below] node[right]{$a$} (q5)
        ;
\end{tikzpicture}
}{
3.

\begin{tikzpicture}
    \node[state, initial] (q1) {$A$};
    \node[state, right of=q1](q2){$C$}
    \node[state, right of=q2](q3){$0$}
    \node[state, below of=q1](q4){$E$}
    \node[state, right of=q4](q5){$D$}
    \node[state, right of=q5](q6){$B$}
    \node[state, accepting, below of=q4](q7){$K$}
    
    \draw 
        (q1) edge[bend left, above] node{$a$}(q2)
        (q1) edge[bend right, below] node[left]{$b$}(q4)
        (q2) edge[bend left, above] node{$a$}(q3)
        (q2) edge[bend right, below] node[left]{$a$}(q5)
        (q3) edge[loop above] node{$a, b$}(q3)
        (q4) edge[bend right, above] node{$b$}(q5)
        (q4) edge[bend right, below] node[left]{$a$}(q7)
        (q5) edge[bend right, below] node{$b$} (q6)
        (q5) edge[bend left, below] node{$a$} (q7)
        (q6) edge[bend right, above] node[right]{$b$}(q3)
        (q6) edge[left, above] node{$a$}(q2)
        (q7) edge[loop below] node{$a$}(q7)
        (q7) edge[bend right, above] node[right]{$b$}(q4)
        ;
\end{tikzpicture}
}

2.

\[
\begin{tabular}{|r|c c|c|c c|c|c c|c|c c|c|c c|c|}
    \hline
    & $a$ & $b$ & $\sim_0$ & $a$ & $b$ & $\sim_1$ & $a$ & $b$ & $\sim_2$ & $a$ & $b$ & $\sim_3$ & $a$ & $b$ & $\sim_4$\\
    \hline
    \hline
    $\rightarrow \{$1, 2$\}$    & $\{5\}$        & $\{1,4,6\}$    & $0$ & $0$ & $K$ & $A$ & $A$ & $K$ & $A$ & $C$ & $K$ & $A$ & $C$ & $E$ & $A$\\
    $\{$5$\}$                   & $\{\emptyset\}$& $\{3,4,6\}$    & $0$ & $0$ & $K$ & $A$ & $0$ & $K$ & $C$ & $0$ & $D$ & $C$ & $0$ & $D$ & $C$\\
    $\leftarrow \{$1, 4, 6$\}$  & $\{2,3,4,5,6\}$& $\{4,6\}$      & $K$ & $K$ & $K$ & $K$ & $K$ & $K$ & $K$ & $K$ & $D$ & $E$ & $K$ & $D$ & $E$\\
    $\leftarrow \{$3,4,6$\}$    & $\{2,3,4,5,6\}$& $\{6\}$        & $K$ & $K$ & $K$ & $K$ & $K$ & $B$ & $D$ & $K$ & $B$ & $D$ & $K$ & $B$ & $D$\\
    $\leftarrow \{$2,3,4,5,6$\}$& $\{2,3,4,5,6\}$& $\{1,3,4,6\}$  & $K$ & $K$ & $K$ & $K$ & $K$ & $K$ & $K$ & $K$ & $K$ & $K$ & $K$ & $E$ & $K$\\
    $\leftarrow \{$4,6$\}$      & $\{2,3,4,5,6\}$& $\{6\}$        & $K$ & $K$ & $K$ & $K$ & $K$ & $B$ & $D$ & $K$ & $B$ & $D$ & $K$ & $B$ & $D$\\
    $\leftarrow \{$6$\}$        & $\{5\}$        & $\{\emptyset\}$& $K$ & $0$ & $0$ & $B$ & $A$ & $0$ & $B$ & $C$ & $0$ & $B$ & $C$ & $0$ & $B$\\
    $\leftarrow\{$1,3,4,6$\}$   & $\{2,3,4,5,6\}$& $\{4,6\}$      & $K$ & $K$ & $K$ & $K$ & $K$ & $K$ & $K$ & $K$ & $D$ & $E$ & $K$ & $D$ & $E$\\
    $\emptyset$                 &$\emptyset$     &$\emptyset$     & $0$ & $0$ & $0$ & $0$ & $0$ & $0$ & $0$ & $0$ & $0$ & $0$ & $0$ & $0$ & $0$\\ 
    \hline
\end{tabular}
\]


% --------------------------------------------------------------
%     You don't have to mess with anything below this line.
% --------------------------------------------------------------
 
\end{document}

