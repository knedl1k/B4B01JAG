\section{Druhé cvičení}

\subsection{Návrh DFA dle jazyka}

Je dán jazyk $L = \{w \mid \text{počet } |w|_a \text{ je sudý}\}$ nad abecedou $\Sigma = \{a,b\}$. Navrhněte konečný 
automat přijímající jazyk $L$ a dokažte, že skutečně tento jazyk přijímá. 

\begin{multicols}{2}
    \begin{tikzpicture}
        \node[state, initial, accepting] (q0) {$q_0$};
        \node[state, right of=q0] (q1) {$q_1$};
    
        \draw 
            (q0) edge[bend left, above] node{$a$} (q1)    
            (q0) edge[loop above] node{$b$} (q0)
            (q1) edge[bend left, below] node{$a$} (q0)
            (q1) edge[loop above] node{$b$} (q1)
            ;
    \end{tikzpicture}
    
    Důkaz pomocí invariantů: 
    \begin{itemize}[noitemsep]
        \item $q_0$: $|w|_a = 2k$, \texttt{in}, \texttt{out}, 
        \item $q_1$: $|w|_a = 2k + 1$,  
    \end{itemize}
    pro $k \in \mathbb{Z}$.

\end{multicols}

\subsection{Návrh DFA dle jazyka}
Je dán jazyk $L = \{w \mid \text{počet } |w|_a \text{ je sudý a počet } |w|_b \text{ je lichý}\}$ nad abecedou 
$\Sigma = \{a,b\}$. Navrhněte konečný automat přijímající jazyk $L$ a dokažte, že skutečně tento jazyk přijímá. 

\begin{multicols}{2}
    
    \begin{tikzpicture}
        \node[state, initial] (q0) {$q_0$};
        \node[state, right of=q0] (q1) {$q_1$};
        \node[state, accepting, below of=q0] (q2) {$q_2$};
        \node[state, right of=q2] (q3) {$q_3$};
        
        \draw 
        (q0) edge[bend left, above] node{$a$} (q1)    
        (q0) edge[bend left, right] node{$b$} (q2)
        (q1) edge[bend left, below] node{$a$} (q0)
        (q1) edge[bend left, right] node{$b$} (q3)
        (q2) edge[bend left, below] node{$a$} (q3)
        (q2) edge[bend left, left] node{$b$} (q0)
        (q3) edge[bend left, above] node{$a$} (q2)
        (q3) edge[bend left, left] node{$b$} (q1)
        ;
    \end{tikzpicture}

    Důkaz pomocí invariantů: 
    \begin{itemize}[noitemsep]
        \item $q_0$: $|w|_a = 2k$, $|w|_b = 2k$, \texttt{in}, 
        \item $q_1$: $|w|_a = 2k + 1$, $|w|_b = 2k$,  
        \item $q_2$: $|w|_a = 2k$, $|w|_b = 2k + 1$, \texttt{out},  
        \item $q_3$: $|w|_a = 2k + 1$ $|w|_b = 2k+ 1$,  
    \end{itemize}
    pro $k \in \mathbb{Z}$.
    

\end{multicols}

\subsection{Návrh DFA dle jazyka}
Pro daný jazyk $L$ navrhněte konečný automat, který tento jazyk přijímá. O automatu ukažte, že opravdu přijímá daný jazyk.

\begin{enumerate}[a), noitemsep]
    \item $\Sigma = \{a,b\}$, jazyk $L$ obsahuje právě všechna slova, která končí $b$ a mají délku $3k+1$.
    \item $\Sigma = \{0,1\}$, jazyk $L$ obsahuje právě všechna slova, která obsahují podslovo $0101$.
    \item $\Sigma = \{0,1\}$, jazyk $L$ obsahuje právě všechna slova, jejichž každé podslovo délky $3$ obsahuje znak $0$.
\end{enumerate}

\newpage 
\noindent

a)
\begin{multicols}{2}
    
    \begin{tikzpicture}
        \node[state, initial] (q0) {$q_0$};
        \node[state, right of=q0] (q1) {$q_1$};
        \node[state, accepting, above of=q1] (q2) {$q_2$};
        \node[state, right of=q1] (q3) {$q_3$};

        \draw 
            (q0) edge[bend left, above] node{$a$} (q1)
            (q0) edge[bend left, above] node{$b$} (q2)
            (q1) edge[bend left, above] node{$a,b$} (q3)
            (q2) edge[bend left, right] node{$a,b$} (q3)
            (q3) edge[bend left, below] node{$a,b$} (q0)
            ;
    \end{tikzpicture}

    Důkaz pomocí invariantů: 
    \begin{itemize}[noitemsep]
        \item $q_0$: končí $a,b$, $|w| = 3k$, \texttt{in}, 
        \item $q_1$: končí $a$, $|w| = 3k+1$, 
        \item $q_2$: končí $b$, $|w| = 3k+1$, \texttt{out}, 
        \item $q_3$: končí $a,b$, $|w| = 3k+2$, $k \in \mathbb{Z}$.
    \end{itemize}
\end{multicols}

\noindent
b)

\begin{tikzpicture}
    \node[state, initial] (q0) {$q_0$};
    \node[state, right of=q0] (q1) {$q_1$};
    \node[state, right of=q1] (q2) {$q_2$};
    \node[state, right of=q2] (q3) {$q_3$};
    \node[state, accepting, right of=q3] (q4) {$q_4$};

    \draw 
        (q0) edge[bend left, above] node{$0$} (q1)
        (q0) edge[loop above] node{$1$} (q0)
        (q1) edge[loop above] node{$0$} (q1)
        (q1) edge[bend left, above] node{$1$} (q2)
        (q2) edge[bend left, above] node{$0$} (q3)
        (q2) edge[bend left, below] node{$1$} (q0)
        (q3) edge[bend left, below] node{$0$} (q1)
        (q3) edge[bend left, above] node{$1$} (q4)
        (q4) edge[loop below] node{$0,1$} (q4)
        ;
\end{tikzpicture}


Důkaz pomocí invariantů: 
\begin{itemize}[noitemsep]
    \item $q_0$: $\varepsilon$, končí $1$, nekončí $01$, neobsahuje $010$, \texttt{in}, 
    \item $q_1$: končí $0$, nekončí $010$, neobsahuje $0101$,
    \item $q_2$: končí $01$, neobsahuje $0101$,
    \item $q_3$: končí $010$, neobsahuje $0101$,
    \item $q_4$: obsahuje $0101$, \texttt{out}.
\end{itemize}


\noindent
c) řešíme doplňkem, tedy jazyk $\mathcal{L}$ obsahuje právě všechna slova, jejižch každé podslovo délky $3$ \textbf{neobsahuje} 
znak $0$. Následně všechny stavy, které nebyly výstupní, se stanou výstupními a stavy, které byly výstupní, se stanou 
nevýstupními. 

\begin{tikzpicture}
    \node[state, accepting, initial] (q0) {$q_0$};
    \node[state, accepting, right of=q0] (q1) {$q_1$};
    \node[state, accepting, right of=q1] (q2) {$q_2$};
    \node[state, right of=q2] (q3) {$q_3$};

    \draw 
        (q0) edge[loop above] node{$0$} (q0)
        (q0) edge[bend left, above] node{$1$} (q1)
        (q1) edge[bend left, above] node{$0$} (q0)
        (q1) edge[bend left, above] node{$1$} (q2)
        (q2) edge[bend left, below] node{$0$} (q0)
        (q2) edge[bend left, above] node{$1$} (q3)
        (q3) edge[loop above] node{$0,1$} (q3)
        ;
\end{tikzpicture}

Důkaz pomocí invariantů: 
\begin{itemize}[noitemsep]
    \item $q_0$: končí $0$, neobsahuje $111$, \texttt{in}, \texttt{out},
    \item $q_1$: končí $1$, neobsahuje $111$, \texttt{out},
    \item $q_2$: končí $11$, neobsahuje $111$, \texttt{out},
    \item $q_3$: obsahuje $111$. 
\end{itemize}

\subsection{Nerodova věta a Pumping lemma}
Pomocí Nerodovy věty a Pumping lemmatu dokažte, že jazyk $L = \{0^i 1^j 0^k \mid 0 \leq i < j <k\}$ není regulární.\\

\noindent
Definice \textbf{Pumping lemmatu.} Když $L$ je regulární, tak existuje $n \geq 0$ takové, že každé ${u \in L}$,
$|u| > n$, je možné rozložit na 3 slova $u = xwy$ splňující:

\begin{enumerate}[1), noitemsep]
    \item $|xw| \leq n$
    \item $w \not= \varepsilon$
    \item $xw^i y \in L, \forall i = 0, 1, 2, ...$
\end{enumerate}
Důkaz sporem:
Kdyby $L$ byl regulární, tak existuje $n$ s vlastnostmi z Pumping lemma.

Zvolíme $u = 0^n 1^{n+1} 0^{n+2}$.

Pak 1) vlastnost říká, že $xw = 0^l, l \leq n$. Zároveň musí platit 2), tedy $w = 0^k, 1 \leq k \leq l$.

Když teď napumpujeme $xw^i y$, například $i=2$, dostaneme $xw^2 y = 0^{n+k} 1^{n+1} 0^{n+2} \not\in L$.

Tedy $L$ není regulární. $\blacksquare$\\

\noindent
Definice \textbf{Nerodovy věty.} $L$ je regulární iff existuje ekvivalence $T$ na $\Sigma^\star$ taková, že:
\begin{enumerate}[1), noitemsep]
    \item $L$ je sjednocení některých tříd $T$
    \item pokud $uTv$, tak $uwTvw$ pro každé $w \in \Sigma^\star$
    \item $T$ má konečný počet tříd
\end{enumerate}
Důkaz sporem:
Kdyby existovala $T$.

$1, 1^2, 1^3, ..., 1^i, ..., 1^n, ... = \{1^j \mid j \geq 1\}$ je nekonečná posloupnost $0$ a $1$. 

$T$ musí mít konečně mnoho tříd, proto musí existovat $i > j, i \not= j \land 1^i T 1^j$.

Zvolíme $w = 0^{j + 1}$.

Pak podle vlastnosti 2) $\underbrace{1^i 0^{j+1}}_{\substack{i \geq j+1 \\ \not\in L}} T 
\underbrace{1^j 0^{j+1}}_{\substack{j < j+1 \\\in L }}$.

Tedy $L$ není regulární. $\blacksquare$