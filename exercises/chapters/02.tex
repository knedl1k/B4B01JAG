\section{Druhé cvičení}

\subsection{Příklad}

Je dán jazyk $L = \{w \mid \text{počet } |w|_a \text{ je sudý}\}$ nad abecedou $\Sigma = \{a,b\}$. Navrhněte konečný 
automat přijímající jazyk $L$ a dokažte, že skutečně tento jazyk přijímá. 

\begin{tikzpicture}
    \node[state, initial, accepting] (q0) {$q_0$};
    \node[state, right of=q0] (q1) {$q_1$};

    \draw 
        (q0) edge[bend left, above] node{$a$} (q1)    
        (q0) edge[loop above] node{$b$} (q0)
        (q1) edge[bend left, below] node{$a$} (q0)
        (q1) edge[loop above] node{$b$} (q1)
        ;
\end{tikzpicture}
%TODO invarianty

\subsection{Příklad}
Je dán jazyk $L = \{w \mid \text{počet } |w|_a \text{ je sudý a počet } |w|_b \text{ je lichý}\}$ nad abecedou 
$\Sigma = \{a,b\}$. Navrhněte konečný automat přijímající jazyk $L$ a dokažte, že skutečně tento jazyk přijímá. 

\begin{tikzpicture}
    \node[state, initial] (q0) {$q_0$};
    \node[state, right of=q0] (q1) {$q_1$};
    \node[state, accepting, below of=q0] (q2) {$q_2$};
    \node[state, right of=q2] (q3) {$q_3$};

    \draw 
        (q0) edge[bend left, above] node{$a$} (q1)    
        (q0) edge[bend left, right] node{$b$} (q2)
        (q1) edge[bend left, below] node{$a$} (q0)
        (q1) edge[bend left, right] node{$b$} (q3)
        (q2) edge[bend right, below] node{$a$} (q3)
        (q2) edge[bend left, left] node{$b$} (q0)
        (q3) edge[bend right, above] node{$a$} (q2)
        (q3) edge[bend left, left] node{$b$} (q1)
        ;
\end{tikzpicture}
%TODO invarianty

\subsection{Příklad}
Pro daný jazyk $L$ navrhněte koečný automat, který tento jazyk přijímá. O automatu ukažte, že opravdu přijímá daný jazyk.

\begin{enumerate}[a), noitemsep]
    \item $\Sigma = \{a,b\}$, jazyk $L$ obsahuje právě všechna slova, která končí $b$ a mají délku $3k+1$.
    \item $\Sigma = \{0,1\}$, jazyk $L$ obsahuje právě všechna slova, která obsahují podslovo $0101$.
    \item $\Sigma = \{0,1\}$, jazyk $L$ obsahuje právě všechna slova, jejichž každé podslovo délky $3$ obsahuje znak $0$.
\end{enumerate}

\newpage %! THIS NEEDS CHECK AFTER COMPLETING TODOs
\noindent
a)

\begin{tikzpicture}
    \node[state, initial] (q0) {$q_0$};
    \node[state, right of=q0] (q1) {$q_1$};
    \node[state, accepting, above of=q1] (q2) {$q_2$};
    \node[state, right of=q1] (q3) {$q_3$};

    \draw 
        (q0) edge[bend left, above] node{$a$} (q1)
        (q0) edge[bend left, above] node{$b$} (q2)
        (q1) edge[bend left, above] node{$a,b$} (q3)
        (q2) edge[bend left, right] node{$a,b$} (q3)
        (q3) edge[bend left, below] node{$a,b$} (q0)
        ;
\end{tikzpicture}
%TODO invarianty
 
\noindent
b)

\begin{tikzpicture}
    \node[state, initial] (q0) {$q_0$};
    \node[state, right of=q0] (q1) {$q_1$};
    \node[state, right of=q1] (q2) {$q_2$};
    \node[state, right of=q2] (q3) {$q_3$};
    \node[state, accepting, right of=q3] (q4) {$q_4$};

    \draw 
        (q0) edge[bend left, above] node{$0$} (q1)
        (q0) edge[loop above] node{$1$} (q0)
        (q1) edge[bend left, above] node{$0$} (q0)
        (q1) edge[bend left, above] node{$1$} (q2)
        (q2) edge[bend left, above] node{$0$} (q3)
        (q2) edge[bend left, below] node{$1$} (q0)
        (q3) edge[bend left, below] node{$0$} (q1)
        (q3) edge[bend left, above] node{$1$} (q4)
        (q4) edge[loop below] node{$0,1$} (q4)
        ;
\end{tikzpicture}
%TODO invarianty

\noindent
c)

\begin{tikzpicture}
    \node[state, accepting, initial] (q0) {$q_0$};
    \node[state, accepting, right of=q0] (q1) {$q_1$};
    \node[state, accepting, right of=q1] (q2) {$q_2$};
    \node[state, right of=q2] (q3) {$q_3$};

    \draw 
        (q0) edge[loop above] node{$0$} (q0)
        (q0) edge[bend left, above] node{$1$} (q1)
        (q1) edge[bend left, above] node{$0$} (q0)
        (q1) edge[bend left, above] node{$1$} (q2)
        (q2) edge[bend left, below] node{$0$} (q0)
        (q2) edge[bend left, above] node{$1$} (q3)
        (q3) edge[loop above] node{$0,1$} (q3)
        ;
\end{tikzpicture}
%TODO invarianty

\newpage %! THIS NEEDS CHECK AFTER COMPLETING TODOs
\subsection{Příklad}
Pomocí Nerodovy věty a Pumping lemmatu dokažte, že jazyk $L = \{0^i 1^j 0^k \mid 0 \leq i < j <k\}$ není regulární.
\\

\noindent
Definice \textbf{Pumping lemma.} Když $L$ je regulární, tak existuje $n \in L, n \geq 1$, takové, že každé ${u \in L},\\
|w| > n$ je 
možné rozložit na 3 slova splňující:
\begin{enumerate}[1), noitemsep]
    \item $|xw| \leq n$
    \item $w \not= \varepsilon$
    \item $xw^i y \in L, \forall i = 0, 1, 2, ...$
\end{enumerate}
Důkaz:
Kdyby $L$ byl regulární, tak existuje $n$ s vlastnostmi z Pumping lemma.

Zvolíme $u = 0^n 1^{n+1} 0^{n+2}$.

Pak 1) vlastnost říká, že $xw = 0^l, l \leq n$. Zároveň musí platit 2), tedy $w = 0^k, 1 \leq k \leq l$.

Když teď napumpujeme $xw^i y$, například $i=2$, dostaneme $xw^2 y = 0^{n+k} 1^{n+1} 0^{n+2} \not\in L$.

Tedy $L$ není regulární. $\blacksquare$
\\

\noindent
Definice \textbf{Nerodovy věty.} $L$ je regulární iff existují ekvivalence $T$ na $\Sigma^\star$ taková, že:
\begin{enumerate}[1), noitemsep]
    \item $L$ je sjednocení některých tříd $T$
    \item pokud $uTv$, tak $uwTvw$ pro každé $w \in \Sigma^\star$
    \item $T$ má konečný počet tříd
\end{enumerate}
Důkaz:
Kdyby existovala $T$.

$1, 1^2, 1^3, ..., 1^i, ..., 1^n, ... = \{1^j \mid j \geq 1\}$ je nekonečná posloupnost $0$ a $1$. 

$T$ musí mít konečně mnoho tříd, proto musí existovat $i > j, i \not= j \land 1^i T 1^j$.

Zvolíme $w = 0^{j + 1}$.

Pak podle vlastnosti 2) $\underbrace{1^i 0^{j+1}}_{\substack{i \geq j+1 \\ \not\in L}} T 
\underbrace{1^j 0^{j+1}}_{\substack{j < j+1 \\\in L }}$.

Tedy $L$ není regulární. $\blacksquare$