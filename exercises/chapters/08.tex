\section{Osmé cvičení}

\subsection{Konstrukce regulární gramatiky k automatu}
K automatu $M$, který je dán následující tabulkou, zkostruujte regulární gramatiku $\G$, která generuje jazyk
$L = L(M)$.

$M$: \hspace{2mm}
\begin{multicols}{2}

    \begin{tabular}{|r c|c c|}
        \hline
        & & $a$ & $b$ \\
        \hline
        \hline
        $\leftrightarrow$&$ A$ & $A,C$ & $B$ \\
        \hline
        &$ B$ & $\emptyset$ & $B, D$ \\
        \hline
        $\leftarrow$ &$ C$ & $\emptyset$ & $\emptyset$ \\
        \hline
        $\rightarrow$&$D$ & $A$ & $C,D$ \\
        \hline
    \end{tabular}

    \vspace*{5mm}
    $\G = (N, \Sigma, S, P)$ \\
    $ N = \bc{S, A, B, C, D}$\\
    $\Sigma = \bc{a, b}$

\columnbreak
    mám více vstupů $\rightarrow$ přidám si $S$

    \begin{tikzpicture}[shorten >=1pt,node distance=20mm,on grid,auto]
        % \tikzstyle{every state}=[fill={rgb:black,1;white,10}, minimum size=18pt, inner sep=1pt]
        \node[state, initial] (S) {$S$};
        \node[state, accepting] (A) [right  of=S] {$A$};
        \node[state] (B) [below  of=A] {$B$};
        \node[state, accepting] (C) [right  of=A] {$C$};
        \node[state] (D) [below of=S] {$D$};

        \path[->]
        (S) edge [bend left] node {$\varepsilon$} (A)
        (S) edge [bend right] node {$\varepsilon$} (D)
        (A) edge [loop above] node {$a$} ()
        (A) edge [bend left] node {$a$} (C)
        (A) edge [bend left] node {$b$} (B)
        (B) edge [bend left] node {$b$} (D)
        (B) edge [loop right] node {$b$} ()
        (D) edge [loop left] node {$b$} ()
        (D) edge [left] node [left]{$b$} (C)
        (D) edge [left] node {$a$} (A)
        ;
    \end{tikzpicture}
    \begin{flalign*}
        P: & S \rightarrow A \mid D & \\
        & A \rightarrow aA \mid aC \mid bB \mid \varepsilon & \\
        & B \rightarrow bB \mid bD & \\
        & C \rightarrow \varepsilon & \\
        & D \rightarrow aA \mid bC \mid bD & \\
    \end{flalign*}
\end{multicols}

\subsection{Tvorba DFA ke gramatice 3. typu}
Ke gramatice $\G$ typu 3 zkonstruujte konečný automat, který přijímá jazyk $L(\G)$. Gramatika
$\G = (N, \bc{a,b}, S, P)$, kde $N = \bc{S, A, B}$ a pravidla jsou
\begin{align*}
        P\text{: } & S \rightarrow abA \mid aB \\
        & A \rightarrow aA \mid aaA \mid a\\
        & B \rightarrow bB \mid b
\end{align*}

\begin{multicols}{2}
    \[
        \begin{array}{l l l}
            P: & S \rightarrow abA \mid aB & \text{překáží mi } \ii{abA}\\
            & C \rightarrow bA & \\
            & S \rightarrow aC \mid aB & \text{nové pravidlo \ii{S}} \\
            & A \rightarrow aA \mid aaA \mid a & \text{zase nám vadí } \ii{aaA}\\
            & D \rightarrow aA & \\
            & E \rightarrow \varepsilon & \\
            & A \rightarrow aA \mid aD \mid aE & \text{nové pravidlo \ii{A}} \\
            & B \rightarrow bB \mid bE & \text{nové pravidlo \ii{B}} \\
        \end{array}
    \]
\columnbreak

    sestavím automat podle nových pravidel

    \begin{tikzpicture}[shorten >=1pt,node distance=20mm,on grid,auto]
        % \tikzstyle{every state}=[fill={rgb:black,1;white,10}, minimum size=18pt, inner sep=1pt]
        \node[state, initial] (S) {$S$};
        \node[state] (B) [below  of=S] {$B$};
        \node[state] (C) [right  of=S] {$C$};
        \node[state] (A) [right  of=C] {$A$};
        \node[state] (D) [below of=A] {$D$};
        \node[state, accepting] (E) [below of=C] {$E$};

        \path[->]
        (S) edge [bend left] node {$a$} (C)
        (S) edge [bend right] node {$a$} (B)
        (C) edge [bend left] node {$b$} (A)
        (A) edge [loop right] node {$a$} ()
        (A) edge [bend right] node {$a$} (E)
        (A) edge [bend left] node {$a$} (D)
        (B) edge [loop below] node {$b$} ()
        (B) edge [bend right] node {$b$} (E)
        (D) edge [bend left] node [right] {$a$} (A)
        ;
    \end{tikzpicture}
\end{multicols}

\newpage
\subsection{Práce s derivačním stromem bezkontextové gramatiky}\noindent
Je dán derivační strom v bezkontextové gramatice:

\begin{center}
    \begin{forest}
        for tree={
            grow=south,                 % Tree grows downward
            edge={->},                  % Draw edges as arrows
            align=center,               % Center the text inside nodes
        }
        [$S$
            [$S$
                [$S$
                    [$\varepsilon$]
                ]
                [$B$
                    [$B$
                        [$b$]
                    ]
                    [$A$
                        [$a$]
                    ]
                ]
            ]
            [$A$
                [$A$
                    [$a$]
                ]
                [$A$
                    [$a$]
                ]
            ]
        ]
    \end{forest}
\end{center}

\begin{enumerate}[a), noitemsep]
    \item Napište pravidla minimální CF gramatiky, ve které je to derivační strom.
    \item Napište levou derivaci odpovídající tomuto derivačnímu stromu.
    \item Rozhodněte, zda je gramatika víceznačná.
\end{enumerate}

\begin{multicols}{2}

    a) \[
        \begin{array}{l l}
            P: & S \rightarrow SA \mid SB \mid \varepsilon \\
            & A \rightarrow AA \mid a  \\
            & B \rightarrow BA \mid b  \\
        \end{array}
        \]
\columnbreak

        c) je víceznačná - tj. je možné alespoň jedno slovo vygenerovat dvěma způsoby. Například díky pravidlu 
        $A \rightarrow AA$.
        
\end{multicols}

b)
$\quad S \stackrel{S \rightarrow SA}{\Longrightarrow} SA \stackrel{S \rightarrow SB}{\Longrightarrow} SBA
\stackrel{S \rightarrow \varepsilon}{\Longrightarrow} BA \stackrel{B \rightarrow BA}{\Longrightarrow} BAA
\stackrel{B \rightarrow b}{\Longrightarrow} bAA \stackrel{A \rightarrow a}{\Longrightarrow} baA
\stackrel{A \rightarrow AA}{\Longrightarrow} baAA\stackrel{A \rightarrow a}{\Longrightarrow}^{(2)} baaa$.

\subsection{Tvorba derivačního stromu k bezkontextové gramatice} % 8.4
Je dána bezkontextová gramatika $\G = (N, \Sigma, S, P)$, kde $N = \bc{S}$, $\Sigma = \bc{+, \star, -, x, y}$,
s pravidly
$$S \rightarrow +SS \mid \star SS \mid -SS \mid x \mid y $$

\begin{itemize}[noitemsep]
    \item Nakreslete derivační strom, který má za výsledek slovo $ w = + x \star - y x y$.
    \item Zkonstruujte levou derivaci slova $w$ odpovídající derivačnímu stromu z části a).
\end{itemize}

\begin{multicols}{2}
    1.
    \begin{center}
        \begin{forest}
            for tree={
                grow=south,                 % Tree grows downward
                edge={->},                  % Draw edges as arrows
                align=center,               % Center the text inside nodes
            }
            [$S$
                [$+$]
                [$S$ [$x$]]
                [$S$
                    [$*$]
                    [$S$
                        [$-$]
                        [$S$ [$y$]]
                        [$S$ [$x$]]
                    ]
                    [$S$ [$y$]]
                ]
            ]
        \end{forest}
    \end{center}

\columnbreak
    \vspace{-30mm}

        2.

        \begin{align*}
            & S \stackrel{S \rightarrow +SS}{\Longrightarrow} +SS
            \stackrel{S \rightarrow x}{\Longrightarrow} +xS
            \stackrel{S \rightarrow \star SS}{\Longrightarrow} +x\star SS \Longrightarrow\\
            & \stackrel{S \rightarrow -SS}{\Longrightarrow} + x \star - SSS
            \stackrel{S \rightarrow y}{\Longrightarrow} +x\star - y SS \Longrightarrow \\
            & \stackrel{S \rightarrow x}{\Longrightarrow} +x \star - y x S
            \stackrel{S \rightarrow y}{\Longrightarrow} +x \star - y x y
    \end{align*}

\end{multicols}

\subsection{Návrh bezkontextové gramatiky pro jazyk}
Navrhněte bezkontextovou gramatiku $\G$, která generuje jazyk $L = \bc{0^i 1^i 2^j; i, j \geq 0}$. Zdůvodněte,
proč gramatika $\G$ jazyk $L$ generuje.
    \begin{align*}
        P\text{: } & S \rightarrow XY  \\
        & X \rightarrow 0X1 \mid \varepsilon  \\
        & Y \rightarrow Y2 \mid \varepsilon
    \end{align*}

1. $L \subseteq L(\G)$ (gramatika vygeneruje vše):

$\quad S \stackrel{S \rightarrow XY}{\Longrightarrow} XY \stackrel{X \rightarrow 0X1 (i)}{\Longrightarrow} 0^iX1^iY
\stackrel{Y \rightarrow 2Y(j)}{\Longrightarrow} 0^iX1^iY2^j \stackrel{X \rightarrow \varepsilon}{\Longrightarrow}
0^i1^iY2^j\stackrel{Y \rightarrow \varepsilon}{\Longrightarrow} 0^i1^i2^j $.

2. $L(\G) \subseteq L$ (gramatika nevygeneruje nic navíc):

Uvažujme derivaci $S \implies\star w$. Pak poslední použité pravidlo musí být $X \rightarrow \varepsilon$ nebo
$Y \rightarrow \varepsilon$. Proto v derivaci musí být použito pravidlo $S \rightarrow XY$. Mezi tím může být použit
nějaký počet pravidel $X \rightarrow 0X1$ a $Y \rightarrow Y2$. Jinak pravidla být použita nemohou. Tedy drivace má tvar
$ S \stackrel{S \rightarrow XY}{\Longrightarrow} XY \stackrel{X \rightarrow 0X1 (i)}{\Longrightarrow} 0^iX1^iY
\stackrel{Y \rightarrow 2Y(j)}{\Longrightarrow} 0^iX1^iY2^j \stackrel{X \rightarrow \varepsilon}{\Longrightarrow}
0^i1^iY2^j\stackrel{X \rightarrow \varepsilon}{\Longrightarrow} 0^i1^i2^j $.

\subsection{Tvorba nevypouštěcí gramatiky}
Ke gramatice $\G$ zkostruujte nevypouštěcí gramatiku $\G_1$, pro kterou $L(\G_1) = L(\G) - \bc{\varepsilon}$.
    \begin{align*}
        P\text{: } & S \rightarrow aSbA \mid \varepsilon \\
           & A \rightarrow aBbA \mid bCB \mid CD \\
           & B \rightarrow bbBa \mid aS \\
           & C \rightarrow aAaA \mid \varepsilon \\
           & D \rightarrow SC \mid aABa
    \end{align*}

obecný formální zápis u nevypouštěcích gramatik:

$V = \bc{A \mid A \implies^{\star} \varepsilon}$\\
$V_1 = \bc{A \mid A \rightarrow \varepsilon \in P}$\\
$V_2 = V_1 \cup \bc{A \mid A \rightarrow \alpha \in P, \alpha \in V_1^{\star}}$\\
$V_{i+1} = V_i \cup \bc{A \mid A \rightarrow \alpha \in P, \alpha \in V_i^{\star}}$\\

\begin{multicols}{2}
    příklad:

    $V = \bc{A \mid A \implies^{\star} \varepsilon}$\\
    $V_1 = \bc{A \mid A \rightarrow \varepsilon \in P} = \bc{S, C}$\\
    $V_2 = V_1 \cup \bc{A \mid A \rightarrow \alpha \in P, \alpha \in V_1^{\star}} = V_1 \cup \bc{D} = \bc{S, C, D}$\\
    $V_3 = V_2 \cup \bc{A} = \bc{S, A, C, D}$

\columnbreak

    $\G_1$:
        \begin{align*}
            P\text{: } & S \rightarrow aSbA \mid abA \mid aSb \mid ab  \\
            & A \rightarrow aBbA \mid aBb \mid bCB \mid bB \mid CD \mid C \mid D \\
            & B \rightarrow bbBa \mid aS \mid a\\
            & C \rightarrow aAaA \mid aAa \mid aaA \mid aa \\
            & D \rightarrow SC \mid S \mid C \mid aABa \mid aBa \\
        \end{align*}

\end{multicols}

\subsection{Konstrukce gramatiky 3. typu k automatu}
K automatu $M$ zkonstruujte gramatiku typu 3, která generuje jazyk $L(M)$, kde $M$ je dán tabulkou
\begin{multicols}{2}
    $M$: \hspace{2mm}
    \begin{tabular}{|c c||c| c|}
        \hline
        & & $a$ & $b$ \\
        \hline
        $\rightarrow$&$ A $& $\bc{A,B}$ & $\bc{C}$ \\
        &$ B$ & $\bc{B}$ & $\bc{C}$ \\
        $\leftrightarrow$ &$ C$ & $\emptyset$ & $\bc{D}$ \\
        $\leftarrow$&$ D$ & $\bc{B}$ & $\bc{D}$ \\
        \hline
    \end{tabular}

    $\G = (N, \Sigma, S, P)$\\
    $N = \bc{S, A, B, C, D}$\\
    $\Sigma = \bc{a, b}$
    \begin{flalign*}
        P: & S \rightarrow A \mid C & \\
        & A \rightarrow aA \mid aB \mid bC & \\
        & B \rightarrow aB \mid bC & \\
        & C \rightarrow bD \mid \varepsilon & \\
        & D \rightarrow aB \mid bD \mid \varepsilon
    \end{flalign*}

\columnbreak

        \begin{tikzpicture}[shorten >=1pt,node distance=35mm,on grid,auto]
            % \tikzstyle{every state}=[fill={rgb:black,1;white,10}, inner sep=1pt]
            \def\radius{22mm}

            \node[state] (A) at (90:\radius) {$A$}; % top center
            \node[state, initial] (S) at (162:\radius) {$S$}; % top-left
            \node[state] (B) at (18:\radius) {$B$}; % bottom-left
            \node[state, accepting] (D) at (306:\radius) {$D$}; % bottom-right
            \node[state, accepting] (C) at (234:\radius) {$C$}; % top-right
            \path[->]

            (S) edge [bend left=20] node {$\varepsilon$} (A)
            (S) edge [bend right=20] node {$\varepsilon$} (C)

            (A) edge [loop above] node {$a$} ()
            (A) edge [bend left=20] node {$a$} (B)
            (A) edge [left] node {$b$} (C)

            (B) edge [loop above] node {$a$} ()
            (B) edge [above] node {$b$} (C)

            (C) edge [bend right=20] node {$b$} (D)

            (D) edge [loop right] node {$b$} ()
            (D) edge [bend right=20] node {$a$} (B)
            ;
        \end{tikzpicture}
\end{multicols}

\subsection{Návrh bezkontextové gramatiky}
Navrhněte bezkontextovou gramatiku $\G$, která generuje jazyk $L = \bc{0^i1^j ; 0 \leq i \leq j}$.
Zdůvodněte, proč gramatika $\G$ jazyk $L$ generuje.
\begin{flalign*}
    &S \rightarrow XY& \\
    &X \rightarrow 0X1 \mid \varepsilon& \\
    &Y \rightarrow Y1 \mid \varepsilon&
\end{flalign*}

Zdůvodnění:

1. Dvě možnosti: $ i = j$, a $i < j$, kde $j = i + n$, $n > 0$.
\[
    S \stackrel{S \rightarrow XY}{\Longrightarrow} XY \stackrel{X \rightarrow 0X1 (i)}{\Longrightarrow} 0^i X 1^i Y
    \Longrightarrow
\begin{cases}
    \ i < j:  & 0^i X 1^i Y \stackrel {Y \rightarrow Y1 (n)}{\Longrightarrow} 0^i X 1^i Y 1^n \stackrel{X \rightarrow
    \varepsilon}{\Longrightarrow}0^i 1^i Y 1^n \stackrel{X \rightarrow \varepsilon}{\Longrightarrow} 0^i 1^{i+n = j} \\
    \ i = j: & 0^i X 1^i Y \stackrel{Y \rightarrow \varepsilon}{\Longrightarrow} 0^i X 1^i \stackrel{X \rightarrow
    \varepsilon}{\Longrightarrow} 0^i 1^i
\end{cases}
\]

2. (fancy důkaz, doslova převzato z autorského řešení paní doc. Demlové)

Uvažujme derivaci \( S \Rightarrow^* w \). Poslední pravidlo musí být \( S \rightarrow \varepsilon \).

Provedeme indukci podle počtu kroků derivace \( n \):
\[
S \Rightarrow^n 0^i S 1^j, \quad \text{kde } i \leq j.
\]

\paragraph{Základní krok (\(n = 1\)):}
Pro \(n = 1\):
\[
S \rightarrow 0 S 1 \quad \text{nebo} \quad S \rightarrow S 1, \quad \text{a tedy } 0^i S 1^j, \; \text{kde } i \leq j.
\]

\paragraph{Indukční krok:}
Předpokládejme, že každá derivace o \(n\) krocích generuje:
\[
S \Rightarrow^n 0^i S 1^j, \quad i \leq j.
\]
Pak derivace o \(n+1\) krocích bude:
\[
S \rightarrow 0 S 1 \Rightarrow^n 0^{i+1} S 1^{j+1}, \quad \text{a tedy } i+1 \leq j+1.
\]
Nebo:
\[
S \Rightarrow^n 0^i S 1^j \Rightarrow 0^i 1^j.
\]

\paragraph{Závěr:}
Z \(S\) je možné odvodit právě slova \(0^i 1^j\), kde \(0 \leq i \leq j\), a nic jiného.