\section{Osmé cvičení}

% \subsection{Příklad}    \noindent
% K automatu $M$, který je dán následující tabulkou, zkostruujte regulární gramatiku $\mathcal{G}$, která generuje jazyk 
% $L = L(M)$. 

% $M$: $\quad$
% \begin{minipage}{0.4\textwidth}    
%     \begin{tabular}{|r c|c c|} 
%         \hline
%          & & $a$ & $b$ \\
%         \hline
%         \hline
%         $\leftrightarrow$&$ A$ & $A,C$ & $B$ \\
%         \hline
%         &$ B$ & $\emptyset$ & $B, D$ \\
%         \hline
%         $\leftarrow$ &$ C$ & $\emptyset$ & $\emptyset$ \\
%         \hline
%         $\rightarrow$&$D$ & $A$ & $C,D$ \\
%         \hline
%     \end{tabular}

%     \vspace*{5mm}
%     $\mathcal{G} = (N, \Sigma, S, P)$ \\
%     $ N = \{S, A, B, C, D\}$\\ 
%     $\Sigma = \{a, b\}$

% \end{minipage}\begin{minipage}{0.5\textwidth}
%     mám více vstupů $\rightarrow$ přidám si $S$ 

% \begin{tikzpicture}[shorten >=1pt,node distance=20mm,on grid,auto]
%         \tikzstyle{every state}=[fill={rgb:black,1;white,10}, minimum size=18pt, inner sep=1pt]
%     \node[state, initial] (S) {$S$}; 
%     \node[state] (A) [right  of=S] {$A$}; 
%     \node[state] (B) [below  of=A] {$B$}; 
%     \node[state, accepting] (C) [right  of=A] {$C$}; 
%     \node[state] (D) [below of=S] {$D$}; 
    
%     \path[->]
%     (S) edge [bend left] node {$\varepsilon$} (A)
%     (S) edge [bend right] node {$\varepsilon$} (D)
%     (A) edge [loop above] node {$a$} ()
%     (A) edge [bend left] node {$a$} (C)
%     (A) edge [bend left] node {$b$} (B)
%     (B) edge [bend left] node {$b$} (D)
%     (B) edge [loop right] node {$b$} ()
%     (D) edge [loop left] node {$b$} ()
%     (D) edge [left] node {$b$} (C)
%     (D) edge [left] node {$a$} (A)
%     ;    
% \end{tikzpicture}

% \[
%     % \hspace*{-45mm}
%     \begin{array}{l l}
%         P: & S \rightarrow A \mid D \\
%         & A \rightarrow aA \mid aC \mid bB \mid \varepsilon\\
%         & B \rightarrow bB \mid bD \\
%         & C \rightarrow \varepsilon \\
%         & D \rightarrow aA \mid bD \mid aA \\
%     \end{array}
% \]

% \end{minipage}

% \subsection{Příklad}
% \noindent
% Ke gramatice $\mathcal{G}$ typu 3 zkonstruujte konečný automat, který přijímá jazyk $L(\mathcal{G})$. Gramatika 
% $\mathcal{G} = (N, \{a,b\}, S, P)$, kde $N = \{S, A, B\}$ a pravidla jsou 
% \[
%     \begin{array}{l l}
%         P: & S \rightarrow abA \mid aB \\
%         & A \rightarrow aA \mid aaA \mid a\\
%         & B \rightarrow bB \mid b \\
%     \end{array}
% \]

% \begin{minipage}{0.5\textwidth}
    
%     \[
%         \begin{array}{l l l}
%             P: & S \rightarrow abA \mid aB & \text{překáží mi } \ii{abA}\\
%             & C \rightarrow bA & \\ 
%             & S \rightarrow aC \mid aB & \text{nové pravidlo \ii{S}} \\
%             & A \rightarrow aA \mid aaA \mid a & \text{zase nám vadí } \ii{aaA}\\
%             & D \rightarrow aA & \\
%             & E \rightarrow \varepsilon & \\
%             & A \rightarrow aA \mid aD \mid aE & \text{nové pravidlo \ii{A}} \\
%             & B \rightarrow bB \mid bE & \text{nové pravidlo \ii{B}} \\
%         \end{array}
%         \]
% \end{minipage}
% \begin{minipage}{0.3\textwidth}
% \end{minipage}
% \begin{minipage}{0.5\textwidth}
%     sestavím automat podle nových pravidel

%     \begin{tikzpicture}[shorten >=1pt,node distance=20mm,on grid,auto]
%     \tikzstyle{every state}=[fill={rgb:black,1;white,10}, minimum size=18pt, inner sep=1pt]
%     \node[state, initial] (S) {$S$}; 
%     \node[state] (B) [below  of=S] {$B$}; 
%     \node[state] (C) [right  of=S] {$C$}; 
%     \node[state] (A) [right  of=C] {$A$}; 
%     \node[state] (D) [below of=A] {$D$}; 
%     \node[state, accepting] (E) [below of=C] {$E$}; 
    
%     \path[->]
%     (S) edge [bend left] node {$a$} (C)
%     (S) edge [bend right] node {$a$} (B)
%     (C) edge [bend left] node {$b$} (A)
%     (A) edge [loop right] node {$a$} ()
%     (A) edge [bend right] node {$a$} (E)
%     (A) edge [bend left] node {$a$} (D)
%     (B) edge [loop below] node {$b$} ()
%     (B) edge [bend right] node {$b$} (E)
%     (D) edge [bend left] node {$a$} (A)
%     ;    
% \end{tikzpicture}
% \end{minipage}

% \newpage
% \subsection{Příklad}\noindent
% Je dán derivační strom v bezkontextové gramatice:


% \begin{center}
        
%     \begin{forest}
%         for tree={
%             grow=south,                 % Tree grows downward
%             edge={->},                  % Draw edges as arrows
%             align=center,               % Center the text inside nodes
%         }
%         [$S$
%             [$S$
%                 [$S$
%                     [$\varepsilon$]
%                 ]
%                 [$B$
%                     [$B$
%                         [$b$]
%                     ]
%                     [$A$
%                         [$a$]
%                     ]
%                 ]
%             ]
%             [$A$
%                 [$A$
%                     [$a$]
%                 ]
%                 [$A$
%                     [$a$]
%                 ]
%             ]
%         ]
%     \end{forest}    \end{center}


%     \begin{enumerate}[a), noitemsep]
%         %\itemsep0em 
%             \item Napište pravidla minimální CF gramatiky, ve které je to derivační strom. 
%             \item Napište levou derivaci odpovídající tomuto derivačnímu stromu.
%             \item Rozhodněte, zda je gramatika víceznačná.
%         \end{enumerate}
        
% \begin{minipage}{0.5\textwidth}
    
%     a) \[
%         \begin{array}{l l}
%             P: & S \rightarrow SA \mid SB \mid \varepsilon \\
%             & A \rightarrow AA \mid a  \\ 
%             & B \rightarrow BA \mid b  \\
%         \end{array}
%         \]
%     \end{minipage}\begin{minipage}{0.5\textwidth}
%         c) je víceznačná - už jen kvůli pravidlu $A \rightarrow AA$
%     \end{minipage}
    
%     \vspace*{2mm}
%     b) 
%     $\quad S \stackrel{S \rightarrow SA}{\Longrightarrow} SA \stackrel{S \rightarrow SB}{\Longrightarrow} SBA 
%     \stackrel{S \rightarrow \varepsilon}{\Longrightarrow} BA \stackrel{B \rightarrow BA}{\Longrightarrow} BAA 
%     \stackrel{B \rightarrow b}{\Longrightarrow} bAA \stackrel{A \rightarrow a}{\Longrightarrow} baAA 
%     \stackrel{A \rightarrow a}{\Longrightarrow} baaA\stackrel{A \rightarrow a}{\Longrightarrow} baaa$

% \subsection{Příklad} % 8.4
% Je dána bezkontextová gramatika $\mathcal{G} = (N, \Sigma, S, P)$, kde $N = \{S\}$, $\Sigma = \{+, \star, -, x, y\}$, 
% s pravidly 
% $$S \rightarrow +SS \mid \star SS \mid -SS \mid x \mid y $$ 

% \begin{itemize}[noitemsep]
%     \item Nakreslete derivační strom, který má za výsledek slovo $ w = + x \star - y x y$.  
%     \item Zkonstruujte levou derivaci slova $w$ odpovídající derivačnímu stromu z části a).
% \end{itemize}

% 1. 

% \begin{minipage}{0.5\textwidth}
    
%     \[
% \begin{tikzpicture}
%     \Tree [.$S$
%             $+$
%             [.$S$ $x$ ]
%             [.$S$
%                 $*$
%                 [.$S$
%                     $-$
%                     [.$S$ $y$ ]
%                     [.$S$ $x$ ]
%                 ]
%                 [.$S$ $y$ ]
%             ]
%           ]
% \end{tikzpicture}
% \]

% % alternative design: 
%     % \begin{center}
%     %     \begin{forest}
%     %         for tree={
%     %             grow=south,                 % Tree grows downward
%     %             edge={->},                  % Draw edges as arrows
%     %             align=center          % Center the text inside nodes
%     %         }
%     %         [$S$
%     %             [$+$]
%     %             [$S$
%     %                 [$x$]
%     %             ]
%     %             [$S$
%     %                 [$*$]
%     %                 [$S$
%     %                     [$-$]
%     %                     [$S$
%     %                         [$y$]
%     %                     ]
%     %                     [$S$
%     %                         [$x$]
%     %                     ]
%     %                 ]
%     %                 [$S$
%     %                     [$y$]
%     %                 ]
%     %             ]
%     %         ]
%     %     \end{forest}
%     %     \end{center}
% \end{minipage}
% \begin{minipage}{0.5\textwidth}
% \vspace{-30mm}
%     2.

%     \begin{align*}
%         & S \stackrel{S \rightarrow +SS}{\Longrightarrow} +SS 
%         \stackrel{S \rightarrow x}{\Longrightarrow} +xS 
%         \stackrel{S \rightarrow \star SS}{\Longrightarrow} +x\star SS \Longrightarrow\\
%         & \stackrel{S \rightarrow -SS}{\Longrightarrow} + x \star - SSS 
%         \stackrel{S \rightarrow y}{\Longrightarrow} +x\star - y SS \Longrightarrow \\
%         & \stackrel{S \rightarrow x}{\Longrightarrow} +x \star - y x S 
%         \stackrel{S \rightarrow y}{\Longrightarrow} +x \star - y x y
% \end{align*}

% \end{minipage}

% \subsection{Příklad} 
% \noindent
% Navrhněte bezkontextovou gramatiku $\mathcal{G}$, která generuje jazyk $L = \{0^ij^i2^j; i, j \geq 0\}$. Zdůvodněte, 
% proč gramatika $\mathcal{G}$ jazyk $L$ generuje. 

% \[
%     \begin{array}{l l}
%         P: & S \rightarrow XY  \\
%         & X \rightarrow 0X1 \mid \varepsilon  \\ 
%         & Y \rightarrow Y2 \mid \varepsilon \\
%     \end{array}
% \]
% \vspace*{3mm}

% 1. $L \subseteq L(\mathcal{G})$ (gramatika vygeneruje vše): \\

% $\quad S \stackrel{S \rightarrow XY}{\Longrightarrow} XY \stackrel{X \rightarrow 0X1 (i)}{\Longrightarrow} 0^iX1^iY 
% \stackrel{Y \rightarrow 2Y(j)}{\Longrightarrow} 0^iX1^iY2^j \stackrel{X \rightarrow \varepsilon}{\Longrightarrow} 
% 0^i1^iY2^j\stackrel{X \rightarrow \varepsilon}{\Longrightarrow} 0^i1^i2^j $

% \vspace*{3mm}
% 2. $L(\mathcal{G}) \subseteq L$ (gramatika nevygeneruje nic navíc): \\

% \noindent
% Uvažujme derivaci $S \implies \star w$. Pak poslední použité pravidlo musí být $X \rightarrow \varepsilon$ nebo 
% $Y \rightarrow \varepsilon$. Proto v derivaci musí být použito pravidlo $S \rightarrow XY$. Mezi tím může být použit 
% nějaký počet pravidel $X \rightarrow 0X1$ a $Y \rightarrow Y2$. Jinak pravidla být použita nemohou. Tedy drivace má tvar 
% $ S \stackrel{S \rightarrow XY}{\Longrightarrow} XY \stackrel{X \rightarrow 0X1 (i)}{\Longrightarrow} 0^iX1^iY 
% \stackrel{Y \rightarrow 2Y(j)}{\Longrightarrow} 0^iX1^iY2^j \stackrel{X \rightarrow \varepsilon}{\Longrightarrow} 
% 0^i1^iY2^j\stackrel{X \rightarrow \varepsilon}{\Longrightarrow} 0^i1^i2^j $.

% \section*{Nevypouštěcí gramatiky}

% \subsection{Příklad}
% \noindent
% Ke gramatice $\mathcal{G}$ zkostruujte nevypouštěcí gramatiku $\G_1$, pro kterou $L(\G_1) = L(\G) - \{\varepsilon\}$.

% \[
%     \begin{array}{l l}
%         P: & S \rightarrow aSbA \mid \varepsilon \\
%            & A \rightarrow aBbA \mid bCB \mid CD \\
%            & B \rightarrow bbBa \mid aS \\
%            & C \rightarrow aAaA \mid \varepsilon \\
%            & D \rightarrow SC \mid aABa \\
%     \end{array}
% \]


% obecný formální zápis u nevypouštěcích gramatik:

% $V = \{A \mid A \implies^{\star} \varepsilon \}$\\
% $V_1 = \{A \mid A \rightarrow \varepsilon \in P\}$\\
% $V_2 = V_1 \cup \{A \mid A \rightarrow \alpha \in P, \alpha \in V_1^{\star}\}$\\
% $V_{i+1} = V_i \cup \{A \mid A \rightarrow \alpha \in P, \alpha \in V_i^{\star}\}$\\

% \begin{minipage}{0.5\textwidth}
% příklad: 

% $V = \{A \mid A \implies^{\star} \varepsilon \}$\\
% $V_1 = \{A \mid A \rightarrow \varepsilon \in P\}$\\
% $V_1 = \{S, D\}$\\
% $V_2 = V_1 \cup \{A \mid A \rightarrow \alpha \in P, \alpha \in V_1^{\star}\}$\\
% $V_2 = V_1 \cup \{C\} = \{S, D, C\}$\\
% $V_3 = v_2 \cup \{A\} = \{S, D, A, C\}$        


% \end{minipage}
% \begin{minipage}{0.5\textwidth}
% $\G_1$:
%     \[
%     \begin{array}{l l}
%         P: & S \rightarrow aSbA \mid abA \mid aSb \mid ab  \\
%            & A \rightarrow aBbA \mid aBb \mid bCB \mid bB \mid CD \mid C \mid D \\
%            & B \rightarrow bbBa \mid aS \mid a\\
%            & C \rightarrow aAaA \mid aAa \mid aaA \mid aa \\
%            & D \rightarrow SC \mid S \mid C \mid aABa \mid aBa \\
%     \end{array}
% \]

% \end{minipage}


% \subsection{Příklad}
% \noindent
% K automatu $M$ zkonstruujte gramatiku typu 3 která generuje jazyk $L(M)$, kde $M$
% je dán tabulkou

% \vspace*{3mm}
% \begin{minipage}{0.5\textwidth}    
%     $M$: $\quad$ 
%     \begin{tabular}{|c c||c| c|} 
%         \hline
%         & & $a$ & $b$ \\
%         \hline
%         $\rightarrow$&$ A $& $\{A,B\}$ & $\{C\}$ \\
%         &$ B$ & $\{B\}$ & $\{C\}$ \\
%         $\leftrightarrow$ &$ C$ & $\emptyset$ & $\{D\}$ \\
%         $\leftarrow$&$ D$ & $\{B\}$ & $\{D\}$ \\
%         \hline
%     \end{tabular}
%     \vspace*{10mm}
    
%     $\mathcal{G} = (N, \Sigma, S, P)$ \vspace*{2mm}
    
%     $N = \{S, A, B, C, D\}$ \vspace*{2mm}
    
%     $\Sigma = \{a, b\}$ 
    
%     \[
%         % \hspace*{-45mm}
%         \begin{array}{l l}
%             P: & S \rightarrow A \mid C \\
%             & A \rightarrow aA \mid aB \mid bC\\
%             & B \rightarrow aB \mid bC \\
%             & C \rightarrow bD \mid \varepsilon \\
%             & D \rightarrow aB \mid bD \mid \varepsilon
%         \end{array}
%         \]
        
%     \end{minipage}  
%     \begin{minipage}{0.5\textwidth}    
        
%         \begin{tikzpicture}[shorten >=1pt,node distance=35mm,on grid,auto]
%             \tikzstyle{every state}=[fill={rgb:black,1;white,10}, inner sep=1pt]
            
%             \def\radius{22mm}
            
%             \node[state] (A) at (90:\radius) {$A$}; % top center
%             \node[state, initial] (S) at (162:\radius) {$S$}; % top-left
%             \node[state] (B) at (18:\radius) {$B$}; % bottom-left
%             \node[state, accepting] (D) at (306:\radius) {$D$}; % bottom-right
%             \node[state, accepting] (C) at (234:\radius) {$C$}; % top-right
%             \path[->]
            
%             (S) edge [bend left=20] node {$\varepsilon$} (A)
%             (S) edge [bend right=20] node {$\varepsilon$} (C)
            
%             (A) edge [loop above] node {$a$} ()
%             (A) edge [bend left=20] node {$a$} (B)
%             (A) edge [left] node {$b$} (C)
            
%             (B) edge [loop above] node {$a$} ()
%             (B) edge [above] node {$b$} (C)
            
%             (C) edge [bend right=20] node {$b$} (D)
            
%             (D) edge [loop right] node {$b$} ()
%             (D) edge [bend right=20] node {$a$} (B)
%             ;    
%         \end{tikzpicture}
        
        
%     \end{minipage}  

% \section*{Navrhněte gramatiku}
% \subsection{Příklad}
%     \noindent
%     Navrhněte bezkontextovou gramatiku $\mathcal{G}$, která generuje jazyk $L = \{0^i1^j ; 0 \leq i \leq j\}$.
%     Zdůvodněte, proč gramatika $\mathcal{G}$ jazyk $L$ generuje.
    
%     \begin{quote}
%     $S \rightarrow XY$\\
%     $X \rightarrow 0X1 \mid \varepsilon$\\
%     $Y \rightarrow Y1 \mid \varepsilon$\\
% \end{quote}

% Zdůvodnění: 

% 1. 

% Dvě možnosti: $ i = j$, a $i < j$, kde $j = i + n$, $n > 0$. 
% \[
%     S \stackrel{S \rightarrow XY}{\Longrightarrow} XY \stackrel{X \rightarrow 0X1 (i)}{\Longrightarrow} 0^i X 1^i Y  
%     \Longrightarrow
% \begin{cases}
%     \ i < j:  & 0^i X 1^i Y \stackrel {Y \rightarrow Y1 (n)}{\Longrightarrow} 0^i X 1^i Y 1^n \stackrel{X \rightarrow 
%     \varepsilon}{\Longrightarrow}0^i 1^i Y 1^n \stackrel{X \rightarrow \varepsilon}{\Longrightarrow} 0^i 1^{i+n = j} \\
%     \ i = j: & 0^i X 1^i Y \stackrel{Y \rightarrow \varepsilon}{\Longrightarrow} 0^i X 1^i \stackrel{X \rightarrow 
%     \varepsilon}{\Longrightarrow} 0^i 1^i
% \end{cases}
% \]

% 2. (fancy důkaz, doslova opsáno z autorského řešení pí. Demlové)

% Uvažujme derivaci \( S \Rightarrow^* w \). Poslední pravidlo musí být \( S \rightarrow \varepsilon \).

% Provedeme indukci podle počtu kroků derivace \( n \):
% \[
% S \Rightarrow^n 0^i S 1^j, \quad \text{kde } i \leq j.
% \]

% \paragraph{Základní krok (\(n = 1\)):}
% Pro \(n = 1\):
% \[
% S \rightarrow 0 S 1 \quad \text{nebo} \quad S \rightarrow S 1, \quad \text{a tedy } 0^i S 1^j, \; \text{kde } i \leq j.
% \]

% \paragraph{Indukční krok:}
% Předpokládejme, že každá derivace o \(n\) krocích generuje:
% \[
% S \Rightarrow^n 0^i S 1^j, \quad i \leq j.
% \]
% Pak derivace o \(n+1\) krocích bude:
% \[
% S \rightarrow 0 S 1 \Rightarrow^n 0^{i+1} S 1^{j+1}, \quad \text{a tedy } i+1 \leq j+1.
% \]
% Nebo:
% \[
% S \Rightarrow^n 0^i S 1^j \Rightarrow 0^i 1^j.
% \]

% \paragraph{Závěr:}
% Z \(S\) je možné odvodit právě slova \(0^i 1^j\), kde \(0 \leq i \leq j\), a nic jiného.


