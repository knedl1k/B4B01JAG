\section{Čtrnácté cvičení}

\subsection{Převod gramatiky do Greibachové normální formy}
Do Greibachové normální formy převeďte gramatiku $\G$, kde $\G = (N, \Sigma, S, P)$, kde $N = \bc{S, A}$, 
\\$\Sigma = \bc{0, 1}$ a $P$ je dáno
\begin{align*}
    S &\rightarrow SA \mid 0 \\
    A &\rightarrow AS \mid 1
\end{align*} 

\textbf{1. krok} oindexování neterminálů.

$X_1 = S \\
X_2 = A \\
X_3 = S^{'} \\
X_4 = A^{'}
$

\textbf{2. krok} odstranění levých rekurzí.
\begin{flalign*}
    S^{\phantom{'}} &\rightarrow 0 \mid 0S^{'} & \\
    S^{'} &\rightarrow A \mid AS^{'} & \\
    A^{\phantom{'}} &\rightarrow 1 \mid 1A^{'} & \\
    A^{'} &\rightarrow S \mid SA^{'} &
\end{flalign*}

\textbf{3. krok} Kontrola správného pořadí indexů (na pravé straně vždy neterminál s větším 
indexem), jinak sloučit pravidla. (postup shora dolů)
\begin{flalign*}
    S^{\phantom{'}} &\rightarrow 0 \mid 0S^{'} & \\
    A^{\phantom{'}} &\rightarrow 1 \mid 1A^{'} & \\
    S^{'} &\rightarrow 1 \mid 1A^{'} \mid 1S^{'} \mid 1A^{'}S^{'}  & \\
    A^{'} &\rightarrow 0 \mid 0S^{'} \mid 0A^{'} \mid 0S^{'}A^{'} &
\end{flalign*}

\textbf{4. krok} za prvním terminálem pravé strany vždy následují pouze neterminály. $\checkmark$

\subsection{Zkoušková ukázka práce na bezkontextové gramatice}
Je dána bezkontextová gramatika $\G = (N, \Sigma, S, P)$, kde $N = \bc{S,A,B,C}$, $\Sigma = \bc{a,b}$ a $P$ je dáno

\begin{align*}
    S &\rightarrow Sa \mid Sb \mid bC \\
    A &\rightarrow CBA \mid BC \mid b \\
    B &\rightarrow aB \mid \varepsilon \\
    C &\rightarrow AA \mid bBb \mid \varepsilon
\end{align*}

\begin{enumerate}[(a),noitemsep]
    \item Ke gramatice $\G$ najděte nevypouštěcí gramatiku $\G_1$. Kroky převodu popište.
    \item Ke gramatice $\G_1$ najděte gramatiku $\G_2$ v Chomského normálním tvaru, která generuje stejný jazyk jako 
    gramatika $\G_1$. Jednotlivé kroky popište, gramatiku v Chomského normálním tvaru definujte.
    \item Pomocí matematické indukce dokažte, že platí $A \Rightarrow_\G^\star A^i C (BA)^{i+1}$ pro každé $i \geq 0$.
    Toho využijte k důkazu, že $b^{i+2}(ab)^{i+1}$ je generováno gramatikou $\G$ pro každé $i \geq 0$.
    \item Je gramatika $\G$ víceznačná? Víceznačnou gramatiku definujte.
    \item V gramatice $\G_1$ odstraňte levou rekurzi u symbolu $S$. Postup popište.
\end{enumerate}

\textbf{(a)} Nevypouštěcí gramatika $\G_1$. 

$V = \bc{A \mid A \Rightarrow^\star \varepsilon }$\\
$V_1 = \bc{A \mid A \rightarrow \varepsilon \in P} = \bc{B, C}$\\
$V_2 = V_1 \cup \bc{A \mid A \rightarrow \alpha \in P, \alpha \in V_1^{\star}} = V_1 \cup \bc{A} = \bc{A, B, C}$\\
$V_3 = V_2 \cup \bc{A \mid A \rightarrow \alpha \in P, \alpha \in V_2^{\star}} = V_2 \cup \varepsilon = V_2 = V$

\begin{flalign*}
    \G_1 \text{: } S &\rightarrow Sa \mid Sb \mid bC \mid b & \\
    A &\rightarrow CBA \mid CA \mid CB \mid BA \mid BC \mid B \mid C \mid b & \\
    B &\rightarrow aB \mid a & \\
    C &\rightarrow AA \mid A \mid bBb \mid bb &
\end{flalign*}

\textbf{(b)} Chomského normální tvar gramatiky $\G_1$.

\textbf{I. krok} nahrazení samostatných terminálů pravidly. (pokud nastane např. $A \rightarrow A$, tak vynechat.)
\begin{flalign*}
    S &\rightarrow Sa \mid Sb \mid bC \mid b & \\
    A &\rightarrow CBA \mid CA \mid CB \mid BA \mid BC \mid \underbrace{aB \mid a}_B \mid \underbrace{AA \mid bBb \mid bb}_C \mid b & \\
    B &\rightarrow aB \mid a & \\
    C &\rightarrow AA \mid \underbrace{CBA \mid CA \mid CB \mid BA \mid BC \mid B \mid C \mid b}_A \mid bBb \mid bb &
\end{flalign*}

\textbf{II. krok} nahrazení terminálů neterminály pokud nejsou samotné.
\begin{flalign*}
    S &\rightarrow SX_1 \mid SX_2 \mid X_2C \mid b & \\
    A &\rightarrow CBA \mid CA \mid CB \mid BA \mid BC \mid X_1B \mid X_1 \mid AA \mid X_2BX_2 \mid X_2X_2 \mid X_2 & \\
    B &\rightarrow X_1B \mid X_1 & \\
    C &\rightarrow AA \mid CBA \mid CA \mid CB \mid BA \mid BC \mid B \mid C \mid X_2 \mid X_2BX_2 \mid X_2X_2 & \\
    X_1 &\rightarrow a & \\
    X_2 &\rightarrow b & \\
\end{flalign*}

\textbf{III. krok} nahrazení pravých stran, která mají délku $\geq 3$.
\begin{flalign*}
    S &\rightarrow SX_1 \mid SX_2 \mid X_2C \mid b & \\
    A &\rightarrow Y \mid CA \mid CB \mid BA \mid BC \mid X_1B \mid X_1 \mid AA \mid Z \mid X_2X_2 \mid X_2 & \\
    Y &\rightarrow CBA & \\
    Z &\rightarrow X_2BX_2 & \\
    B &\rightarrow X_1B \mid X_1 & \\
    C &\rightarrow AA \mid Y \mid CA \mid CB \mid BA \mid BC \mid B \mid C \mid X_2 \mid Z \mid X_2X_2 & \\
    X_1 &\rightarrow a & \\
    X_2 &\rightarrow b & \\
\end{flalign*}

\textbf{(c)} Důkaz.

$A \Rightarrow_\G^\star A^i C (BA)^{i+1}$

Základní krok: $i=0$: $A^0 C (BA)^1 = CBA$ $\checkmark$ $A \Rightarrow_\G^\star CBA$.

Indukční krok: $i \geq 0$: indukční předpoklad: $A \Rightarrow^\star A^i C (BA)^{i+1}$.

$A \xRightarrow{A \rightarrow CBA} CBA \xRightarrow{C \rightarrow AA} A\underline{A}_{IP} (BA) \xRightarrow{IP}^\star 
AA^i C (BA)^{i+1} (BA) = A^{i+1} C (BA)^{i+2}$. $\checkmark$

A tedy, $b^{i+2} (ab)^{i+1} \in L(\G)$?

$S \xRightarrow{S \rightarrow bC} bC \xRightarrow{C \rightarrow AA} bAA \xRightarrow{A \rightarrow b} b^2 A \Rightarrow
|\text{dle důkazu výše}| \xRightarrow{}^\star b^2 A^i C (BA)^{i+1} \xRightarrow{A \rightarrow b} b^{i+2} C (BA)^{i+1} 
\xRightarrow{C \rightarrow \varepsilon} b^{i+2} (BA)^{i+1} \xRightarrow{B \rightarrow aB} b^{i+2} (aBA)^{i+1} 
\xRightarrow{B \rightarrow \varepsilon} b^{i+2} (aA)^{i+1} \xRightarrow{A \rightarrow b} b^{i+2} (ab)^{i+1}$. $\checkmark$

\textbf{(d)} Je gramatika $\G$ víceznačná?

Víceznačnost = existují alespoň 2 derivační stromy / 2 levé derivace pro jedno libovolné slovo z $\G$.

Například mějme slovo $w=bbb$.

První způsob vygenerování slova $w$: $S \xRightarrow{S \rightarrow bC} bC \xRightarrow{C \rightarrow AA} bAA
\xRightarrow{A \rightarrow b}^2 bbb.$

Druhý způsob vygenerování slova $w$: $S \xRightarrow{S \rightarrow bC} bC \xRightarrow{C \rightarrow bBb} bbBb
\xRightarrow{B \rightarrow \varepsilon} bbb.$

A tedy gramatika $\G$ je víceznačná.

\textbf{(e)} Odstranění levých rekurzí.

Levá rekurze se vyskytuje pouze v pravidlu $S \rightarrow Sa \mid Sb \mid bC \mid b$.

Je potřeba přidat pouze jeden neterminál. Pokud by se jich přidalo více, nová gramatika by generovala méně slov, než původní.
\begin{flalign*}
    S^{\phantom{'}} &\rightarrow bC \mid bCS^{'} \mid b \mid S^{'} & \\
    S^{'} &\rightarrow a \mid aS^{'} \mid b \mid bS^{'} &
\end{flalign*}

\subsection{Zkoušková ukázka práce na bezkontextové gramatice}
Je dána bezkontextová gramatika $\G = (N, \Sigma, S, P)$, kde $N = \bc{S,A,B,C,D}$, $\Sigma = \bc{a,b,c}$ a $P$ je dáno

\begin{align*}
    S &\rightarrow BSA \mid b \\
    A &\rightarrow CB \mid Aa \mid c \\
    B &\rightarrow bA \mid bB \mid \varepsilon \\
    C &\rightarrow cA \mid \varepsilon \\
    D &\rightarrow SA \mid a
\end{align*}

\begin{enumerate}[(a), noitemsep]
    \item Ke gramatice $\G$ najděte nevypouštěcí gramatiku $\G_1$. Kroky převodu popište.
    \item Ke gramatice $\G_1$ najděte redukovanou gramatiku $\G_2$ . Kroky převodu popište.
    \item Ke gramatice $\G_2$ najděte gramatiku $\G_3$ v Chomského normálním tvaru, která generuje stejný jazyk jako 
    gramatika $\G_2$. Jednotlivé kroky popište, gramatiku v Chomského normálním tvaru definujte.
    \item Pomocí matematické indukce dokažte, že platí $A \Rightarrow_\G^\star (cb)^i A$ pro každé $i \geq 0$.
    Toho využijte k důkazu, že $b(cb)^i a$ je generováno gramatikou $\G$ pro každé $i \geq 0$.
    \item Je gramatika $\G$ víceznačná? Víceznačnou gramatiku definujte.
    \item V gramatice $\G_2$ odstraňte (přímou) levou rekurzi. Postup popište.
\end{enumerate}

\textbf{(a)} Nevypouštěcí gramatika $\G_1$.

$V = \bc{A \mid A \Rightarrow^\star \varepsilon }$\\
$V_1 = \bc{A \mid A \rightarrow \varepsilon \in P} = \bc{B, C}$\\
$V_2 = V_1 \cup \bc{A \mid A \rightarrow \alpha \in P, \alpha \in V_1^{\star}} = V_1 \cup \bc{A} = \bc{A, B, C}$\\
$V_3 = V_2 \cup \bc{A \mid A \rightarrow \alpha \in P, \alpha \in V_2^{\star}} = V_2 \cup \varepsilon = V_2 = V$

\begin{flalign*}
    \G_1 \text{: } S &\rightarrow BSA \mid BS \mid SA \mid b & \\
    A &\rightarrow CB \mid C \mid B \mid Aa \mid a \mid c & \\
    B &\rightarrow bA \mid b \mid bB & \\
    C &\rightarrow cA \mid c & \\
    D &\rightarrow SA \mid S \mid a &
\end{flalign*}

\textbf{(b)} Redukce gramatiky $\G_1$.

\textbf{I. krok}\\
$V = \bc{A \mid A \Rightarrow^\star \alpha, \alpha^\star} = \bc{S,A,B,C,D} = N$\\
$V_{i+1} = V_i \cup \bc{A \mid A \Rightarrow w, w \in (V_i \cup \Sigma)^\star}$

\textbf{II. krok}\\
$U = \bc{A \mid A \Rightarrow^\star \alpha A \beta}$\\
$U_0 = \bc{S}$\\
$U_{i+1} = U_i \cup \bc{A \mid B \in U_i, \alpha, \beta \in (U_i \cup \Sigma)^\star : B \rightarrow \alpha A \beta}$\\
$U_1 = \bc{S,A,B}$\\
$U_2 = \bc{S,A,B,C}$

\begin{flalign*}
    \G_2 \text{: } S &\rightarrow BSA \mid BS \mid SA \mid b & \\
    A &\rightarrow CB \mid C \mid B \mid Aa \mid a \mid c & \\
    B &\rightarrow bA \mid b \mid bB & \\
    C &\rightarrow cA \mid c & \\
\end{flalign*}

\textbf{(c)} Chomského normální tvar k $\G_2$.

\textbf{1. krok} nevypouštěcí $\checkmark$\\
\textbf{2. krok} nahradit samotné neterminály jejich pravidly
\begin{flalign*}
    S &\rightarrow BSA \mid BS \mid SA \mid b & \\
    A &\rightarrow CB \mid CA \mid c \mid bA \mid b \mid bB \mid Aa \mid a & \\
    B &\rightarrow bA \mid b \mid bB & \\
    C &\rightarrow cA \mid c &
\end{flalign*}
\textbf{3. krok} nahradit nesamotné terminály neterminály
\begin{flalign*}
    S &\rightarrow BSA \mid BS \mid SA \mid b & \\
    A &\rightarrow CB \mid X_cA \mid c \mid X_bA \mid b \mid X_bB \mid AX_a \mid a & \\
    B &\rightarrow X_bA \mid b \mid X_bB & \\
    C &\rightarrow X_cA \mid c & \\
    X_a &\rightarrow a & \\
    X_b &\rightarrow b & \\
    X_c &\rightarrow c &
\end{flalign*}
\textbf{4. krok} rozbít dlouhá ($\geq 3$) slova
\begin{flalign*}
    S_{\phantom{a}} &\rightarrow BK \mid BS \mid SA \mid b & \\
    K_{\phantom{a}} &\rightarrow SA & \\
    A_{\phantom{a}} &\rightarrow CB \mid X_cA \mid c \mid X_bA \mid b \mid X_bB \mid AX_a \mid a & \\
    B_{\phantom{a}} &\rightarrow X_bA \mid b \mid X_bB & \\
    C_{\phantom{a}} &\rightarrow X_cA \mid c & \\
    X_a &\rightarrow a & \\
    X_b &\rightarrow b & \\
    X_c &\rightarrow c &
\end{flalign*}

\textbf{(d)} indukce

1) základní krok: $i=0$: $A \Rightarrow^\star A$, což odpovídá $(cb)^0 A = A \checkmark$.

2) indukční krok: $i \geq 0$: Indukční předpoklad $A \Rightarrow^\star (cb)^n A$.

Pak $A \xRightarrow{IP} (cb)^n A \xRightarrow{A \rightarrow CB} (cb)^n CB \xRightarrow{C \rightarrow c} (cb)^n c B
\xRightarrow{B \rightarrow bA} (cb)^n cb A \equiv (cb)^{n+1} A \checkmark$.

A tedy $b(cb)^n a \in L$?

$S \xRightarrow{S \rightarrow SA} SA \xRightarrow{S \rightarrow b} bA \Rightarrow^\star |\text{dle důkazu výše}| 
\Rightarrow b (cb)^n A \xRightarrow{A \rightarrow a} b (cb)^n a \checkmark$. Slovo je generováno pravidly gramatiky, je
tedy v jazyce L. 

\textbf{(e)} Gramatika je víceznačná pokud umožňuje vygenerovat alespoň jedno slovo alespoň dvěma různými způsoby.

Podle oka asi ne, je potřeba formálně ověřit % TODO: dodělat

\textbf{(f)} odstranění přímé levé rekuze z $\G_2$
\begin{flalign*}
    \G_2 \text{: } S &\rightarrow BSA \mid BS \mid SA \mid b & \\
    A &\rightarrow CB \mid C \mid B \mid Aa \mid a \mid c & \\
    B &\rightarrow bA \mid b \mid bB & \\
    C &\rightarrow cA \mid c &
\end{flalign*}

\begin{flalign*}
    S^{\phantom{'}} &\rightarrow BSA \mid BS \mid b \mid BSAS^{'} \mid BSS^{'} \mid bS^{'} & \\
    S^{'} &\rightarrow A \mid AS^{'}  & \\
    A^{\phantom{'}} &\rightarrow CB \mid C \mid B \mid a \mid c \mid CBA^{'} \mid CA^{'} \mid BA^{'} \mid aA^{'} \mid cA^{'} & \\
    A^{'} &\rightarrow a \mid aA^{'}  & \\
    B^{\phantom{'}} &\rightarrow bA \mid b \mid bB & \\
    C^{\phantom{'}} &\rightarrow cA \mid c & 
\end{flalign*}
