%%% Ugly hack for no spacing around alignments, just a local thing
\setlength{\abovedisplayskip}{0pt}
\setlength{\belowdisplayskip}{0pt}
\setlength{\abovedisplayshortskip}{0pt}
\setlength{\belowdisplayshortskip}{0pt}
%%% 

\section{Třinácté cvičení}

\subsection{Tvorba a práce zásobníkového automatu podle jazyka}
Je dán jazyk $L$ nad abecedou $\Sigma = \{0,1\}$. Sestrojte zásobníkové automaty $A,B$ tak, že $L = N(A)$ a $L = L(B)$
(tj. $A$ přijímá $L$ prázdným zásobníkem, $B$ přijímá $L$ koncovým stavem), kde
\[L = \{0^i 1^j 0^k \mid 0 \leq i < k, j > 0\}\text{.}\]
Ukažte práci jednoho ze zásobníkových automatů nad slovem $011000$ a nad slovem $001110$.

Přímou metodou:

\begin{multicols}{2}
    \begin{tikzpicture}
        \node[state, initial] (0) {$q_0$};
        \node[state, right of=0] (1) {$q_1$};
        \node[state, right of=1] (2) {$q_2$};
        \node[state, accepting, below of=1] (f) {$q_f$};
    
        \draw
            (0) edge[loop above] node[align=center]{$0, A / AA$ \\ $0, Z_0 / A Z_0$} (0)
            (0) edge[bend left, above] node{$1, Z_0 / Z_0$} (1)
            (0) edge[bend left, below] node{$1, A / A$} (1)
            (1) edge[loop above] node[align=center]{$1, A / A$ \\ $1, Z_0 / Z_0$} (1)
            (1) edge[bend left, above] node{$0, Z_0 / Z_0$} (2)
            (1) edge[bend left, below] node{$0, A / A$} (2)
            (2) edge[loop above] node[align=center]{$0, A / \varepsilon$ \\ $0, Z_0 / Z_0$} (2)
            (2) edge[bend left, left] node{$\varepsilon, Z_0 / \varepsilon$} (f) 
            ;
  \end{tikzpicture}

    \columnbreak\

    Práce nad slovem $w_1 = 011000$.

    ${(q_0, 011000, Z_0) \vdash (q_0, 11000, A Z_0) \vdash (q_1, 1000, A Z_0)}$
    \begin{flalign*}
        \vdash (q_1, 000, A Z_0) \vdash (q_2, 00, A Z_0) &\vdash (q_f, 0, A Z_0) \xmark & \\
        &\vdash (q_2, 0, Z_0)
    \end{flalign*}
    \begin{flalign*}
        (q_2, 0, Z_0) &\vdash (q_f, 0, Z_0) \xmark & \\
        &\vdash (q_2, \varepsilon, Z_0) \vdash (q_f, \varepsilon, \varepsilon) \checkmark\\
    \end{flalign*}
    
    Práce nad slovem $w_2 = 001110$.

    ${(q_0, 001110, Z_0) \vdash^{(2)} (q_0, 1110, AA Z_0) \vdash} \\ 
    (q_1, 110, AA Z_0) \vdash^{(2)} (q_1, 0, AA Z_0) \vdash (q_2, \varepsilon, AA Z_0) \xmark $

\end{multicols}

Přes gramatiku:
\begin{flalign*}
    \G : S &\rightarrow S0 \mid 0S0 \mid A0 & \\
    A &\rightarrow 1A \mid 1 \\
\end{flalign*}

Důkaz.

1) $L \subseteq L(\G)$\\
$S \Rightarrow^\star 0^i S 0^k \xRightarrow{S \rightarrow A0} 0^i A 0^{k+1} \xRightarrow{A \rightarrow 1A}^{(j)} 
0^i 1^j A 0^{k+1} \xRightarrow{A \rightarrow 1} 0^i 1^{j+1} 0^{k+1}$, $i \leq k, j > 0$. %? Is this in the right spot?

2) $L(\G) \subseteq L$\\
% TODO: complete proof 2.


\subsection{Tvorba nevypouštěcí gramatiky}
Je dána bezkontextová gramatika $\G = (N, \Sigma, S, P)$, kde $N = \{S,A,B,C\}, \Sigma = \{0,1\}$ a $P$ je dáno
\begin{align*}
    S &\rightarrow SA \mid 0 \\
    A &\rightarrow BAB \mid 1 \\
    B &\rightarrow CB \mid \varepsilon \\
    C &\rightarrow AS \mid 0 \mid \varepsilon \\
\end{align*}
Ke gramatice $\G$ vytvořte nevypouštěcí gramatiku $\G_1$. V gramatice $\G_1$ odstraňte levou rekurzi.

1. krok Vytvoření nevypouštěcí gramatiky $\G_1$.
\begin{flalign*}
    V &= \{x \mid x \Rightarrow^\star \varepsilon\} & \\
    V_1 &= \{x \mid x \rightarrow \varepsilon \in P\} = \{B,C\} & \\
    V_2 &= V_1 \cup \{x \mid x \rightarrow \alpha \in P, \alpha \in V_1^+\} = V_1 \cup \emptyset = V_1 = V.
\end{flalign*}

% TODO: dopsat sem nevypousteci gramatiku 

2. krok odstranění levé rekurze. (postup shora dolů)
\begin{flalign*}
    S    &\rightarrow 0 \mid 0S^{'} & \\
    S^{'}&\rightarrow A \mid AS^{'} & \\
    A    &\rightarrow BAB \mid BA \mid 1 \mid BABA^{'} \mid BAA^{'} \mid 1A^{'} & \\
    A^{'}&\rightarrow B \mid BA^{'} & \\
    B    &\rightarrow CB \mid C & \\
    C    &\rightarrow AS \mid 0 &
\end{flalign*}


\subsection{Převod gramatiky do Greibachové normální formy}
Do Greibachové normální formy převěďte gramatiku $\G$, kde $\G = (N, \Sigma, S, P)$, kde $N = \{S, E, F\},\\ 
\Sigma = \{a, *, +, ), (\}$ a $P$ je dáno 
\begin{align*}
    S &\rightarrow (E)\\
    E &\rightarrow F * F \mid F + F\\
    F &\rightarrow a \mid S
\end{align*}
\textbf{1. krok} oindexování neterminálů.

$A_1 = S \\
A_2 = E \\
A_3 = F
$

\textbf{2. krok} odstranění levých rekurzí. Kontrola správného pořadí indexů (na pravé straně vždy neterminál s větším 
indexem), jinak sloučit pravidla. (postup shora dolů)
\begin{flalign*}
    S &\rightarrow (E) & \\
    E &\rightarrow F * F \mid F + F& \\
    F &\rightarrow a \mid (E)
\end{flalign*}

\textbf{3. krok} nahrazení prvních neterminálů pravých stran, které neterminálem začínají, pravidly. (postup zespoda 
nahoru)
\begin{flalign*}
    S &\rightarrow (E) & \\
    E &\rightarrow a * F \mid a + F \mid (E) * F \mid (E) + F& \\
    F &\rightarrow a \mid (E)
\end{flalign*}

\textbf{4. krok} za prvním terminálem pravé strany vždy následují pouze neterminály.
\begin{flalign*}
    S &\rightarrow (EX & \\
    E &\rightarrow a Y F \mid a Z F \mid (EX Y F \mid (EX Z F & \\
    F &\rightarrow a \mid (EX & \\
    X &\rightarrow ) & \\
    Y &\rightarrow * & \\
    Z &\rightarrow +
\end{flalign*}

\subsection{Převod gramatiky do Greibachové normální formy}
Do Greibachové normální formy převěďte gramatiku $\G$, kde $\G = (N, \Sigma, S, P)$, kde $N = \{S, A, B\},\\
\Sigma = \{a, b, c\}$ a $P$ je dáno 
\begin{align*}
    S &\rightarrow Ab \mid B\\
    A &\rightarrow Aba \mid Bcc\\
    B &\rightarrow Sa \mid b
\end{align*}
\textbf{1. krok} oindexování neterminálů.

$A_1 = S \\
A_2 = A \\
A_3 = B$

% TODO: prepsat gramatiky z flalign do table \raggedleft, at to je zarovnany 

\textbf{2. krok} odstranění levých rekurzí. Kontrola správného pořadí indexů (na pravé straně vždy neterminál s větším 
indexem), jinak sloučit pravidla. (postup shora dolů)
\begin{flalign*}
    S &\rightarrow Ab \mid B & \\
    A &\rightarrow Ba \mid BaA^{'} & \\
    A^{'} &\rightarrow ba \mid baA^{'} & \\
    B &\rightarrow Aba \mid Ba \mid b & \\
    B &\rightarrow Baba \mid BaA^{'}ba \mid Ba \mid b & \\
    B &\rightarrow b \mid bB^{'} & \\
    B^{'} &\rightarrow aba \mid aA^{'}ba \mid a \mid abaB^{'} \mid aA^{'}baB^{'} \mid aB^{'}& \\
\end{flalign*}

% \begin{table}[h]
%     \raggedright % Aligns the table to the left of the page
%     \begin{tabular}{l l l} % 'l' ensures left alignment within the columns
%     $S$ &\rightarrow Ab \mid B & \\
%     $A $&\rightarrow Ba \mid BaA^{'} & \\
%     $A^{'}$ &\rightarrow ba \mid baA^{'} & \\
%     $B $&\rightarrow Aba \mid Ba \mid b & \\
%     $B $&\rightarrow Baba \mid BaA^{'}ba \mid Ba \mid b & \\
%     $B $&\rightarrow b \mid bB^{'} & \\
%     $B^{'}$ &$\rightarrow aba \mid aA^{'}ba \mid a \mid abaB^{'} \mid aA^{'}baB^{'} \mid aB^{'}$\\
%     \end{tabular}
%     \caption{Example Table Aligned to the Left}
% \end{table}

\textbf{3. krok} nahrazení prvních neterminálů pravých stran, které neterminálem začínají, pravidly. (postup zespoda 
nahoru)
\begin{flalign*}
    S &\rightarrow Bab \mid BaA^{'} \mid b \mid bB^{'} & \\
    A &\rightarrow ba \mid bB^{'}a \mid baA^{'} \mid bB^{'}aA^{'} & \\
    A^{'} &\rightarrow ba \mid baA^{'} & \\
    B &\rightarrow b \mid bB^{'} & \\
    B^{'} &\rightarrow aba \mid aA^{'}ba \mid a \mid abaB^{'} \mid aA^{'}baB^{'} \mid aB^{'}
\end{flalign*}

\textbf{4. krok} za prvním terminálem pravé strany vždy následují pouze neterminály.
\begin{flalign*}
    S &\rightarrow BaY \mid BaA^{'} \mid b \mid bB^{'} & \\
    A &\rightarrow bX \mid bB^{'}X \mid bXA^{'} \mid bB^{'}aA^{'} & \\
    A^{'} &\rightarrow bX \mid bXA^{'} & \\
    B &\rightarrow b \mid bB^{'} & \\
    B^{'} &\rightarrow aXY \mid aA^{'}YX \mid a \mid aYXB^{'} \mid aA^{'}YXB^{'} \mid aB^{'} & \\
    X &\rightarrow a & \\
    Y &\rightarrow b
\end{flalign*}