\section{Třinácté cvičení}

\subsection{Příklad}
Je dán jazyk $L$ nad abecedou $\Sigma = \{a,b\}$. Sestrojte zásobníkové automaty $A,B$ tak, že $L = N(A)$ a $L = L(B)$
(tj. $A$ přijímá $L$ prázdným zásobníkem, $B$ přijímá $L$ koncovým stavem), kde
\[L = \{0^i 1^j 0^k \mid 0 \leq i < k, j > 0\}\text{.}\]

\subsection{Příklad}
Je dána bezkontextová gramatika $\G = (N, \Sigma, S, P)$, kde $N = \{S,A,B,C\}, \Sigma = \{0,1\}$ a $P$ je dáno
\begin{align*}
    S &\rightarrow SA \mid 0 \\
    A &\rightarrow BAB \mid 1 \\
    B &\rightarrow CB \mid \varepsilon \\
    C &\rightarrow AS \mid 0 \mid \varepsilon \\
\end{align*}
Ke gramatice $\G$ vytvořte nevypouštěcí gramatiku $\G_1$. V gramatice $\G_1$ odstraňte levou rekurzi.

\subsection{Příklad}
Do Greibachové normální formy převěďte gramatiku $\G$, kde $\G = (N, \Sigma, S, P)$, kde $N = \{S, E, F\},\\ 
\Sigma = \{a, *, +, ), (\}$ a $P$ je dáno 
\begin{align*}
    S &\rightarrow (E)\\
    E &\rightarrow F * F \mid F + F\\
    F &\rightarrow a \mid S
\end{align*}

\subsection{Příklad}
Do Greibachové normální formy převěďte gramatiku $\G$, kde $\G = (N, \Sigma, S, P)$, kde $N = \{S, A, B\},\\
\Sigma = \{a, b, c\}$ a $P$ je dáno 
\begin{align*}
    S &\rightarrow Ab \mid B\\
    A &\rightarrow Aba \mid Bcc\\
    B &\rightarrow Sa
\end{align*}
