\section{První cvičení}

\subsection{Charakterizace jazyka}
Jazyk $L$ nad abecedou $\Sigma = \{a,b\}$ je dán induktivně
\begin{gather*}
    \varepsilon \in L \\
    u \in L \implies aub \in L\\
    u \in L \implies bua \in L\\
    u, v \in L \implies uv \in L\\
\end{gather*}
Charakterizujte slova jazyka $L$, tj. najděte vlastnost $\mathcal{V}$ takovou, že $L = \{u \mid \text{slovo } u
\text{ má vlastnost } \mathcal{V}\}$. Své tvrzení dokažte.

$L_1 = \{w \mid w \in \{a,b\}^\star, |w|_a=|w|_b\}$.

Důkaz:

a) $L \subseteq L_1$

\begin{enumerate}
    \item $|\varepsilon|_a = 0 = |\varepsilon|_b$
    \item $|u|_a = |u|_b \Rightarrow |aub|_a = |u|_a + 1 = |aub|_b = |u|_b + 1$
\end{enumerate}

b) $L_1 \subseteq L$

\begin{enumerate}
    \item $|\varepsilon|_a = |\varepsilon|_b = 0$, $\varepsilon \in L_1$, $\varepsilon \in L$.
    \item Každé slovo $w \in L_1$ lze rozdělit na následující případy, které umožňují jeho postupné rozdělení až na
    prázdné slovo $\varepsilon$:
    \begin{enumerate}[label={}, noitemsep]
        \item \bb{Možnost 1:} $w$ začíná $a$ a končí $b$,
        \item \bb{Možnost 2:} $w$ začíná $b$ a končí $a$,
        \item \bb{Možnost 3:} $w$ začíná a končí tím stejným písmenem ($a$ nebo $b$).
    \end{enumerate}
\end{enumerate}

\begin{enumerate}[label={}]
    \item \bb{Možnost 1: $w = bua$}. Rozdělíme slovo $w$ na $w = bua$, kde $u$ je prostřední část slova splňující
    $|u|_a = |u|_b$. Podle definice pravidel $L$, pokud $u \in L$, pak $aub \in L$.
    \item \bb{Možnost 2: $w = aub$}. Rozdělíme slovo $w$ na $w = aub$, kde $u$ je prostřední část slova splňující
    $|u|_a = |u|_b$. Podle definice pravidel $L$, pokud $u \in L$, pak $bua \in L$.
    \item \bb{Možnost 3: $w = axa$, $w = bxb$}. Předpokládejme, že $w$ začíná i končí znakem $a$. Procházíme $a$ zleva
    doprava a hledáme první $b$, kde počet znaků $a$ od začátku do tohoto $b$ je stejný jako počet znaků $b$ (včetně
    tohoto $b$). Toto $b$ rozděluje $w$ na dvě části: $w = uv$, kde $u$ obsahuje první část $w$ (od prvního znaku do
    tohoto $b$ včetně), kde $|u|_a = |u|_b$, a $v$ obsahuje druhou část slova $w$, kde $|v|_a = |v|_b$. Podle pravidel
    $L$, pokud $u, v \in L$, pak $uv \in L$. \\
    Analogicky postupujume pro slovo začínající i končící znakem $b$.
\end{enumerate}

\newpage
\subsection{Práce na konečném automatu}
Je dán konečný automat $M$ tabulkou

\begin{multicols}{2}

\begin{tabular}{|r|c|c|}
    \hline
    & $a$ & $b$\\
    \hline
    \hline
    $1$            & $2$   & $1$\\
    $\leftarrow 2$ & $2$   & $1$\\
    $3$            & $7$   & $5$\\
    $\leftarrow 4$ & $7$   & $4$\\
    $\rightarrow 5$& $2$   & $4$\\
    $\leftarrow 6$ & $6$   & $3$\\
    $ 7$           & $7$   & $4$\\
    \hline
\end{tabular}

\columnbreak

\begin{enumerate}[noitemsep]
    \item Nakreslete stavový diagram automatu.
    \item Simulujte krok po kroku výpočet automatu nad slovem $bbaaab$.
    \item Z induktivní definice odvoďte $\delta^\star(2, bab)$.
\end{enumerate}

\end{multicols}

\begin{multicols}{2}
1.

    \begin{tikzpicture}

        \node[state, accepting] (2) {$2$};
        \node[state, below of=2] (1) {$1$};
        \node[state, initial, right of=2] (5) {$5$};
        \node[state, right of=5] (3) {$3$};
        \node[state, accepting, below of=5] (4) {$4$};
        \node[state, accepting, above of=5] (6) {$6$};
        \node[state, below of=3] (7) {$7$};

        \draw
            (1) edge[loop right] node{$b$} (1)
            (1) edge[bend left, above] node[left]{$a$} (2)
            (2) edge[bend left, below] node[right]{$b$} (1)
            (2) edge[loop above] node{$a$} (2)
            (3) edge[bend right, above] node{$b$} (5)
            (3) edge[bend left, below] node[right]{$a$} (7)
            (4) edge[loop below] node{$b$} (4)
            (4) edge[bend left, above] node{$a$} (7)
            (5) edge[bend right, below] node[left]{$b$} (4)
            (5) edge[bend right, above] node{$a$} (2)
            (6) edge[loop above] node{$a$} (6)
            (6) edge[bend left, above] node{$b$}(3)
            (7) edge[loop above] node{$a$} (7)
            (7) edge[bend left, below] node{$b$} (4)
            ;
    \end{tikzpicture}

\columnbreak

\begin{enumerate} %it's better with spacing between items
    \setItemnumber{2}
    \item $\rightarrow5-4-4-7-7-7-4 \rightarrow$
    \item ${\delta^\star(2, bab) = \delta(\delta^\star (2, ba), b) =} \delta(\delta(\delta(2,b),a),b)$.
\end{enumerate}

\end{multicols}

\subsection{Posuvný registr}
Navrhněte automat modelující posuvný registr, který provádí celočíselné dělení $4$ binárně zadaného čísla (číslo se čte
od nejvyššího řádu). O jaký typ automatu se jedná?
\begin{multicols}{2}

    \begin{tikzpicture}
        \node[state, initial] (q0) {$00$};
        \node[state, right of=q0] (q1) {$01$};
        \node[state, accepting, below of=q0] (q2) {$10$};
        \node[state, right of=q2] (q3) {$11$};

        \draw
            (q0) edge[bend left, above] node{$a$} (q1)
            (q0) edge[bend left, right] node{$b$} (q2)
            (q1) edge[bend left, below] node{$a$} (q0)
            (q1) edge[bend left, right] node{$b$} (q3)
            (q2) edge[bend right, below] node{$a$} (q3)
            (q2) edge[bend left, left] node{$b$} (q0)
            (q3) edge[bend right, above] node{$a$} (q2)
            (q3) edge[bend left, left] node{$b$} (q1)
            ;
    \end{tikzpicture}

\columnbreak
\raggedcolumns
Jedná se o \ii{Mealyho automat} $$M = (Q, \Sigma, Y, \delta, q_o, \lambda)\text{.}$$

\end{multicols}

\newpage
\subsection{Stavové diagramy pro DFA}
Pro uvedené automaty nakreslete stavový diagram. Najděte vlastnost $\mathcal{V}$, která charakterizuje slova přijímaná
daným automatem. Dokažte, že automat přijímá právě všechna slova s vlastností $\mathcal{V}$.

\begin{multicols}{3}
\raggedcolumns

\begin{tabular}{|r|c|c|}
    \hline
    & $0$ & $1$\\
    \hline
    \hline
    $\leftrightarrow q_0$& $q_1$   & $q_0$\\
    $q_1$                & $q_2$   & $q_1$\\
    $q_2$                & $q_0$   & $q_2$\\
    \hline
\end{tabular}

\begin{tikzpicture}
    \node[state, initial, accepting] (q0) at (0, 0) {$q_0$};
    \node[state] (q1) at (2, 0) {$q_1$};
    \node[state] (q2) at ($(q0)!0.5!(q1) + (0, -2)$) {$q_2$};

    \draw
        (q0) edge[loop above] node{$1$} (q0)
        (q0) edge[bend left, above] node{$0$} (q1)
        (q1) edge[loop above] node{$1$} (q1)
        (q1) edge[bend left, below] node[left]{$0$} (q2)
        (q2) edge[bend left, above] node[right]{$0$} (q0)
        (q2) edge[loop below] node{$1$} (q2)
        ;
\end{tikzpicture}

$w \in L$ iff $|w|_0$ je dělitelný 3.

Invarianty:
\begin{itemize}[noitemsep]
    \setlength{\itemindent}{-1em}
    \item $q_0$: \texttt{in}, \texttt{out}, $|w|_0 = 3k$,
    \item $q_1$: $|w|_0 = 3k + 1$,
    \item $q_2$: $|w|_0 = 3k + 2$, ${k \in \mathbb{Z} \text{.}}$
\end{itemize}

\columnbreak
\begin{tabular}{|r|c|c|}
    \hline
    & $0$ & $1$\\
    \hline
    \hline
    $\rightarrow q_0$& $q_1$   & $q_0$\\
    $\leftarrow q_1$ & $q_2$   & $q_1$\\
    $\leftarrow q_2$ & $q_0$   & $q_2$\\
    \hline
\end{tabular}

\begin{tikzpicture}
    \node[state, initial] (q0) at (0, 0) {$q_0$};
    \node[state, accepting] (q1) at (2, 0) {$q_1$};
    \node[state, accepting] (q2) at ($(q0)!0.5!(q1) + (0, -2)$) {$q_2$};

    \draw
        (q0) edge[loop above] node{$1$} (q0)
        (q0) edge[bend left, above] node{$0$} (q1)
        (q1) edge[loop above] node{$1$} (q1)
        (q1) edge[bend left, below] node[left]{$0$} (q2)
        (q2) edge[bend left, above] node[right]{$0$} (q0)
        (q2) edge[loop below] node{$1$} (q2)
        ;
\end{tikzpicture}

$w \in L$ iff $|w|_0$ není dělitelný 3.

Invarianty:
\begin{itemize}[noitemsep]
    \setlength{\itemindent}{-1em}
    \item $q_0$: \texttt{in}, $|w|_0 = 3k$,
    \item $q_1$: \texttt{out}, ${|w|_0 = 3k + 1}$,
    \item $q_2$: \texttt{out}, ${|w|_0 = 3k + 2\text{, } k \in \mathbb{Z} \text{.}}$
\end{itemize}

\columnbreak
\begin{tabular}{|r|c|c|}
    \hline
    & $0$ & $1$\\
    \hline
    \hline
    $\rightarrow q_0$& $q_0$   & $q_1$\\
    $q_1$            & $q_0$   & $q_2$\\
    $\leftarrow q_2$ & $q_0$   & $q_2$\\
    \hline
\end{tabular}

\begin{tikzpicture}
    \node[state, initial] (q0) at (0, 0) {$q_0$};
    \node[state] (q1) at (2, 0) {$q_1$};
    \node[state, accepting] (q2) at ($(q0)!0.5!(q1) + (0, -2)$) {$q_2$};

    \draw
        (q0) edge[loop above] node{$0$} (q0)
        (q0) edge[bend left, above] node{$1$} (q1)
        (q1) edge[bend left, above] node{$0$} (q0)
        (q1) edge[bend left, below] node[left]{$1$} (q2)
        (q2) edge[bend left, above] node[right]{$0$} (q0)
        (q2) edge[loop below] node{$1$} (q2)
        ;
\end{tikzpicture}

$w \in L$ iff $w$ končí 11.

Invarianty:
\begin{itemize}[noitemsep]
    \setlength{\itemindent}{-1em}
    \item $q_0$: \texttt{in}, končí $0$,
    \item $q_1$: nekončí $11$, končí $1$,
    \item $q_2$: \texttt{out}, končí $11$.
\end{itemize}
\medspace


\end{multicols}