\section{Třetí cvičení}

\subsection{Nerodova věta a Pumping lemma}
Pomocí Nerodovy věty a pomocí Pumping lemmatu dokažte, že jazyk $L = \{0^n 1^m \mid n > m \geq 0\}$ není regulární.

\noindent
Definice \textbf{Nerodovy věty.} $L$ je regulární iff existují ekvivalence $T$ na $\Sigma^\star$ taková, že:
\begin{enumerate}[1), noitemsep]
    \item $L$ je sjednocení některých tříd $T$
    \item pokud $uTv$, tak $uwTvw$ pro každé $w \in \Sigma^\star$
    \item $T$ má konečný počet tříd
\end{enumerate}
Důkaz sporem:
Kdyby existovala $T$ na $\{0,1\}^\star$.

$1, 1^2, 1^3, ..., 1^i, ..., 1^n, ... = \{1^j \mid j \geq 1\}$ je nekonečná posloupnost z $\{0,1\}$.

$T$ má konečně mnoho tříd, tudíž $0^i T 0^j$ pro nějaké $i \not= j, i>j$.

Protože platí 2), tak $0^i w T 0^j w$ pro $w \in \{0,1\}^\star$.

Zvolme $w = 1^{i-1}$. Pak $\underbrace{0^i 1^{i-1}}_{\substack{i \geq i-1 \\ \in L}} T 
\underbrace{0^j 1^{i-1}}_{\substack{i-1 \geq j  \\\not\in L }}$. Tedy $L$ není regulární. $\blacksquare$

\noindent
Definice \textbf{Pumping lemmatu.} Když $L$ je regulární, tak existuje $n \in L, n \geq 1$, takové, že každé ${u \in L},\\
|u| > n$ je 
možné rozložit na 3 slova $u = xwy$ splňující:
\begin{enumerate}[1), noitemsep]
    \item $|xw| \leq n$
    \item $w \not= \varepsilon$
    \item $xw^i y \in L, \forall i = 0, 1, 2, ...$
\end{enumerate}
Důkaz sporem:
Kdyby $L$ byl regulární, tak existuje $n$ s vlastnostmi z Pumping lemma.

Zvolíme $u = 0^{n+1} 1^n$.

Kdyby $u = xwy$, tak 1) vlastnost říká, že $xw = 0^l, l \leq n$. Zároveň musí platit 2), tedy $w = 0^k, 1 \leq k \leq l$.
Když teď napumpujeme $xw^i y$, například $i=0$, dostaneme $xw^0 y = 0^{n+1-k} 1^{n} \not\in L$.

Tedy $L$ není regulární. $\blacksquare$

\subsection{Nalezení slova rozlišujícího stavy}
Je dán DFA tabulkou:
\[
\begin{tabular}{|r|c|c|}
    \hline
    & $a$ & $b$\\
    \hline
    \hline
    $\leftrightarrow 0$& $1$ & $2$\\
    $1$                & $3$ & $0$\\
    $2$                & $4$ & $5$\\
    $3$                & $0$ & $2$\\
    $4$                & $2$ & $5$\\
    $5$                & $0$ & $3$\\
    \hline
\end{tabular}
\]
Najděte slovo nejkratší délky, jestliže existuje, které rozliší
\begin{enumerate}[a), noitemsep]
    \item stavy 3 a 5.
    \item stavy 2 a 4.
\end{enumerate}
To, že slovo $u$ rozliší dva stavy znamená, že přechodová funkce při práci nad slovem $u$ převede jeden ze stavů do 
koncového stavu a druhý do stavu, který není koncový.

a) $\delta(3,a) = 0, \delta(5,a)=0 \implies \delta^\star(3, au) = \delta^\star (5, au)$.
Slovo nezačíná $a$.

$\delta(3,b) = 2, \delta(5, b) = 3$.
\[
\begin{rcases*}
\delta^\star(3,ba) = \delta(2,a) = 4 \not\in F\\
\delta^\star(5,ba) = \delta(3,a) = 0 \in F
\end{rcases*} \implies u = ba
\]

b) $\delta(2,a) = 4, \delta(4,a) =2$. $\delta(2,b)=5, \delta(4,b)=5$. Tyto stavy nelze rozlišit.

\subsection{Návrh a redukce DFA podle jazyka}
Navrhněte DFA, který přijímá jazyk $L$ skládající se ze všech slov nad abecedou $\Sigma = \{0, 1\}$, která začínají 
$1100$ a končí $000$. Navržený automat redukujte.

\begin{tikzpicture}
    \node[state, initial] (0) {$q_0$};
    \node[state, right of=0] (1) {$q_1$};
    \node[state, right of=1] (2) {$q_2$};
    \node[state, right of=2] (3) {$q_3$};
    \node[state, right of=3] (4) {$q_4$};
    \node[state, below of=4] (5) {$q_5$};
    \node[state, left of=5] (6) {$q_6$};
    \node[state, left of=6] (7) {$q_7$};
    \node[state, accepting, left of=7] (8) {$q_8$};
    \node[state, below of=0] (9) {chyba};

    \draw
        (0) edge[bend right, left] node{$0$} (9)
        (0) edge[bend left, above] node{$1$} (1)
        (1) edge[above] node{$0$} (9)
        (1) edge[bend left, above] node{$1$} (2)
        (2) edge[bend left, above] node{$0$} (3)
        (2) edge[above] node{$1$} (9)
        (3) edge[bend left, above] node{$0$} (4)
        (3) edge[above] node{$1$} (9)
        (4) edge[above] node{$0$}(8)
        (4) edge[bend left] node[right]{$1$} (5)
        (5) edge[bend right, above] node{$0$} (6)
        (5) edge[loop right] node{$1$} (5)
        (6) edge[bend right, above] node{$0$} (7)
        (6) edge[bend right, above] node{$1$} (5)
        (7) edge[bend right, above] node{$0$} (8)
        (7) edge[bend right, above] node{$1$} (5)
        (8) edge[loop below] node{$0$} (7)
        (8) edge[bend right, above] node{$1$} (5)
        (9) edge[loop left] node{$0,1$} (9)
        ;
\end{tikzpicture}

Invarianty: 
\begin{itemize}[noitemsep]
    \item $q_0$: \texttt{in}, 
    \item $q_1$: začíná $1$, končí $1$,  
    \item $q_2$: začíná $11$, končí $11$,  
    \item $q_3$: začíná $110$, končí $110$,  
    \item $q_4$: začíná $1100$, končí $1100$,  
    \item $q_5$: začíná $1100$, končí $1$,  
    \item $q_6$: začíná $1100$, končí $10$,  
    \item $q_7$: začíná $1100$, končí $100$,  
    \item $q_8$: začíná $1100$, končí $000$, \texttt{out},  
    \item chyba: nezačíná $1100$.  
\end{itemize}

Redukce podmnožinovou konstrukcí: 
\[
\begin{tabular}{|r r|c c|c|c c|c|c c|c|c c|c|c c|c|c c|c|c c| c|}
    \hline
    & & $0$ & $1$ & $\sim_0$ & $0$ & $1$ & $\sim_1$ & $0$ & $1$ & $\sim_2$ & $0$ & $1$ & $\sim_3$ & $0$ & $1$ & $\sim_4$ & $0$ & $1$ & $\sim_5$ & $0$ & $1$ & $\sim_6$\\
    \hline
    \hline
    $\rightarrow$ & $q_0$ & ch &$1$ &$O$ &$O$ &$O$ &$O$ &$O$ &$O$ &$O$ &$O$ &$O$ &$O$ &$O$ &$O$ &$O$ &$O$ &$D$ &$G$ &$O$ &$D$ &$G$\\
              & $q_1$ & ch &$2$ &$O$ &$O$ &$O$ &$O$ &$O$ &$O$ &$O$ &$O$ &$O$ &$O$ &$O$ &$C$ &$D$ &$O$ &$C$ &$D$ &$O$ &$C$ &$D$\\
              & $q_2$ &$3$ & ch &$O$ &$O$ &$O$ &$O$ &$O$ &$O$ &$O$ &$B$ &$O$ &$C$ &$B$ &$O$ &$C$ &$B$ &$O$ &$C$ &$B$ &$O$ &$C$\\
              & $q_3$ &$4$ & ch &$O$ &$O$ &$O$ &$O$ &$A$ &$O$ &$B$ &$A$ &$O$ &$B$ &$A$ &$O$ &$B$ &$A$ &$O$ &$B$ &$A$ &$O$ &$B$\\
              & $q_4$ &$8$ &$5$ &$O$ &$K$ &$O$ &$A$ &$K$ &$O$ &$A$ &$K$ &$O$ &$A$ &$K$ &$C$ &$A$ &$K$ &$E$ &$A$ &$K$ &$E$ &$A$\\
              & $q_5$ &$6$ &$5$ &$O$ &$O$ &$O$ &$O$ &$O$ &$O$ &$O$ &$B$ &$O$ &$C$ &$B$ &$C$ &$E$ &$F$ &$E$ &$E$ &$F$ &$E$ &$E$\\
              & $q_6$ &$7$ &$5$ &$O$ &$O$ &$O$ &$O$ &$A$ &$O$ &$B$ &$A$ &$O$ &$B$ &$A$ &$C$ &$F$ &$A$ &$E$ &$F$ &$A$ &$E$ &$F$\\
              & $q_7$ &$8$ &$5$ &$O$ &$K$ &$O$ &$A$ &$K$ &$O$ &$A$ &$K$ &$O$ &$A$ &$K$ &$C$ &$A$ &$K$ &$E$ &$A$ &$K$ &$E$ &$A$\\
    $\leftarrow$  & $q_8$ &$8$ &$5$ &$K$ &$K$ &$O$ &$K$ &$K$ &$O$ &$K$ &$K$ &$O$ &$K$ &$K$ &$C$ &$K$ &$K$ &$E$ &$K$ &$K$ &$E$ &$K$\\
              &  ch   & ch & ch &$O$ &$O$ &$O$ &$O$ &$O$ &$O$ &$O$ &$O$ &$O$ &$O$ &$O$ &$O$ &$O$ &$O$ &$O$ &$O$ &$O$ &$O$ &$O$\\
    \hline
\end{tabular}
\]

Redukovaný automat: 

\begin{tikzpicture}
    \node[state, initial] (0) {$G$};
    \node[state, right of=0] (1) {$D$};
    \node[state, right of=1] (2) {$C$};
    \node[state, right of=2] (3) {$B$};
    \node[state, right of=3] (4) {$A$};
    \node[state, accepting, right of=4] (5) {$K$};
    \node[state, below of=4] (6) {$E$};
    \node[state, below of=3] (7) {$F$};
    \node[state, below of=0] (9) {O};

    \draw
        (0) edge[bend right, left] node{$0$} (9)
        (0) edge[bend left, above] node{$1$} (1)
        (1) edge[above] node{$0$} (9)
        (1) edge[bend left, above] node{$1$} (2)
        (2) edge[bend left, above] node{$0$} (3)
        (2) edge[above] node{$1$} (9)
        (3) edge[bend left, above] node{$0$} (4)
        (3) edge[above] node{$1$} (9)
        (4) edge[bend left, above] node{$0$}(5)
        (4) edge[bend right] node[right]{$1$} (6)
        (5) edge[left] node{$1$} (6)
        (5) edge[loop right] node{$0$} (5)
        (6) edge[bend right, above] node{$0$} (7)
        (6) edge[loop right] node{$1$} (6)
        (7) edge[bend left] node{$0$} (4)
        (7) edge[bend right, above] node{$1$} (6)
        (9) edge[loop below] node{$0,1$} (9)
        ;
\end{tikzpicture}


\subsection{Srovnání dvou DFA automatů}
Jsou dány dva automaty. Rozhodněte, zda jsou ekvivalentní, tj. zda přijímají stejný jazyk.
\[
M_1:
\begin{tabular}{|r|c|c|}
    \hline
    & $a$ & $b$\\
    \hline
    \hline
    $\leftrightarrow 0$& $0$ & $5$\\
    $1$                & $1$ & $3$\\
    $2$                & $2$ & $7$\\
    $3$                & $3$ & $2$\\
    \hline
    $\leftarrow 4$     & $6$ & $1$\\
    $5$                & $5$ & $1$\\
    $\leftarrow 6$     & $4$ & $2$\\
    $7$                & $7$ & $0$\\
    \hline
\end{tabular}
\quad
M_2:
\begin{tabular}{|r|c|c|}
    \hline
    & $a$ & $b$\\
    \hline
    \hline
    $A$                & $H$ & $G$\\
    $B$                & $B$ & $A$\\
    $C$                & $E$ & $D$\\
    $D$                & $D$ & $B$\\
    \hline
    $E$                & $C$ & $D$\\
    $F$                & $F$ & $E$\\
    $\leftrightarrow G$& $G$ & $F$\\
    $H$                & $A$ & $G$\\
    \hline
\end{tabular}
\]

Nejdříve odstraníme nedosažitelné vztahy a pak provedeme redukci. 

Odstranění nedosažitelných vztahů:
\[
M_1:
\begin{tabular}{|r|c|c|}
    \hline
    & $a$ & $b$\\
    \hline
    \hline
    $\leftrightarrow 0$& $0$ & $5$\\
    $1$                & $1$ & $3$\\
    $2$                & $2$ & $7$\\
    $3$                & $3$ & $2$\\
    \hline
    $5$                & $5$ & $1$\\
    $7$                & $7$ & $0$\\
    \hline
\end{tabular}
\quad
M_2:
\begin{tabular}{|r|c|c|}
    \hline
    & $a$ & $b$\\
    \hline
    \hline
    $A$                & $H$ & $G$\\
    $B$                & $B$ & $A$\\
    $C$                & $E$ & $D$\\
    $D$                & $D$ & $B$\\
    \hline
    $E$                & $C$ & $D$\\
    $F$                & $F$ & $E$\\
    $\leftrightarrow G$& $G$ & $F$\\
    $H$                & $A$ & $G$\\
    \hline
\end{tabular}
\]

A teď zredukovat oba automaty.

$M_1$:

\begin{tikzpicture}
    \node[state] (7) at (0, 1) {$7$}; % 7
    \node[state] (3) at (3.75, 1) {$3$}; % 3 
    \node[state] (2) at (1.875, 0) {$2$}; % 2 
    \node[state] (5) at (1.875, 4) {$5$}; % 5 
    \node[state] (1) at (3.75, 3) {$1$}; % 1 
    \node[state, initial, accepting] (0) at (0, 3) {$0$}; % 0 

    \draw
        (0) edge[loop above] node{$a$} (0)
        (0) edge[bend left=18] node{$b$} (5)
        (5) edge[loop above] node{$a$} (5)
        (5) edge[bend left=18] node{$b$} (1)
        (1) edge[loop right] node{$a$} (1)
        (1) edge[bend left=18] node{$b$} (3)
        (3) edge[loop right] node{$a$} (3)
        (3) edge[bend left=18] node[below]{$b$} (2)
        (2) edge[loop above] node{$a$} (2)
        (2) edge[bend left=18] node{$b$} (7)
        (7) edge[loop below] node{$a$} (7)
        (7) edge[bend left=18] node{$b$} (0)
        ;
\end{tikzpicture}

$M_1$ je již redukovaný.

$M_2$:

\[
\begin{tabular}{|r r|c c|c|c c|c|c c|c|c c|c|c c|c|c c|c|}
    \hline
    & & $a$ & $b$ & $\sim_0$ & $a$ & $b$ & $\sim_1$ & $a$ & $b$ & $\sim_2$ & $a$ & $b$ & $\sim_3$ & $a$ & $b$ & $\sim_4$ & $a$ & $b$ & $\sim_5$\\
    \hline
    \hline
                     & $A$ &$H$ &$G$ &$O$ &$O$ &$K$ &$A$ &$A$ &$K$ &$A$ &$A$ &$K$ &$A$ &$A$ &$K$ &$A$ &$A$ &$K$ &$A$\\
                     & $B$ &$B$ &$A$ &$O$ &$O$ &$O$ &$O$ &$O$ &$A$ &$B$ &$B$ &$A$ &$B$ &$B$ &$A$ &$B$ &$B$ &$A$ &$B$\\
                     & $C$ &$E$ &$D$ &$O$ &$O$ &$O$ &$O$ &$O$ &$O$ &$O$ &$O$ &$O$ &$O$ &$O$ &$C$ &$O$ &$D$ &$C$ &$E$\\
                     & $D$ &$D$ &$B$ &$O$ &$O$ &$O$ &$O$ &$O$ &$O$ &$O$ &$O$ &$B$ &$C$ &$C$ &$B$ &$C$ &$C$ &$B$ &$C$\\
                     & $E$ &$C$ &$D$ &$O$ &$O$ &$O$ &$O$ &$O$ &$O$ &$O$ &$O$ &$O$ &$O$ &$O$ &$C$ &$D$ &$O$ &$C$ &$D$\\
                     & $F$ &$F$ &$E$ &$O$ &$O$ &$O$ &$O$ &$O$ &$O$ &$O$ &$O$ &$O$ &$O$ &$O$ &$O$ &$O$ &$O$ &$D$ &$O$\\
    $\leftrightarrow$& $G$ &$G$ &$F$ &$K$ &$K$ &$O$ &$K$ &$K$ &$O$ &$K$ &$K$ &$O$ &$K$ &$K$ &$O$ &$K$ &$K$ &$O$ &$K$\\
                     & $H$ &$A$ &$G$ &$O$ &$O$ &$K$ &$A$ &$A$ &$K$ &$A$ &$A$ &$K$ &$A$ &$A$ &$K$ &$A$ &$A$ &$K$ &$A$\\
    \hline
\end{tabular}
\]

Protože má každý řádek svou vlastní třídu, $M_2$ je již redukovaný.

\newpage

\begin{multicols}{2}
    $M_1$:\\
    \begin{tikzpicture}
        \node[state] (7) at (0, 1) {$7$}; % 7
        \node[state] (3) at (3.75, 1) {$3$}; % 3 
        \node[state] (2) at (1.875, 0) {$2$}; % 2 
        \node[state] (5) at (1.875, 4) {$5$}; % 5 
        \node[state] (1) at (3.75, 3) {$1$}; % 1 
        \node[state, initial, accepting] (0) at (0, 3) {$0$}; % 0 

        \draw
            (0) edge[loop above] node{$a$} (0)
            (0) edge[bend left=18] node{$b$} (5)
            (5) edge[loop above] node{$a$} (5)
            (5) edge[bend left=18] node{$b$} (1)
            (1) edge[loop right] node{$a$} (1)
            (1) edge[bend left=18] node{$b$} (3)
            (3) edge[loop right] node{$a$} (3)
            (3) edge[bend left=18] node[below]{$b$} (2)
            (2) edge[loop above] node{$a$} (2)
            (2) edge[bend left=18] node{$b$} (7)
            (7) edge[loop left] node{$a$} (7)
            (7) edge[bend left=18] node{$b$} (0)
            ;
    \end{tikzpicture}

\columnbreak

    $M_2$:\\
    \begin{tikzpicture}
        \node[state] (a) at (0, 1) {$A$}; % k 
        \node[state] (c) at (3.75, 1) {$C$}; % c 
        \node[state] (b) at (1.875, 0) {$B$}; % b 
        \node[state] (o) at (1.875, 4) {$O$}; % o
        \node[state] (d) at (3.75, 3) {$D$}; % d  
        \node[state, initial, accepting] (k) at (0, 3) {$K$}; % k

        \draw
            (k) edge[loop above] node{$a$} (k)
            (k) edge[bend left=18] node{$b$} (o)
            (o) edge[loop above] node{$a$} (o)
            (o) edge[bend left=18] node{$b$} (d)
            (d) edge[bend left=18] node{$a$} (o)
            (d) edge[bend left=18] node{$b$} (c)
            (c) edge[loop right] node{$a$} (c)
            (c) edge[bend left=18] node{$b$} (b)
            (b) edge[loop above] node{$a$} (b)
            (b) edge[bend left=18] node{$b$} (a)
            (a) edge[loop left] node{$a$} (a)
            (a) edge[bend left=18] node{$b$} (k)
            ;
    \end{tikzpicture}

\end{multicols}

Nejsou ekvivalnetní - $M_1$ přijme např. slovo $bbabbbb$, které $M_2$ nepřijme. 

\newpage
\subsection{Návrh a redukce DFA podle jazyka}
Navrhněte DFA, který přijímá $L$ nad abecedou $\{a,b\}$, kde $L$ obsahuje právě všechna slova $w$ taková, že
\begin{itemize}[noitemsep]
    \item druhý znak slova $w$ je $a$,
    \item předposlední znak slova $w$ je $b$,
    \item $|w| \geq 3$.
\end{itemize}
Výsledný DFA redukujte.

\begin{tikzpicture}
    \node[state, initial] (0) {$q_0$};
    \node[state, above right of=0] (1) {$q_1$};
    \node[state, below right of=1] (2) {$q_2$};
    \node[state, below right of=2] (3) {$q_3$};
    \node[state, accepting, above right of=2] (4) {$q_4$};
    \node[state, accepting, below right of=4] (5) {$q_5$};
    \node[state, below right of=0] (9) {chyba};

    \draw
        (0) edge[bend left] node{$a,b$} (1)
        (1) edge[bend left, above] node{$a$} (2)
        (1) edge[] node[left]{$b$} (9)
        (2) edge[loop left] node{$a$}(2)
        (2) edge[bend right, above] node{$b$}(3)
        (3) edge[bend right, above] node{$b$} (5)
        (3) edge[bend left] node[left]{$a$} (4)
        (4) edge[bend left] node[left]{$b$} (3)
        (4) edge[bend right] node[left]{$a$} (2)
        (5) edge[loop above] node{$b$} (5)
        (5) edge[bend right] node[right]{$a$} (4)
        (9) edge[loop right] node{$a,b$} (9)
        ;
\end{tikzpicture}

Invarianty: 
\begin{itemize}[noitemsep]
    \item $q_0$: $|w| < 3$, \texttt{in}, 
    \item chyba: druhý znak je $b$,  
    \item $q_1$: $|w| < 3$, druhý znak je $\varepsilon$, předposlední znak je $\varepsilon$,   
    \item $q_2$: $|w| < 3$, druhý znak je $a$, předposlední znak je $a,b$, 
    \item $q_3$: $|w| \geq 3$, druhý znak je $a$, předposlední znak je $a$, 
    \item $q_4$: $|w| \geq 3$, druhý znak je $a$, předposlední znak je $b$, končí $ba$, \texttt{out}, 
    \item $q_5$: $|w| \geq 3$, druhý znak je $a$, předposlední znak je $b$, končí $bb$, \texttt{out}. 
\end{itemize}

Redukce podmnožinovou konstrukcí: 
\[
\begin{tabular}{|r r|c c|c|c c|c|c c|c|c c|c|c c|c|}
    \hline
    & & $a$ & $b$ & $\sim_0$ & $a$ & $b$ & $\sim_1$ & $a$ & $b$ & $\sim_2$ & $a$ & $b$ & $\sim_3$ & $a$ & $b$ & $\sim_4$\\
    \hline
    \hline
    $\rightarrow$ & $q_0$ &$q_1$ &$q_1$ &$O$ &$O$ &$O$ &$O$ &$O$ &$O$ &$O$ &$O$ &$O$ &$O$ &$D$ &$D$ &$E$\\
                  & $q_1$ &$q_2$ &Chyba &$O$ &$O$ &$O$ &$O$ &$O$ &$O$ &$O$ &$C$ &$O$ &$D$ &$C$ &$O$ &$D$\\
                  & $q_2$ &$q_2$ &$q_3$ &$O$ &$O$ &$O$ &$O$ &$O$ &$A$ &$C$ &$C$ &$A$ &$C$ &$C$ &$A$ &$C$\\
                  & $q_3$ &$q_4$ &$q_5$ &$O$ &$K$ &$K$ &$A$ &$B$ &$K$ &$A$ &$B$ &$K$ &$A$ &$B$ &$K$ &$A$\\
    $\leftarrow$  & $q_4$ &$q_2$ &$q_3$ &$K$ &$O$ &$O$ &$B$ &$O$ &$A$ &$B$ &$C$ &$A$ &$B$ &$C$ &$A$ &$B$\\
    $\leftarrow$  & $q_5$ &$q_4$ &$q_5$ &$K$ &$K$ &$K$ &$K$ &$B$ &$K$ &$K$ &$B$ &$K$ &$K$ &$B$ &$K$ &$K$\\
                  & Chyba &Chyba &Chyba &$O$ &$O$ &$O$ &$O$ &$O$ &$O$ &$O$ &$O$ &$O$ &$O$ &$O$ &$O$ &$O$\\
    \hline
\end{tabular}
\]
Protože každý řádek má svou vlastní třídu, původní DFA je již redukovaný.
