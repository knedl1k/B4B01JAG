\section{Třetí cvičení}

\subsection{Příklad}
Pomocí Nerodovy věty a pomocí pumping lemmatu dokažte, že jazyk $L = \{0^n 1^m \mid n > m \geq 0\}$ není regulární.

\subsection{Příklad}
Je dán DFA tabulkou:
\[
\begin{tabular}{|r|c|c|}
    \hline
    & $a$ & $b$\\
    \hline
    \hline
    $\leftrightarrow 0$& $1$ & $2$\\
    $1$                & $3$ & $0$\\
    $2$                & $4$ & $5$\\
    $3$                & $0$ & $2$\\
    $4$                & $2$ & $5$\\
    $5$                & $0$ & $3$\\
    \hline
\end{tabular}
\]
Najděte slovo nejkratší délky, jestliže existuje, které rozliší
\begin{enumerate}[a), noitemsep]
    \item stavy 3 a 5.
    \item stavy 2 a 4.
\end{enumerate}
To, že slovo $u$ rozliší dva stavy znamená, že přechodová funkce při práci nad slovem $u$ převede jeden ze stavů do koncového stavu a druhý do stavu, který není koncový.

\subsection{Příklad}
Navrhněte DFA, který přijímá jazyk $L$ skládající se ze všech slov nad abecedou $\Sigma = \{0, 1\}$, která začínají $1100$ a končí $000$. Navržený automat redukujte.

\subsection{Příklad}
Jsou dány dva automaty. Rozhodněte, zda jsou ekvivalentní, tj. zda přijímají stejný jazyk.
\[
M_1:
\begin{tabular}{|r|c|c|}
    \hline
    & $a$ & $b$\\
    \hline
    \hline
    $\leftrightarrow 0$& $0$ & $5$\\
    $1$                & $1$ & $3$\\
    $2$                & $2$ & $7$\\
    $3$                & $3$ & $2$\\
    \hline
    $\leftarrow 4$     & $6$ & $1$\\
    $5$                & $5$ & $1$\\
    $\leftarrow 6$     & $4$ & $2$\\
    $7$                & $7$ & $0$\\
    \hline
\end{tabular}
\quad
M_2:
\begin{tabular}{|r|c|c|}
    \hline
    & $a$ & $b$\\
    \hline
    \hline
    $A$                & $H$ & $G$\\
    $B$                & $B$ & $A$\\
    $C$                & $E$ & $D$\\
    $D$                & $D$ & $B$\\
    \hline
    $E$                & $C$ & $D$\\
    $F$                & $F$ & $E$\\
    $\leftrightarrow G$& $G$ & $F$\\
    $H$                & $A$ & $G$\\
    \hline
\end{tabular}
\]

\subsection{Příklad}
Navrhněte DFA, který přijímá $L$ nad abecedou $\{a,b\}$, kde $L$ obsahuje právě všechna slova $w$ taková, že
\begin{itemize}[noitemsep]
    \item druhý znak slova $w$ je $a$,
    \item předposlední znak slova $w$ je $b$,
    \item $|w| \geq 3$.
\end{itemize}
Výsledný DFA redukujte.