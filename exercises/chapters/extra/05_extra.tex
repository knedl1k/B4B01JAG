\section{Páté cvičení - BONUS}

\subsection{Příklad}
Zjednodušte následující regulární výrazy:
\begin{itemize}[a), noitemsep]
    \item $\varepsilon + 0 + 1 + (\varepsilon + 0 + 1)(\varepsilon + 0 + 1)^\star (\varepsilon + 0 + 1)$.
    \item $1^\star + 1^\star 0 (\varepsilon + 0 + 1)^\star \emptyset$.
\end{itemize}


\subsection{Příklad}
Navrhněte redukovaný DFA $M$, který přijímá jazyk $L$ nad $\Sigma = \{0,1\}$, kde
\[L = \{w \mid |w|_0 \text{ je sudé a za každým symbolem } 1 \text{ je symbol } 0\}\text{.}\]
Postupujte buď součinovou konstrukcí nebo přímo. V druhém případě řádně zdůvodněte, proč $M$ opravdu přijímá jazyk $L$.


\subsection{Příklad}
Navrhněte redukovaný DFA $M$, který přijímá jazyk $L$ nad $\Sigma = \{0,1\}$, kde 
\[L = \{w \mid w \text{ začíná } 10 \text{ nebo končí } 01\}\text{.}\]
Zdůvodněte, proč $M$ přijímá jazyk $L$.