\section{Jedenácté cvičení}

\subsection{Stavový diagram zásobníkového automatu, práce na automatu}
Je dán zásobníkový automat $A = (Q, \Sigma, \Gamma, \delta, q_0, Z_0, F)$, kde jednotlivé části jsou 
$Q = \{q_0, q_1, q_2, q_f\}$, $\Sigma = \{a, b\}$, $\Gamma =\{Z_0, X\}$ a přechodová funkce je daná tabulkou
\[
\begin{tabular}{|r|c|c|c|c|c|c|}
    \hline
    & $(a, Z_0)$ & $(a, X)$ & $(b, Z_0)$ & $(b, X)$ & $(\varepsilon, Z_0)$ & $(\varepsilon, X)$\\
    \hline
    \hline
    $\rightarrow q_0$& $(q_0, X Z_0)$ & $(q_0, XX)$ & $(q_1, Z_0)$ & $(q_1, \varepsilon)$ & $(q_f, \varepsilon)$ & $-$\\
    $q_1$            & $-$            & $-$         & $(q_1, Z_0)$ & $(q_1, \varepsilon)$ & $(q_f, \varepsilon)$ & $-$\\
    $\leftarrow q_f$ & $-$& $-$& $-$ & $-$ & $-$ & $-$\\
    \hline
\end{tabular}
\]

\begin{enumerate}[a), noitemsep]
    \item Nakreslete stavový diagram zásobníkového automatu $A$.
    \item Ukažte práci zásobníkového automatu nad slovem $aabba$ a slovem $abbbb$.
    \item Charakterizujte jazyk $L$, který tento zásobníkový automat přijímá. Tvrzení zdůvodněte.
\end{enumerate}

\begin{multicols}{2}
    Stavový diagram automatu $A$.

    \begin{tikzpicture}
        \node[state, initial] (0) {$q_0$};
        \node[state, right of=0] (1) {$q_1$};
        \node[state, accepting, below of=0] (f) {$q_f$};

        \draw
            (0) edge[loop above] node[align=center] {$a,Z_0 / X Z_0$ \\ $a,X / XX$} (0)
            (0) edge[bend right, below] node {$b, X / \varepsilon$} (1)
            (0) edge[bend right, above] node {$b, Z_0 / Z_0$} (1)
            (0) edge[bend right, left] node{$\varepsilon, Z_0 / \varepsilon$} (f)
            (1) edge[loop above] node[align=center] {$b,Z_0 / Z_0$ \\ $b,X / \varepsilon$} (1)
            (1) edge[bend left, right] node{$\varepsilon, Z_0 / \varepsilon$} (f)

            ;
    \end{tikzpicture}

    \columnbreak

    \noindent
    Práce nad slovem $w_1 = aabba$.

    \noindent
    $(q_0, aabba, Z_0) \vdash (q_0, abba, X Z_0) \vdash (q_0, bba, XXZ_0) \vdash (q_1, ba, X Z_0) \vdash (q_1, a, Z_0)$. 
    Konec, neúspěch.

    \noindent
    Práce nad slovem $w_2 = abbbb$.

    \noindent
    $(q_0, abbbb, Z_0) \vdash (q_0, bbbb, X Z_0) \vdash (q_1, bbb, Z_0) \vdash (q_1, bb, Z_0) \vdash (q_1, b, Z_0)
    \vdash (q_1, \varepsilon, Z_0) \vdash (q_f, \varepsilon, \varepsilon)$. \\Konec, úspěch.
\end{multicols}

$L(A) \stackrel{?}{=} \overbrace{\{a^i b^j \mid 0 \leq i \leq j\}}^L$

Důkaz.

a) $L \subseteq L(A)$
\begin{itemize}[leftmargin=*]
    \item $i=j=0 \text{: } (q_0, \varepsilon, Z_0) \vdash (q_f, \varepsilon, \varepsilon)$
    \item $0=i < j \text{: } (q_0, b^j, Z_0) \vdash (q_1, b^{j-1}, Z_0) \vdash^{(j-1)} (q_1, \varepsilon, Z_0) 
    \vdash (q_f, \varepsilon, \varepsilon)$
    \item ${0<i \leq j \text{: } (q_0, a^i b^j, Z_0) \vdash^i (q_0, b^{i+k}, X^i Z_0) \vdash (q_0, b^{i+k-1}, X^{i-1} Z_0)
    \vdash^{(i-1)} (q_1, b^k, Z_0) \vdash^k (q_1, \varepsilon, Z_0) \vdash (q_f, \varepsilon, \varepsilon),}\\
    i+k = j, k=j-i \geq 0$.
\end{itemize}

b) $L(A) = N(A) \subseteq L$

$w \in L(A)$ % TODO: dodělat důkaz

