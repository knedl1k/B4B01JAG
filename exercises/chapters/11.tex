\section{Jedenácté cvičení}

\subsection{Příklad}
Je dán zásobníkový automat $A = (Q, \Sigma, \Gamma, \delta, q_0, Z_0, F)$, kde jednotlivé části jsou 
$Q = \{q_0, q_1, q_2, q_f\}$, $\Sigma = \{a, b\}$, $\Gamma =\{Z_0, X\}$ a přechodová funkce je daná tabulkou
\[
\begin{tabular}{|r|c|c|c|c|c|c|}
    \hline
    & $(a, Z_0)$ & $(a, X)$ & $(b, Z_0)$ & $(b, X)$ & $(\varepsilon, Z_0)$ & $(\varepsilon, X)$\\
    \hline
    \hline
    $\rightarrow q_0$& $(q_0, X Z_0)$ & $(q_0, XX)$ & $(q_1, Z_0)$ & $(q_1, \varepsilon)$ & $(q_f, \varepsilon)$ & $-$\\
    $q_1$            & $-$            & $-$         & $(q_1, Z_0)$ & $(q_1, \varepsilon)$ & $(q_f, \varepsilon)$ & $-$\\
    $\leftarrow q_2$ & $-$& $-$& $-$ & $-$ & $-$ & $-$\\
    \hline
\end{tabular}
\]

\begin{enumerate}[a), noitemsep]
    \item Nakreslete stavový diagram zásobníkového automatu $A$.
    \item Ukažte práci zásobníkového automatu nad slovem $aabba$ a slovem $abbbb$.
    \item Charakterizujte jazyk $L$, který tento zásobníkový automat přijímá. Tvrzení zdůvodněte.
\end{enumerate}