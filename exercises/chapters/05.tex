\section{Páté cvičení}

\subsection{Příklad}
Dokažte, že pro libovolné jazyky $L_1$ a $L_2$ nad stejnou abecedou platí $(L_1 \cup L_2)^\star = (L_1^\star L_2^\star)^\star$.

\subsection{Příklad}
Napište regulární výraz, který reprezentuje jazyk $L$ nad abecedou $\Sigma = \{0,1\}$, jestliže výraz existuje.
\begin{enumerate}[a), noitemsep]
    \item $L$ se skládá ze všech slov, které obsahují pouze $0$.
    \item $L$ se skládá ze všech slov, které obsahují přesně jednu $1$.
    \item $L$ se skládá ze všech slov, které obsahují alespoň jednu $1$.
    \item $L$ se skládá ze všech slov, které obsahují alespoň dvě $1$.
    \item $L$ se skládá ze všech slov, které obsahují sudý počet $1$.
    \item $L$ se skládá ze všech slov, které obsahují lichý počet $1$.
\end{enumerate}
Odpovědi zdůvodněte.

\subsection{Příklad}
Jazyk $L_1$ je reprezentován regulárním výrazem $r_1 = 0^\star1^\star0^\star1^\star0^\star$ a jazyk $L_2$ je reprezentován 
regulárním výrazem $r_2 = (01 + 10)^\star$.
\begin{enumerate}[a), noitemsep]
    \item Najděte nejkratší neprázdné slovo, které patří do průniku $L_1 \cap L_2$.
    \item Najděte nejdělší neprázdné slovo, které patří do průniku $L_1 \cap L_2$.
    \item Najděte nejkratší slovo, které leží v $L_1$, ale neleží v $L_2$.
    \item Najděte nejkratší slovo, které neleží ve sjednocení $L_1 \cup L_2$.
\end{enumerate}
Odpovědi zdůvodněte.

\subsection{Příklad}
Je dán regulární výraz $r=(baa + bab) (ab)$. K $r$ zkonstruujte redukovaný DFA, který přijímá jazyk reprezentovaný tímto 
regulárním výrazem.\\
(Návod: Postupujte dvěma způsoby; jednak obecným postupem z přednášky, jednak rozdělením na podvýrazy, pro které je možné 
najít NFA přímo a pak použitím konstrukcí z důkazů faktu, že třída regulárních jazyků je uzavřena na sjednocení, zřetězení 
a Kleeneho operátor.)

\subsection{Příklad}
Je dán jazyk $L$ nad $\Sigma = \{0,1\}$, kde $L = \{w \mid w \text{ neobsahuje } 11 \text{ jako podslovo}\}$. Navrhněte 
redukovaný DFA $M$, který přijímá $L$. Pro jazyk $L$ najděte regulární výraz, který ho reprezentuje (použijte úpravy 
grafu z přednášky).