\section{Páté cvičení}

\subsection{Příklad}
Dokažte, že pro libovolné jazyky $L_1$ a $L_2$ nad stejnou abecedou platí $(L_1 \cup L_2)^\star = (L_1^\star L_2^\star)^\star$.

\subsection{Příklad}
Napište regulární výraz, který reprezentuje jazyk $L$ nad abecedou $\Sigma = \{0,1\}$, jestliže výraz existuje.
\begin{enumerate}[a), noitemsep]
    \item $L$ se skládá ze všech slov, které obsahují pouze $0$.
    \item $L$ se skládá ze všech slov, které obsahují přesně jednu $1$.
    \item $L$ se skládá ze všech slov, které obsahují alespoň jednu $1$.
    \item $L$ se skládá ze všech slov, které obsahují alespoň dvě $1$.
    \item $L$ se skládá ze všech slov, které obsahují sudý počet $1$.
    \item $L$ se skládá ze všech slov, které obsahují lichý počet $1$.
\end{enumerate}
Odpovědi zdůvodněte.

\begin{multicols}{2}
    a) $0^\star$.\\
    b) $0^\star 1 0^\star$.\\
    c) $0^\star 1 (0+1)^\star$.
\columnbreak

    d) $0^\star 1 0^\star 1 (0+1)^\star$.\\
    e) $(0^\star 1 0^\star 1)^\star 0^\star$.\\
    f) $0^\star 1 (0^\star 1 0^\star)^\star 0^\star$.
\end{multicols}
%TODO pridat zduvodneni - dukazy

\subsection{Příklad}
Jazyk $L_1$ je reprezentován regulárním výrazem $r_1 = 0^\star1^\star0^\star1^\star0^\star$ a jazyk $L_2$ je reprezentován 
regulárním výrazem $r_2 = (01 + 10)^\star$.
\begin{enumerate}[a), noitemsep]
    \item Najděte nejkratší neprázdné slovo, které patří do průniku $L_1 \cap L_2$.
    \item Najděte nejdělší neprázdné slovo, které patří do průniku $L_1 \cap L_2$.
    \item Najděte nejkratší slovo, které leží v $L_1$, ale neleží v $L_2$.
    \item Najděte nejkratší slovo, které neleží ve sjednocení $L_1 \cup L_2$.
\end{enumerate}
Odpovědi zdůvodněte.

\begin{multicols}{2}
    a) $01$ nebo $10$.\\
    b) $01100110$.
\columnbreak

    c) $0$ nebo $1$, protože délka $\not\in L_2$.\\
    d) $10101$.
\end{multicols}
%TODO pridat zduvodneni - dukazy

\subsection{Příklad}
Je dán regulární výraz $r=(baa + bab)^\star (ab)^\star$. K $r$ zkonstruujte redukovaný DFA, který přijímá jazyk reprezentovaný tímto 
regulárním výrazem.\\
(Návod: Postupujte dvěma způsoby; jednak obecným postupem z přednášky, jednak rozdělením na podvýrazy, pro které je možné 
najít NFA přímo a pak použitím konstrukcí z důkazů faktu, že třída regulárních jazyků je uzavřena na sjednocení, zřetězení 
a Kleeneho operátor.)

\textbf{1. Obecný postup}

\begin{multicols}{2}

    $(baa + bab) \equiv ba(a+b)$

    \begin{tikzpicture}
        \node[state, initial, accepting] (0) {$0$};
        \node[state, above of=0] (1) {$1$};
        \node[state, right of=0] (2) {$2$};
        \node[state, initial, accepting, right of=2] (3) {$3$};
        \node[state, right of=3] (4) {$4$};

        \draw
            (0) edge[bend left, left] node{$b$} (1)
            (0) edge[bend right, below] node{$\varepsilon$} (3)
            (1) edge[bend left, right] node{$b$} (2)
            (2) edge[bend right, above] node{$a,b$} (0)
            (3) edge[bend left, above] node{$a$} (4)
            (4) edge[bend left, below] node{$b$} (3)
            ;
    \end{tikzpicture}

\columnbreak

    \begin{tabular}{|r|c c c|c|}
        \hline
        & $\varepsilon$ & $a$ & $b$ & $\varepsilon$ uzávěry\\
        \hline
        \hline
        $1$ & $4$ & $\emptyset$ & $2$         & $\varepsilon$-uz(1) = $\{1,4\}$\\
        $2$ & $\emptyset$ & $3$ & $\emptyset$ & $\varepsilon$-uz(2) = $\{2\}$\\
        $3$ & $\emptyset$ & $1$ & $1$         & $\varepsilon$-uz(3) = $\{3\}$\\
        $4$ & $\emptyset$ & $5$ & $\emptyset$ & $\varepsilon$-uz(4) = $\{4\}$\\
        $5$ & $\emptyset$ & $\emptyset$ & $4$ & $\varepsilon$-uz(5) = $\{5\}$\\
        \hline
    \end{tabular}
\end{multicols}

Podmnožinová konstrukce:

\[
\begin{tabular}{|r r|c c|c|c c|c|c c|c|c c|c|}
    \hline
    & & $a$ & $b$ & $\sim_0$ & $a$ & $b$ & $\sim_1$ & $a$ & $b$ & $\sim_2$ & $a$ & $b$ & $\sim_3$\\
    \hline
    \hline
    $\leftrightarrow$& $\{1,4\}$ &$5$         &$2$         &$K$ &$O$ &$O$ &$K$ &$A$ &$O$ &$K$ &$A$ &$C$ &$D$\\
    $$&             $5$          &$\emptyset$ &$4$         &$O$ &$O$ &$K$ &$A$ &$O$ &$K$ &$A$ &$O$ &$K$ &$A$\\
    $$&             $2$          &$3$         &$\emptyset$ &$O$ &$O$ &$O$ &$O$ &$B$ &$O$ &$C$ &$B$ &$O$ &$C$\\
    $\leftarrow $&  $4$          &$5$         &$\emptyset$ &$K$ &$O$ &$O$ &$K$ &$A$ &$O$ &$K$ &$A$ &$O$ &$K$\\
    $$&             $3$          &$\{1,4\}$   &$\{1,4\}$   &$O$ &$K$ &$K$ &$B$ &$K$ &$K$ &$B$ &$K$ &$K$ &$B$\\
    $$&             $\emptyset$  &$\emptyset$ &$\emptyset$ &$O$ &$O$ &$O$ &$O$ &$O$ &$O$ &$O$ &$O$ &$O$ &$O$\\
    \hline
\end{tabular}
\]

\begin{tikzpicture}
    \node[state, initial] (d) {$D$};
    \node[state, right of=d] (a) {$A$};
    \node[state, below of=d] (c) {$C$};
    \node[state, accepting, right of=a] (k) {$K$};
    \node[state, below of=a] (o) {$O$};
    \node[state, below of=k] (b) {$B$};

    \draw
        (d) edge[bend left, above] node{$a$} (a)
        (d) edge[bend right, left] node{$b$} (c)
        (a) edge[bend right, left] node{$a$} (o)
        (a) edge[bend left, above] node{$b$} (k)
        (c) edge[bend left, above] node{$b$} (o)
        (c) edge[bend right, below] node{$a$} (b)
        (k) edge[bend left, below] node{$a$} (a)
        (k) edge[bend left, below] node{$b$} (o)
        (b) edge[bend right, right] node{$a,b$} (k)
        (o) edge[loop above] node{$a,b$} (o)

        ;
\end{tikzpicture}


\textbf{2. Rozdělení na podvýrazy} 

I. krok očíslování 

$(b_1 a_2 a_3 + b_4 a_5 b_6)^\star (a_7 b_8)^\star \equiv (b_1 a_2 (a_3 + b_4))^\star (a_5 b_6)^\star$

II. krok

Pro jazyk, který je přijímaný regulárním výrazem $r$ platí:

výraz může začínat: $b_1, a_5$ \\
mohou po sobě následovat: $\bm{b_1} : a_2$; $\bm{a_2} : a_3, b_4$; $\bm{a_3} : b_1, a_5$; $\bm{b4} : b_1, a_5$; $\bm{a_5} : b_6$; $\bm{b_6} : a_5$\\
výraz může končit: $a_3, b_4, b_6$\\
je $\varepsilon$ v $L$? Ano.

III. krok

\begin{tikzpicture}
    \node[state, initial] (s) {$S$};
    \node[state, right of=s] (1) {$b_1$};
    \node[state, below of=1] (5) {$a_5$};
    \node[state, accepting, left of=5] (6) {$b_6$};
    \node[state, right of=1] (2) {$a_2$};
    \node[state, accepting, right of=2] (3) {$a_3$};
    \node[state, accepting, below of=2] (4) {$b_4$};

    \draw
        (s) edge[bend left, above] node{$b$} (1)
        (s) edge[bend left, left] node{$a$} (5)
        (1) edge[bend right, above] node{$a$} (2)
        (2) edge[bend right, above] node{$a$} (3)
        (2) edge[] node[right]{$b$} (4)
        (3) edge[bend left=90, below] node{$a$} (5)
        (3) edge[bend right, above] node{$b$} (1)
        (4) edge[] node{$a$} (5)
        (4) edge[bend left,above] node{$b$} (1)
        (5) edge[bend left, below] node{$b$} (6)
        (6) edge[bend left, above] node{$a$} (5)
        ;
\end{tikzpicture}

IV. podmnožinová konstrukce DFA + redukce

\subsection{Příklad}
Je dán jazyk $L$ nad $\Sigma = \{0,1\}$, kde $L = \{w \mid w \text{ neobsahuje } 11 \text{ jako podslovo}\}$. Navrhněte 
redukovaný DFA $M$, který přijímá $L$. Pro jazyk $L$ najděte regulární výraz, který ho reprezentuje (použijte úpravy 
grafu z přednášky).

$\mathcal{L} = \{w \mid w \textbf{ obsahuje } 11 \text{ jako podslovo}\}\text{.}$

$\mathcal{L}$ NFA:
\begin{tikzpicture}
    \node[state, initial] (0) {$q_0$};
    \node[state, right of=0] (1) {$q_1$};
    \node[state, right of=1] (2) {$q_2$};
    \node[state, accepting, right of=2] (3) {$q_3$};

    \draw
        (0) edge[loop above] node{$0, 1$} (0)
        (0) edge[bend left, above] node{$1$} (1)
        (1) edge[bend left, above] node{$1$} (2)
        (2) edge[bend left, above] node{$0,1$} (3)
       ;
\end{tikzpicture}

$\mathcal{L}$ DFA:
\begin{tikzpicture}
    \node[state, initial] (0) {$q_0$};
    \node[state, right of=0] (1) {$q_1$};
    \node[state, accepting, right of=1] (2) {$q_2$};

    \draw
        (0) edge[loop above] node{$0$} (0)
        (0) edge[bend left, above] node{$1$} (1)
        (1) edge[bend left, below] node{$0$} (0)
        (1) edge[bend left, above] node{$1$} (2)
        (2) edge[loop above] node{$0,1$} (2)
       ;
\end{tikzpicture}

L DFA:
\begin{tikzpicture}
    \node[state, initial, accepting] (0) {$q_0$};
    \node[state, accepting, right of=0] (1) {$q_1$};
    \node[state, right of=1] (2) {chyba};

    \draw
        (0) edge[loop above] node{$0$} (0)
        (0) edge[bend left, above] node{$1$} (1)
        (1) edge[bend left, below] node{$0$} (0)
        (1) edge[bend left, above] node{$1$} (2)
        (2) edge[loop above] node{$0,1$} (2)
       ;
\end{tikzpicture}