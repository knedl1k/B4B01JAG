\section{Páté cvičení}

\subsection{Důkaz obecné vlastnosti dvou libovolných jazyků}
Dokažte, že pro libovolné jazyky $L_1$ a $L_2$ nad stejnou abecedou platí $(L_1 \cup L_2)^\star = (L_1^\star L_2^\star)^\star$.

\textbf{a)} $(L_1 \cup L_2)^\star \subseteq (L_1^\star L_2^\star)^\star$

Nechť $w \in (L_1 \cup L_2)^\star$. Tedy $w = x_1 x_2 \dots x_k, x_i \in (L_1 \cup L_2)$.

Protože každý $x_i$ patří buď do $L_1$ nebo do $L_2$, můžeme každé takové $w$ chápat jako zřetězení slov, která střídavě
patří do $L_1^\star$ nebo $L_2^\star$. Tedy slova z $(L_1 \cup L_2)$ lze seskupit do bloků tak, že tyto bloky patří buď do
$L_1^\star$ nebo $L_2^\star$.

Například: $w = u_1 v_1 u_2 v_2 \dots u_m v_m \text{, kde } u_i \in L_1^\star, v_i \in L_2^\star$.

To znamená, že $w \in (L_1^\star \cup L_2^\star)$. Jelikož $w$ je libovolné slovo z $(L_1^\star L_2^\star)^\star$ platí
$(L_1 \cup L_2)^\star \subseteq (L_1^\star L_2^\star)^\star$.

\textbf{b)} $(L_1^\star L_2^\star)^\star \subseteq (L_1 \cup L_2)^\star$

Nechť $w \in (L_1^\star L_2^\star)^\star$. Tedy $w = y_1 y_2 \dots y_k, y_i \in L_1^\star L_2^\star$.

Každé $y_i \in L_1^\star L_2^\star$ lze dále rozepsat jako zřetězení slov $u_1 v_1 \dots u_m v_m \text{, kde } u_j \in 
L_1^\star, v_j \in L_2^\star$.

Tedy každé slovo $y_i$ je posloupnost slov z $L_1$ a $L_2$, což znamená, že všechny $y_i$ jsou tvořeny prvky z $L_1 \cup L_2$.
Proto celé $w$ lze vyjádřit jako zřetězení slov z $L_1 \cup L_2$, což znamená, že $w \in (L_1 \cup L_2)^\star$. A tedy
platí $(L_1^\star L_2^\star)^\star \subseteq (L_1 \cup L_2)^\star$.

\subsection{Tvorba regulárního výrazu reprezentujícího jazyk}
Napište regulární výraz, který reprezentuje jazyk $L$ nad abecedou $\Sigma = \{0,1\}$, jestliže výraz existuje.
\begin{enumerate}[a), noitemsep]
    \item $L$ se skládá ze všech slov, které obsahují pouze $0$.
    \item $L$ se skládá ze všech slov, které obsahují přesně jednu $1$.
    \item $L$ se skládá ze všech slov, které obsahují alespoň jednu $1$.
    \item $L$ se skládá ze všech slov, které obsahují alespoň dvě $1$.
    \item $L$ se skládá ze všech slov, které obsahují sudý počet $1$.
    \item $L$ se skládá ze všech slov, které obsahují lichý počet $1$.
\end{enumerate}
Odpovědi zdůvodněte.

\begin{multicols}{2}
    a) $0^\star 0 0^\star$.\\
    b) $0^\star 1 0^\star$.\\
    c) $0^\star 1 (0+1)^\star$.
\columnbreak

    d) $0^\star 1 0^\star 1 (0+1)^\star$.\\
    e) $(0^\star 1 0^\star 1)^\star 0^\star$.\\
    f) $0^\star 1 (0^\star 1 0^\star 1)^\star 0^\star$.
\end{multicols}
Zdůvodnění:
\begin{enumerate}[a), noitemsep]
    \item Vynutíme si alespoň jednu nulu, protože prázdné slovo neobsahuje $0$. Zároveň umožníme pomocí Kleenyho 
    operátoru libovolný počet $0$ na obou stranách slova. Žádnou $1$ nepovolíme.
    \item Před první $1$ povolíme libovolný počet $0$. Po první $1$ následně povolíme pouze libovolný počet $0$.
    \item Před první $1$ povolíme libovolný počet $0$. Po první $1$ následně povolíme všechny znaky abecedy 
    $\Sigma$ v libovolném počtu.
    \item Dovolíme libovolný počet $0$, po kterých následuje právě jedna $1$. Následovat může následovat libovolný počet 
    $0$. Po té zas vynutíme právě jednu $1$. Po této sekvenci může následovat libovolný počet znaků abecedy $\Sigma$.
    \item Zde mohou nastat dva případy: slovo neobsahuje žádnou $1$, nebo jich obsahuje sudý počet. Pokud neobsahuje 
    žádnou $1$, můžeme celou závorku přeskočit díky Kleenyho operátoru, pak přijmeme pouze libovolný počet $0$. Pokud 
    obsahuje jedničku, závorka nám zaručí, že jich bude nutně sudý počet. Mezi nimi na všech stranách může být libovolný
    počet $0$.
    \item Zde vynutíme alespoň jednu $1$ a pokud by se mělo ve slově vyskytovat více $1$, tak závorka zaručí, že vždy
    přijmeme dvojici $1$. Tím bude v celém slově stále lichý počet již zpracovaných $1$. Znovu z obou stran obklopeny 
    libovolným počtem $0$.
\end{enumerate}

\subsection{Hledání slov splňující různé regulární výrazy}
Jazyk $L_1$ je reprezentován regulárním výrazem $r_1 = 0^\star1^\star0^\star1^\star0^\star$ a jazyk $L_2$ je reprezentován
regulárním výrazem $r_2 = (01 + 10)^\star$.
\begin{enumerate}[a), noitemsep]
    \item Najděte nejkratší neprázdné slovo, které patří do průniku $L_1 \cap L_2$.
    \item Najděte nejdelší neprázdné slovo, které patří do průniku $L_1 \cap L_2$.
    \item Najděte nejkratší slovo, které leží v $L_1$, ale neleží v $L_2$.
    \item Najděte nejkratší slovo, které neleží ve sjednocení $L_1 \cup L_2$.
\end{enumerate}
Odpovědi zdůvodněte.

\begin{multicols}{2}
    a) $01$ nebo $10$.\\
    b) $01100110$.
\columnbreak

    c) $0$ nebo $1$, protože délka $\not\in L_2$.\\
    d) $10101$.
\end{multicols}
Zdůvodnění:
\begin{enumerate}[a), noitemsep]
    \item Silnější restrikci do průniku vnáší $L_2$. $L_2$ vynucuje vybrání si jedné z dvojic $10$ nebo $01$. $L_1$ nám 
    obě varianty povoluje.
    \item Teď naopak využijeme flexibility $L_1$. Jedinou povinností je používat vždy dvojice, kvůli $L_2$. Pro 
    visualizaci: $0 1^2 0^2 1^2 0$. To má přesně tvar jako $r_1$.
    \item Protože $L_2$ vyžaduje dvojice, zaměříme se na slova s menší délkou: $0$, $1$. Obě varianty $L_1$ přijme.
    \item Musíme porušit dvojice z $L_2$ a zároveň pravidelné střídání $0$ a $1$ z $L_1$. Stačí tedy prohodit pořadí 
    střídání a aby slovo mělo lichou délku. Také je potřeba, aby délka byla alespoň $|r_1|$, kvůli $L_1$. 
\end{enumerate}

\subsection{Konstrukce redukovaného DFA k regulárnímu výrazu}
Je dán regulární výraz $r=(baa + bab)^\star (ab)^\star$. K $r$ zkonstruujte redukovaný DFA, který přijímá jazyk 
reprezentovaný tímto regulárním výrazem.\\
(Návod: Postupujte dvěma způsoby; jednak obecným postupem z přednášky, jednak rozdělením na podvýrazy, pro které je možné
najít NFA přímo a pak použitím konstrukcí z důkazů faktu, že třída regulárních jazyků je uzavřena na sjednocení, zřetězení
a Kleeneho operátor.)

\newpage
\textbf{1. Obecný postup}

\begin{multicols}{2}
    $(baa + bab)^\star \equiv (ba(a+b))^\star$

    \begin{tikzpicture}
        \node[state, initial, accepting] (0) {$0$};
        \node[state, above of=0] (1) {$1$};
        \node[state, right of=0] (2) {$2$};
        \node[state, initial, accepting, right of=2] (3) {$3$};
        \node[state, right of=3] (4) {$4$};

        \draw
            (0) edge[bend left, left] node{$b$} (1)
            (0) edge[bend right, below] node{$\varepsilon$} (3)
            (1) edge[bend left, right] node{$b$} (2)
            (2) edge[bend right, above] node{$a,b$} (0)
            (3) edge[bend left, above] node{$a$} (4)
            (4) edge[bend left, below] node{$b$} (3)
            ;
    \end{tikzpicture}

\columnbreak

    \begin{tabular}{|r|c c c|c|}
        \hline
        & $\varepsilon$ & $a$ & $b$ & $\varepsilon$ uzávěry\\
        \hline
        \hline
        $1$ & $4$ & $\emptyset$ & $2$         & $\varepsilon$-uz(1) = $\{1,4\}$\\
        $2$ & $\emptyset$ & $3$ & $\emptyset$ & $\varepsilon$-uz(2) = $\{2\}$\\
        $3$ & $\emptyset$ & $1$ & $1$         & $\varepsilon$-uz(3) = $\{3\}$\\
        $4$ & $\emptyset$ & $5$ & $\emptyset$ & $\varepsilon$-uz(4) = $\{4\}$\\
        $5$ & $\emptyset$ & $\emptyset$ & $4$ & $\varepsilon$-uz(5) = $\{5\}$\\
        \hline
    \end{tabular}
\end{multicols}
Podmnožinová konstrukce:
\[
\begin{tabular}{|r r|c c|c|c c|c|c c|c|c c|c|}
    \hline
    & & $a$ & $b$ & $\sim_0$ & $a$ & $b$ & $\sim_1$ & $a$ & $b$ & $\sim_2$ & $a$ & $b$ & $\sim_3$\\
    \hline
    \hline
    $\leftrightarrow$& $\{1,4\}$ &$5$         &$2$         &$K$ &$O$ &$O$ &$K$ &$A$ &$O$ &$K$ &$A$ &$C$ &$D$\\
    $$&             $5$          &$\emptyset$ &$4$         &$O$ &$O$ &$K$ &$A$ &$O$ &$K$ &$A$ &$O$ &$K$ &$A$\\
    $$&             $2$          &$3$         &$\emptyset$ &$O$ &$O$ &$O$ &$O$ &$B$ &$O$ &$C$ &$B$ &$O$ &$C$\\
    $\leftarrow $&  $4$          &$5$         &$\emptyset$ &$K$ &$O$ &$O$ &$K$ &$A$ &$O$ &$K$ &$A$ &$O$ &$K$\\
    $$&             $3$          &$\{1,4\}$   &$\{1,4\}$   &$O$ &$K$ &$K$ &$B$ &$K$ &$K$ &$B$ &$K$ &$K$ &$B$\\
    $$&             $\emptyset$  &$\emptyset$ &$\emptyset$ &$O$ &$O$ &$O$ &$O$ &$O$ &$O$ &$O$ &$O$ &$O$ &$O$\\
    \hline
\end{tabular}
\]
\begin{tikzpicture}
    \node[state, initial] (d) {$D$};
    \node[state, right of=d] (a) {$A$};
    \node[state, below of=d] (c) {$C$};
    \node[state, accepting, right of=a] (k) {$K$};
    \node[state, below of=a] (o) {$O$};
    \node[state, below of=k] (b) {$B$};

    \draw
        (d) edge[bend left, above] node{$a$} (a)
        (d) edge[bend right, left] node{$b$} (c)
        (a) edge[bend right, left] node{$a$} (o)
        (a) edge[bend left, above] node{$b$} (k)
        (c) edge[bend left, above] node{$b$} (o)
        (c) edge[bend right, below] node{$a$} (b)
        (k) edge[bend left, below] node{$a$} (a)
        (k) edge[bend left, below] node{$b$} (o)
        (b) edge[bend right, right] node{$a,b$} (k)
        (o) edge[loop above] node{$a,b$} (o)

        ;
\end{tikzpicture}

\textbf{2. postup Rozdělením na podvýrazy}\\
I. krok očíslování\\
$(b_1 a_2 a_3 + b_4 a_5 b_6)^\star (a_7 b_8)^\star \equiv (b_1 a_2 (a_3 + b_4))^\star (a_5 b_6)^\star$\\
II. krok\\
Pro jazyk, který je přijímaný regulárním výrazem $r$ platí:\\
výraz může začínat: $b_1, a_5$ \\
mohou po sobě následovat: $\bm{b_1} \text{: } a_2$; $\bm{a_2} \text{: } a_3, b_4$; $\bm{a_3} \text{: } b_1, a_5$; 
$\bm{b_4} \text{: } b_1, a_5$; $\bm{a_5} \text{: } b_6$; $\bm{b_6} \text{: } a_5$\\
výraz může končit: $a_3, b_4, b_6$\\
je $\varepsilon$ v $L$? Ano.

\newpage

III. krok\\
\begin{tikzpicture}
    \node[state, initial] (s) {$S$};
    \node[state, right of=s] (1) {$b_1$};
    \node[state, below of=1] (5) {$a_5$};
    \node[state, accepting, left of=5] (6) {$b_6$};
    \node[state, right of=1] (2) {$a_2$};
    \node[state, accepting, right of=2] (3) {$a_3$};
    \node[state, accepting, below of=2] (4) {$b_4$};

    \draw
        (s) edge[bend left, above] node{$b$} (1)
        (s) edge[bend left, left] node{$a$} (5)
        (1) edge[bend right, above] node{$a$} (2)
        (2) edge[bend right, above] node{$a$} (3)
        (2) edge[] node[right]{$b$} (4)
        (3) edge[bend left=90, below] node{$a$} (5)
        (3) edge[bend right, above] node{$b$} (1)
        (4) edge[] node{$a$} (5)
        (4) edge[bend left,above] node{$b$} (1)
        (5) edge[bend left, below] node{$b$} (6)
        (6) edge[bend left, above] node{$a$} (5)
        ;
\end{tikzpicture}

IV. podmnožinová konstrukce DFA + redukce

\begin{center}
    \begin{tabular}{|r r|c c|c|c c|c|c c|c|c c|c|}
        \hline
        & & $a$ & $b$ & $\sim_0$ & $a$ & $b$ & $\sim_1$ & $a$ & $b$ & $\sim_2$ & $a$ & $b$ & $\sim_3$\\
        \hline
        \hline
        $\rightarrow$& $S$        &$5$        &$1$        &$O$ &$O$ &$O$ &$O$ &$A$ &$O$ &$C$ &$A$ &$D$ &$C$ \\
        $$&            $5$        &$\emptyset$&$6$        &$O$ &$O$ &$K$ &$A$ &$O$ &$K$ &$A$ &$O$ &$E$ &$A$ \\
        $$&            $1$        &$2$        &$\emptyset$&$O$ &$O$ &$O$ &$O$ &$B$ &$O$ &$D$ &$B$ &$O$ &$D$ \\
        $$&            $\emptyset$&$\emptyset$&$\emptyset$&$O$ &$O$ &$O$ &$O$ &$O$ &$O$ &$O$ &$O$ &$O$ &$O$ \\
        $\leftarrow$&  $6$        &$5$        &$\emptyset$&$K$ &$O$ &$O$ &$K$ &$A$ &$O$ &$E$ &$A$ &$O$ &$E$ \\
        $$&            $2$        &$3$        &$4$        &$O$ &$K$ &$K$ &$B$ &$K$ &$K$ &$B$ &$K$ &$K$ &$B$ \\
        $\leftarrow$&  $3$        &$5$        &$1$        &$K$ &$O$ &$O$ &$K$ &$A$ &$O$ &$K$ &$A$ &$D$ &$K$ \\
        $\leftarrow$&  $4$        &$5$        &$1$        &$K$ &$O$ &$O$ &$K$ &$A$ &$O$ &$K$ &$A$ &$D$ &$K$ \\
        \hline
    \end{tabular}

    \begin{tikzpicture}
        \node[state, initial] (c) {$C$};
        \node[state, right of=c] (a) {$A$};
        \node[state, below of=c] (d) {$D$};
        \node[state, below of=a] (o) {$O$};
        \node[state, right of=o] (e) {$E$};
        \node[state, below of=d] (b) {$B$};
        \node[state, accepting, left of=c] (k) {$K$};
    
        \draw
            (c) edge[bend left, above] node{$a$} (a)
            (c) edge[bend right, right] node{$b$} (d)
            (a) edge[] node[left]{$a$} (o)
            (a) edge[bend left, right] node{$b$} (e)
            (d) edge[] node[right]{$a$} (b)
            (d) edge[] node[below]{$b$} (o)
            (o) edge[loop below] node{$a,b$} (o)
            (e) edge[bend left, above] node{$a$} (a)
            (e) edge[] node[below]{$b$} (o)
            (b) edge[bend left] node[right]{$a,b$} (k)
            (k) edge[bend left=60, below] node{$a$} (a)
            (k) edge[bend right] node{$b$} (d)
            ;
    \end{tikzpicture}
\end{center}

\newpage
\subsection{Tvorba DFA a regulárního výrazu dle jazyka}
Je dán jazyk $L$ nad $\Sigma = \{0,1\}$, kde $L = \{w \mid w \text{ neobsahuje } 11 \text{ jako podslovo}\}$. Navrhněte
redukovaný DFA $M$, který přijímá $L$. Pro jazyk $L$ najděte regulární výraz, který ho reprezentuje (použijte úpravy
grafu z přednášky).

$\mathcal{L} = \{w \mid w \textbf{ obsahuje } 11 \text{ jako podslovo}\}\text{.}$

\begin{multicols}{2}
    $\mathcal{L}$ NFA:

    \begin{tikzpicture}
        \node[state, initial] (0) {$q_0$};
        \node[state, right of=0] (1) {$q_1$};
        \node[state, accepting, right of=1] (2) {$q_2$};
        % \node[state, accepting, right of=2] (3) {$q_3$};

        \draw
            (0) edge[loop above] node{$0, 1$} (0)
            (0) edge[bend left, above] node{$1$} (1)
            (1) edge[bend left, above] node{$1$} (2)
            (2) edge[loop above] node{$0,1$} (2)
        ;
    \end{tikzpicture}
\columnbreak

    $\mathcal{L}$ DFA:

    \begin{tikzpicture}
        \node[state, initial] (0) {$q_0$};
        \node[state, right of=0] (1) {$q_1$};
        \node[state, accepting, right of=1] (2) {$q_2$};

        \draw
            (0) edge[loop above] node{$0$} (0)
            (0) edge[bend left, above] node{$1$} (1)
            (1) edge[bend left, below] node{$0$} (0)
            (1) edge[bend left, above] node{$1$} (2)
            (2) edge[loop above] node{$0,1$} (2)
        ;
    \end{tikzpicture}
\end{multicols}

\begin{multicols}{2}
    L DFA:
    
    \begin{tikzpicture}
        \node[state, initial, accepting] (0) {$q_0$};
        \node[state, accepting, right of=0] (1) {$q_1$};
        \node[state, right of=1] (2) {chyba};

        \draw
            (0) edge[loop above] node{$0$} (0)
            (0) edge[bend left, above] node{$1$} (1)
            (1) edge[bend left, below] node{$0$} (0)
            (1) edge[bend left, above] node{$1$} (2)
            (2) edge[loop above] node{$0,1$} (2)
        ;
    \end{tikzpicture}

    1. zavedu stavy $S$, $F$:

    \begin{tikzpicture}[shorten >=1pt,node distance=20mm,on grid,auto]
        \tikzstyle{every state}=[fill={rgb:black,1;white,10}, minimum size=18pt, inner sep=1pt]

        \node[state, initial] (s) {$S$};
        \node[state, right of=s] (0) {$q_0$};
        \node[state, right of=0] (1) {$q_1$};
        \node[state, accepting, below of=0] (f) {$F$};
        \node[state, below of=1] (c) {chyba};

        \path[->]
        (s) edge [bend left] node {$\varepsilon$} (0)
        (0) edge[loop above] node {$0$} (0)
        (0) edge[bend left] node{$1$} (1)
        (0) edge[bend right] node{$\varepsilon$} (f)
        (1) edge[bend left] node{$0$} (0)
        (1) edge[bend left] node{$1$} (c)
        (1) edge[bend left, above] node{$\varepsilon$} (f)
        (c) edge[loop right] node{$0,1$} (c)
        ;
    \end{tikzpicture}

    2. odstraňuji smyčky:

    \begin{tikzpicture}[shorten >=1pt,node distance=20mm,on grid,auto]
        \tikzstyle{every state}=[fill={rgb:black,1;white,10}, minimum size=18pt, inner sep=1pt]

        \node[state, initial] (s) {$S$};
        \node[state, right of=s] (0) {$q_0$};
        \node[state, right of=0] (1) {$q_1$};
        \node[state, accepting, below of=0] (f) {$F$};
        \node[state, below of=1] (c) {chyba};

        \path[->]
        (s) edge [bend left] node {$\varepsilon$} (0)
        (0) edge[bend left] node{$0^\star 1$} (1)
        (0) edge[bend right] node{$\varepsilon$} (f)
        (1) edge[bend left] node{$0$} (0)
        (1) edge[bend left] node{$1$} (c)
        (1) edge[bend left, above] node{$\varepsilon$} (f)
        ;
    \end{tikzpicture}

\columnbreak

    3. odstraňuji vrchol \textit{chyba}:

    \begin{tikzpicture}[shorten >=1pt,node distance=20mm,on grid,auto]
        \tikzstyle{every state}=[fill={rgb:black,1;white,10}, minimum size=18pt, inner sep=1pt]

        \node[state, initial] (s) {$S$};
        \node[state, right of=s] (0) {$q_0$};
        \node[state, right of=0] (1) {$q_1$};
        \node[state, accepting, below of=0] (f) {$F$};

        \path[->]
        (s) edge [bend left] node {$\varepsilon$} (0)
        (0) edge[bend left] node{$0^\star 1$} (1)
        (0) edge[bend right] node{$\varepsilon$} (f)
        (1) edge[bend left] node{$0$} (0)
        (1) edge[bend left, above] node{$\varepsilon$} (f)
        ;
    \end{tikzpicture}

    4. odstraňuji vrchol $q_1$:

    \begin{tikzpicture}[shorten >=1pt,node distance=20mm,on grid,auto]
        \tikzstyle{every state}=[fill={rgb:black,1;white,10}, minimum size=18pt, inner sep=1pt]

        \node[state, initial] (s) {$S$};
        \node[state, right of=s] (0) {$q_0$};
        \node[state, accepting, right of=0] (f) {$F$};

        \path[->]
        (s) edge [bend left] node {$\varepsilon$} (0)
        (0) edge[bend left] node{$0^\star 1$} (f)
        (0) edge[bend right] node{$\varepsilon$} (f)
        (0) edge[loop below] node{$0^\star 10$} (0)
        ;
    \end{tikzpicture}

    5. odstraňuji smyčky:

    \begin{tikzpicture}[shorten >=1pt,node distance=20mm,on grid,auto]
        \tikzstyle{every state}=[fill={rgb:black,1;white,10}, minimum size=18pt, inner sep=1pt]

        \node[state, initial] (s) {$S$};
        \node[state, right of=s] (0) {$q_0$};
        \node[state, accepting, right of=0] (f) {$F$};

        \path[->]
        (s) edge [bend left] node {$\varepsilon$} (0)
        (0) edge[bend left] node{$0^\star 1 0 0^\star 1$} (f)
        (0) edge[bend right] node{$0^\star 10$} (f)
        ;
    \end{tikzpicture}

    6. odstraňuji paralelní hrany:

    \begin{tikzpicture}[shorten >=1pt,node distance=20mm,on grid,auto]
        \tikzstyle{every state}=[fill={rgb:black,1;white,10}, minimum size=18pt, inner sep=1pt]

        \node[state, initial] (s) {$S$};
        \node[state, right of=s] (0) {$q_0$};
        \node[state, accepting, right of=0] (f) {$F$};

        \path[->]
        (s) edge [bend left] node {$\varepsilon$} (0)
        (0) edge[bend left] node{$0^\star 1 0 0^\star 1 + 0^\star 10$} (f)
        ;
    \end{tikzpicture}

    7. odstraňuji vrchol $A$:

    \begin{tikzpicture}[shorten >=1pt,node distance=20mm,on grid,auto]
        \tikzstyle{every state}=[fill={rgb:black,1;white,10}, minimum size=18pt, inner sep=1pt]

        \node[state, initial] (s) {$S$};
        \node[state, accepting, right of=s] (f) {$F$};

        \path[->]
        (s) edge[bend left] node{$0^\star 1 0 0^\star 1 + 0^\star 10$} (f)
        ;
    \end{tikzpicture}
\end{multicols}

Finální regulární výraz pro $L$: $r=0^\star 1 0 0^\star 1 + 0^\star 10$.
