\section{Sedmé cvičení}

% \section*{Regulární výrazy}

\subsection{Konstrukce redukovaného DFA k regulárnímu výrazu}
Je dán regulární výraz \m{(aaa^{\star} + bbb^{\star})^{\star}}. K danému výrazu sestrojte redukovaný DFA $M$,
který přijímá jazyk reprezentovaný regulárním výrazem.

\begin{multicols}{2}

    I. \m{(aaa^{\star} + bbb^{\star})^{\star}}

    \begin{tikzpicture}
        \node[state,initial, accepting] (1) {$q_0$};
        \node[state] (3) [right of=1] {$q_2$};
        \node[state] (2) [right of=3] {$q_1$};
        \node[state] (4) [below of=1] {$q_3$};
        \node[state] (5) [below of=4] {$q_4$};

        \path[->]
        (1) edge [bend left] node {$a$} (3)
        (3) edge [bend left] node {$a$} (2)
        (2) edge [loop above] node {$a$} ()
        (2) edge [bend left] node {$\varepsilon$} (1)
        (1) edge [bend right] node {$b$} (4)
        (4) edge [bend right] node {$b$} (5)
        (5) edge [loop right] node {$b$} ()
        (5) edge [bend right] node {$\varepsilon$} (1)
        ;
    \end{tikzpicture}

\columnbreak

    zjednodušení: výraz se dá přepsat jako \\ II. \m{(aa^{\star}a + bb^{\star}b)^{\star}}

    \begin{tikzpicture}
        \node[state,initial, accepting] (1) {$q_0$};
        \node[state] (3) [right of=1] {$q_1$};
        \node[state] (4) [below of=1] {$q_2$};

        \path[->]
        (1) edge [bend left] node {$a$} (3)
        (3) edge [bend left] node {$a$} (1)
        (3) edge [loop right] node {$a$} ()
        (1) edge [bend right] node {$b$} (4)
        (4) edge [bend right] node {$b$} (1)
        (4) edge [loop right] node {$b$} ()
        ;
    \end{tikzpicture}

    NFA: 
    \begin{tabular}{|r r|c c|}
        \hline
        &  & $a$ & $b$ \\ \hline \hline
        $\leftrightarrow$& $q_0$ & $q_1$          & $q_2$ \\
                        & $q_1$ & $\{q_0, q_1\}$ & $\emptyset$ \\
                        & $q_2$ & $\emptyset$    & $\{q_0, q_2\}$ \\
        \hline
    \end{tabular}
\end{multicols}

Podmnožinová konstrukce pro II.

\begin{tabular}{|r r|c c||c|c c||c|c c||c|}
    \hline
    &  & $a$ & $b$ & $ \sim_0 $ & $a$ & $b$ & $\sim_1$ & $a$ & $b$ & $\sim_2$\\ \hline \hline
    $\leftrightarrow$& $q_0$          & $q_1$          & $q_2$          & $K$ & $O$ & $O$ & $A$ & $B$ & $C$ & $A$ \\
                     & $q_1$          & $\{q_0, q_1\}$ & $\emptyset$    & $O$ & $K$ & $O$ & $B$ & $K$ & $O$ & $B$ \\
              $\gets$& $q_2$          & $\emptyset$    & $\{q_0, q_2\}$ & $O$ & $O$ & $K$ & $C$ & $O$ & $D$ & $C$ \\
                     & $\{q_0, q_1\}$ & $\{q_0, q_1\}$ & $q_2$          & $K$ & $K$ & $O$ & $K$ & $K$ & $C$ & $K$ \\
                     & $\emptyset$    & $\emptyset$    & $\emptyset$    & $O$ & $O$ & $O$ & $O$ & $O$ & $O$ & $O$ \\
              $\gets$& $\{q_0, q_2\}$ & $q_1$          & $\{q_0, q_2\}$ & $K$ & $O$ & $K$ & $D$ & $B$ & $D$ & $D$ \\
    \hline
\end{tabular}

% \begin{tikzpicture} % TODO: dodělat diagram
%     \node[state, initial] (0) {$q_0$};
%     \node[state, right of=0] (1) {$q_1$};
%     \node[state, accepting, right of=1] (2) {$q_2$};
%     \node[state, below of=0] (3) {$q_3$};
%     \node[state, right of=3] (4) {$q_4$};
%     \node[state, accepting, below of=4] (5) {$q_5$};

%     \draw
%         (0) edge[bend left, above] node{$1$} (1)
%         (0) edge[bend right, right] node{$0$} (3)
%         (1) edge[bend left, above] node{$0$} (2)
%         (1) edge[bend left, right] node{$1$} (4)
%         (2) edge[loop below] node{$0, 1$} (2)
%         (3) edge[loop left] node{$0$} (3)
%         (3) edge[bend left] node{$1$} (5)
%         (4) edge[bend right, above] node{$0$} (3)
%         (4) edge[loop right] node{$1$} (4)
%         (5) edge[bend left] node{$0$} (3)
%         (5) edge[bend right, right] node{$1$} (4)
%         ;
% \end{tikzpicture}

\subsection{Konstrukce redukovaného DFA k regulárnímu výrazu}
Je dán regulární výraz \m{(a + b(ab^{\star}a)^{\star}b)^{\star}}. K danému výrazu sestrojte redukovaný DFA $M$,
který přijímá jazyk reprezentovaný regulárním výrazem.

\begin{multicols}{2}
    rozebereme si výraz: \\
    veprostřed máme \m{(ab^{\star}a)^{\star}}

    \begin{tikzpicture}
        \node[state] (1) {$q_1$};
        \node[state] (2) [right of=1] {$q_2$};

        \path[->]
        (1) edge [bend left] node {$a$} (2)
        (2) edge [bend left] node {$a$} (1)
        (2) edge [loop right] node {$b$} (2)
        ;
    \end{tikzpicture}

\columnbreak

    a nabalujeme zbytek

    \m{(a + b(ab^{\star}a)^{\star}b)^{\star}}

    \begin{tikzpicture}
        \node[state, initial, accepting] (1) {$q_0$};
        \node[state] (2) [right of=1] {$q_1$};
        \node[state] (3) [right of=2] {$q_2$};

        \path[->]
        (1) edge [bend left] node {$b$} (2)
        (2) edge [bend left] node {$a$} (3)
        (3) edge [loop right] node {$b$} ()
        (3) edge [bend left] node {$a$} (2)
        (2) edge [bend left] node {$b$} (1)
        (1) edge [loop below] node {$a$} ()
        ;
    \end{tikzpicture}
\end{multicols}

Podmnožinová konstrukce.

\begin{tabular}{|r r|c c||c|c c||c|}
    \hline
    &  & $a$ & $b$ & $\sim_0$ & $a$ & $b$ & $\sim_1$ \\ \hline \hline
    $\leftrightarrow$&$q_0$ & $q_0$ & $q_1$ & $K$ & $K$ & $O$ & $K$\\
                    & $q_1$ & $q_2$ & $q_0$ & $O$ & $O$ & $K$ & $A$\\
                    & $q_2$ & $q_1$ & $q_2$ & $O$ & $O$ & $O$ & $O$\\
    \hline
\end{tabular} 
Protože každý řádek má svou třídu, automat je již redukovaný.

\newpage

\subsection{Konstrukce redukovaného DFA k regulárnímu výrazu}
Je dán regulární výraz \m{a(ab^{\star} + b)^{\star}ba}. K danému regulární výrazu sestrojte redukovaný DFA $M$,
který přijímá jazyk reprezentovaný regulárním výrazem.

\m{a(ab^{\star} + b)^{\star}ba}: \\$a_1(a_2 b_3^{\star} + b_4)^{\star} b_5 a_6$

\begin{multicols}{3}

    \begin{tabular}{|c c|c c|}
        \hline
        NFA& & $a$ & $b$ \\
        \hline
        \hline
        $\rightarrow$ & $S$ & $1$& $-$\\
        &$ 1$ & $2$ & $4,5$ \\
        &$ 2$ & $2$ & $3,4,5$ \\
        $$ &$ 3$ & $2$ & $3,4,5$ \\
        &$ 4$ & $2$ & $4,5$ \\
        & 5& 6 & $-$ \\
        $\leftarrow$& 6& $-$ &$-$ \\
        \hline
    \end{tabular}

\columnbreak

    \begin{tikzpicture}
        \node[state, initial] (S) {$S$};
        \node[state] (1) [right of=S] {$1$};
        \node[state] (2) [right of=1] {$2$};
        \node[state] (3) [right of=2] {$3$};
        \node[state] (5) [below of=1] {$5$};
        \node[state] (4) at ($(1)!0.5!(5)!0.5!(2)$) {$4$};
        \node[state, accepting] (6) [left of=5] {$6$};

        \path[->]
            (S) edge [bend left] node {$a$} (1)

            (1) edge [bend left] node {$a$} (2)
            (1) edge [bend left] node {$b$} (4)
            (1) edge [bend right] node {$b$} (5)

            (2) edge [loop above] node {$a$} ()
            (2) edge [bend left] node {$b$} (3)
            (2) edge [bend left] node {$b$} (4)
            (2) edge [bend left] node {$b$} (5)

            (3) edge [bend left] node {$a$} (2)
            (3) edge [loop above] node {$b$} (3)
            (3) edge [bend left] node {$b$} (4)
            (3) edge [bend left] node {$b$} (5)

            (4) edge [bend left] node {$a$} (2)
            (4) edge [loop left] node {$b$} ()
            (4) edge [bend left] node {$b$} (5)

            (5) edge [bend left] node {$a$} (6)

            ;
    \end{tikzpicture}

\end{multicols}

Podmnožinová konstrukce.

\begin{tabular}{|r r|c c|c|c c|c|c c|c|c c|c|c c|c|}
    \hline
    & & $a$ & $b$ & $\sim_0$ & $a$ & $b$ & $\sim_1$ & $a$ & $b$ & $\sim_2$ & $a$ & $b$ & $\sim_3$ & $a$ & $b$ & $\sim_4$\\
    \hline
    \hline
    $\rightarrow$ &$\{ S\}$     &$\{ 1\}$    &$\emptyset$  &$O$ &$O$ &$O$ &$O$ &$O$ &$O$ &$O$ &$B$ &$O$ &$C$ &$B$ &$O$ &$C$\\
                  &$\{ 1\}$     &$\{ 2\}$    &$\{ 4,5 \}$  &$O$ &$O$ &$O$ &$O$ &$O$ &$A$ &$B$ &$B$ &$A$ &$B$ &$B$ &$A$ &$B$\\
                  &$\emptyset$  &$\emptyset$ &$\emptyset$  &$O$ &$O$ &$O$ &$O$ &$O$ &$O$ &$O$ &$O$ &$O$ &$O$ &$O$ &$O$ &$O$\\
                  &$\{ 2\}$     &$\{ 2\}$    &$\{ 3,4,5\}$ &$O$ &$O$ &$O$ &$O$ &$O$ &$A$ &$B$ &$B$ &$A$ &$B$ &$B$ &$A$ &$B$\\
                  &$\{ 4,5 \}$  &$\{ 2,6 \}$ &$\{ 4,5 \}$  &$O$ &$K$ &$O$ &$A$ &$K$ &$A$ &$A$ &$K$ &$A$ &$A$ &$K$ &$A$ &$A$\\
                  &$\{ 3,4,5\}$ &$\{ 2,6 \}$ &$\{ 3,4,5\}$ &$O$ &$K$ &$O$ &$A$ &$K$ &$A$ &$A$ &$K$ &$A$ &$A$ &$K$ &$A$ &$A$\\
    $\leftarrow$  &$\{ 2,6 \}$  &$\{ 2\}$    &$\{ 3,4,5\}$ &$K$ &$O$ &$O$ &$K$ &$O$ &$A$ &$K$ &$B$ &$A$ &$K$ &$B$ &$A$ &$K$\\
    \hline
\end{tabular}

DFA:

\begin{tikzpicture}
    \def\radius{22mm}

    \node[state, initial] (D) at (90:\radius) {$C$}; % Top center
    \node[state] (B) at (162:\radius) {$B$}; % Top-left
    \node[state] (O) at (18:\radius) {$O$}; % Bottom-left
    \node[state] (A) at (306:\radius) {$A$}; % Bottom-right
    \node[state, accepting] (K) at (234:\radius) {$K$}; % Top-right
    \path[->]
    (D) edge [bend left] node {$a$} (B)
    (D) edge [bend left] node {$b$} (O)

    (B) edge [loop above] node {$a$} ()
    (B) edge [bend left] node {$b$} (A)

    (O) edge [loop above] node {$a,b$} ()

    (A) edge [loop below] node {$b$} ()
    (A) edge [bend right] node {$a$} (K)

    (K) edge [bend left] node {$a$} (B)
    (K) edge [bend right] node {$b$} (A)
    ;
\end{tikzpicture}

\newpage

\subsection{Tvorba regulárního výrazu z DFA}
Pro daný DFA $M$ vytvořte regulární výraz, který reprezentuje jazyk $L(M)$.

\begin{multicols}{2}

    $M$: \hspace{2mm}
    \begin{tabular}{|c c|c c|}
        \hline
         & & $a$ & $b$ \\
        \hline
        \hline
        $\leftrightarrow$&$ 1$ & $1$ & $2$ \\
        \hline
        &$ 2$ & $3$ & $4$ \\
        \hline
        $\leftarrow$ &$ 3$ & $1$ & $2$ \\
        \hline
        &$ 4$ & $4$ & $4$ \\
        \hline
    \end{tabular}

    0. DFA:

    \begin{tikzpicture}[shorten >=1pt,node distance=20mm,on grid,auto]
        \tikzstyle{every state}=[fill={rgb:black,1;white,10}, minimum size=18pt, inner sep=1pt]
        \node[state,initial] (1) {$1$};
        \node[state, accepting] (3) [below of=1] {$3$};
        \node[state] (2) [right of=1] {$2$};
        \node[state] (4) [right of=3] {$4$};

        \path[->]
        (1) edge [loop above] node {$a$} ()
        (1) edge [bend left] node {$b$} (2)
        (2) edge [bend right] node {$a$} (3)
        (2) edge [bend left] node {$b$} (4)
        (3) edge [bend left] node {$a$} (1)
        (3) edge [bend right] node {$b$} (2)
        (4) edge [loop right] node {$a,b$} ()
        ;
    \end{tikzpicture}

    1. zavedu stavy $S$, $F$:

    \begin{tikzpicture}[shorten >=1pt,node distance=20mm,on grid,auto]
        \tikzstyle{every state}=[fill={rgb:black,1;white,10}, minimum size=18pt, inner sep=1pt]

        \node[state,initial] (S) {$S$};
        \node[state] (1) [right of=S]{$1$};
        \node[state, accepting] (F) [below of=S] {$F$};
        \node[state] (3) [below of=1] {$3$};
        \node[state] (2) [right of=1] {$2$};
        \node[state] (4) [right of=3] {$4$};

        \path[->]
        (S) edge [bend left] node {$\varepsilon$} (1)
        (1) edge [bend right] node {$\varepsilon$} (F)
        (3) edge [bend left] node {$\varepsilon$} (F)
        (1) edge [loop above] node {$a$} ()
        (1) edge [bend left] node {$b$} (2)
        (2) edge [bend right] node {$a$} (3)
        (2) edge [bend left] node {$b$} (4)
        (3) edge [bend left] node {$a$} (1)
        (3) edge [bend right] node {$b$} (2)
        (4) edge [loop right] node {$a,b$} ()
        ;
    \end{tikzpicture}

    2. odstraňuji smyčky:

    \begin{tikzpicture}[shorten >=1pt,node distance=20mm,on grid,auto]
        \tikzstyle{every state}=[fill={rgb:black,1;white,10}, minimum size=18pt, inner sep=1pt]

        \node[state,initial] (S) {$S$};
        \node[state] (1) [right of=S]{$1$};
        \node[state, accepting] (F) [below of=S] {$F$};
        \node[state] (3) [below of=1] {$3$};
        \node[state] (2) [right of=1] {$2$};
        \node[state] (4) [right of=3] {$4$};

        \path[->]
        (S) edge [bend left] node {$\varepsilon$} (1)
        (1) edge [bend right] node {$a^{\star}$} (F)
        (3) edge [bend left] node {$\varepsilon$} (F)
        (1) edge [bend left] node {$a^{\star}b$} (2)
        (2) edge [bend right] node {$a$} (3)
        (2) edge [bend left] node {$b$} (4)
        (3) edge [bend left] node {$a$} (1)
        (3) edge [bend right] node {$b$} (2)
        ;
    \end{tikzpicture}

    3. odstraňuji vrchol $4$:

    \begin{tikzpicture}[shorten >=1pt,node distance=20mm,on grid,auto]
        \tikzstyle{every state}=[fill={rgb:black,1;white,10}, minimum size=18pt, inner sep=1pt]

        \node[state,initial] (S) {$S$};
        \node[state] (1) [right of=S]{$1$};
        \node[state, accepting] (F) [below of=S] {$F$};
        \node[state] (3) [below of=1] {$3$};
        \node[state] (2) [right of=1] {$2$};

        \path[->]
        (S) edge [bend left] node {$\varepsilon$} (1)
        (1) edge [bend right] node {$a^{\star}$} (F)
        (3) edge [bend left] node {$\varepsilon$} (F)
        (1) edge [bend left] node {$a^{\star}b$} (2)
        (2) edge [bend right] node {$a$} (3)
        (3) edge [bend left] node {$a$} (1)
        (3) edge [bend right] node {$b$} (2)
        ;
    \end{tikzpicture}

\columnbreak

    4. odstraňuji vrchol $2$:

    \begin{tikzpicture}[shorten >=1pt,node distance=20mm,on grid,auto]
        \tikzstyle{every state}=[fill={rgb:black,1;white,10}, minimum size=18pt, inner sep=1pt]

        \node[state,initial] (S) {$S$};
        \node[state] (1) [right of=S]{$1$};
        \node[state, accepting] (F) [below of=S] {$F$};
        \node[state] (3) [below of=1] {$3$};

        \path[->]
        (S) edge [bend left] node {$\varepsilon$} (1)
        (1) edge [bend right] node {$a^{\star}$} (F)
        (3) edge [bend left] node {$\varepsilon$} (F)
        (1) edge [bend left] node {$a^{\star}ba$} (3)
        (3) edge [bend left] node {$a$} (1)
        (3) edge [loop right] node {$ba$} ()
        ;
    \end{tikzpicture}

    5. odstraňuji smyčky:

    \begin{tikzpicture}[shorten >=1pt,node distance=20mm,on grid,auto]
        \tikzstyle{every state}=[fill={rgb:black,1;white,10}, minimum size=18pt, inner sep=1pt]

        \node[state,initial] (S) {$S$};
        \node[state] (1) [right of=S]{$1$};
        \node[state, accepting] (F) [below of=S] {$F$};
        \node[state] (3) [below of=1] {$3$};

        \path[->]
        (S) edge [bend left] node {$\varepsilon$} (1)
        (1) edge [bend right] node[swap] {$a^{\star}$} (F)
        (3) edge [bend left] node {$(ba)^{\star}$} (F)
        (1) edge [bend left] node {$a^{\star}ba$} (3)
        (3) edge [bend left] node {$(ba)^{\star}a$} (1)
        ;
    \end{tikzpicture}

    6. odstraňuji vrchol $3$:

    \begin{tikzpicture}[shorten >=1pt,node distance=20mm,on grid,auto]
        \tikzstyle{every state}=[fill={rgb:black,1;white,10}, minimum size=18pt, inner sep=1pt]

        \node[state,initial] (S) {$S$};
        \node[state] (1) [right of=S]{$1$};
        \node[state, accepting] (F) [below of=S] {$F$};

        \path[->]
        (S) edge [bend left] node {$\varepsilon$} (1)
        (1) edge [bend right] node {$a^{\star}$} (F)
        (1) edge [bend left] node {$a^{\star}ba(ba)^{\star}$} (F)
        (1) edge [loop right] node {$a^{\star}ba(ba)^{\star}a$} ()
        ;
    \end{tikzpicture}

    7. odstraňuji paralelní hrany:

    \begin{tikzpicture}[shorten >=1pt,node distance=20mm,on grid,auto]
        \tikzstyle{every state}=[fill={rgb:black,1;white,10}, minimum size=18pt, inner sep=1pt]

        \node[state,initial] (S) {$S$};
        \node[state] (1) [right of=S]{$1$};
        \node[state, accepting] (F) [below of=S] {$F$};

        \path[->]
        (S) edge [bend left] node {$\varepsilon$} (1)
        (1) edge [bend right] node {$a^{\star} + (a^{\star}ba(ba)^{\star})$} (F)
        (1) edge [loop right] node {$a^{\star}ba(ba)^{\star}a$} ()
        ;
    \end{tikzpicture}

    8. odstraňuji smyčky:

    \begin{tikzpicture}[shorten >=1pt,node distance=22mm,on grid,auto]
        \tikzstyle{every state}=[fill={rgb:black,1;white,10}, minimum size=18pt, inner sep=1pt]

        \node[state,initial] (S) {$S$};
        \node[state] (1) [right of=S]{$1$};
        \node[state, accepting] (F) [right of=1] {$F$};

        \path[->]
        (S) edge [bend left] node {$\varepsilon$} (1)
        (1) edge [bend left] node {\hspace*{21mm}$(a^{\star}ba(ba)^{\star}a)^{\star}(a^{\star}+(a^{\star}ba(ba)^{\star}))$}
        (F)
        ;
    \end{tikzpicture}

    9. odstraňuji vrchol $1$:

    \begin{tikzpicture}[shorten >=1pt,node distance=65mm,on grid,auto]
        \tikzstyle{every state}=[fill={rgb:black,1;white,10}, minimum size=18pt, inner sep=1pt]

        \node[state,initial] (S) {$S$};
        \node[state, accepting] (F) [right of=S] {$F$};

        \path[->]
        (S) edge [above] node {$(a^{\star}ba(ba)^{\star}a)^{\star}(a^{\star} + (a^{\star}ba(ba)^{\star}))$} (F)
        ;
    \end{tikzpicture}

\end{multicols}

\newpage
lifehack: dá se redukovat (stavy $1$ a $3$ jsou ekvivalentní):

\begin{multicols}{2}
    \begin{tabular}{|c c|c c|}
        \hline
         & & $a$ & $b$ \\
        \hline
        \hline
        $\leftrightarrow$&$ 1$ & $1$ & $2$ \\
        \hline
        &$ 2$ & $3$ & $4$ \\
        \hline
        $\leftarrow$ &$ 3$ & $1$ & $2$ \\
        \hline
        &$ 4$ & $4$ & $4$ \\
        \hline
    \end{tabular}

\columnbreak

    \begin{tikzpicture}[shorten >=1pt,node distance=20mm,on grid,auto]
        \tikzstyle{every state}=[fill={rgb:black,1;white,10}, minimum size=18pt, inner sep=1pt]
        \node[state, initial, accepting] (1) {$1, 3$};
        \node[state] (2) [right of=1] {$2$};
        \node[state] (4) [right of=2] {$4$};

        \path[->]
        (1) edge [loop above] node {$a$} (a)
        (1) edge [bend left] node {$b$} (2)
        (2) edge [bend left] node {$b$} (4)
        (2) edge [bend left] node {$a$} (1)
        (4) edge [loop right] node {$b$} ()
        ;
    \end{tikzpicture}
\end{multicols}

\begin{multicols}{2}

    \bb{1. přidám $S$, $F$:}

    \begin{tikzpicture}[shorten >=1pt,node distance=20mm,on grid,auto]
        \tikzstyle{every state}=[fill={rgb:black,1;white,10}, minimum size=18pt, inner sep=1pt]
        \node[state, initial] (S) {$S$};
        \node[state] (1) [right of=S]{$1, 3$};
        \node[state] (2) [right of=1] {$2$};
        \node[state] (4) [right of=2] {$4$};
        \node[state, accepting] (F) [below of=S] {$F$};

        \path[->]
        (S) edge [bend left] node {$\varepsilon$} (1)
        (1) edge [loop above] node {$a$} (a)
        (1) edge [bend left] node {$b$} (2)
        (2) edge [bend left] node {$b$} (4)
        (2) edge [bend left] node {$a$} (1)
        (4) edge [loop right] node {$b$} ()
        (1) edge [bend right] node {$\varepsilon$} (F)
        ;
    \end{tikzpicture}

    \bb{2. zbavím se nodu č. 4:}

    \begin{tikzpicture}[shorten >=1pt,node distance=20mm,on grid,auto]
        \tikzstyle{every state}=[fill={rgb:black,1;white,10}, minimum size=18pt, inner sep=1pt]
        \node[state, initial] (S) {$S$};
        \node[state] (1) [right of=S]{$1, 3$};
        \node[state] (2) [right of=1] {$2$};
        \node[state, accepting] (F) [below of=S] {$F$};

        \path[->]
        (S) edge [bend left] node {$\varepsilon$} (1)
        (1) edge [loop above] node {$a$} (a)
        (1) edge [bend left] node {$b$} (2)
        (2) edge [bend left] node {$a$} (1)
        (1) edge [bend right] node {$\varepsilon$} (F)
        ;
    \end{tikzpicture}

\columnbreak

\bb{3. odstraním node č. 2:}

    \begin{tikzpicture}[shorten >=1pt,node distance=20mm,on grid,auto]
        \tikzstyle{every state}=[fill={rgb:black,1;white,10}, minimum size=18pt, inner sep=1pt]
        \node[state, initial] (S) {$S$};
        \node[state] (1) [right of=S]{$1, 3$};
        \node[state, accepting] (F) [right of=1] {$F$};

        \path[->]
        (S) edge [bend left] node {$\varepsilon$} (1)
        (1) edge [loop above] node {$a$} (a)
        (1) edge [loop below] node {$ba$} (1)
        (1) edge [loop above] node {$a$} (1)
        (1) edge [bend left] node {$\varepsilon$} (F)
        ;
    \end{tikzpicture}


\bb{4.spojím obsah smyček, \\odstraním smyčku a node:}

\vspace*{2mm}
    \begin{tikzpicture}[shorten >=1pt,node distance=20mm,on grid,auto]
        \tikzstyle{every state}=[fill={rgb:black,1;white,10}, minimum size=18pt, inner sep=1pt]
        \node[state, initial] (S) {$S$};
        \node[state, accepting] (F) [right of=1] {$F$};

        \path[->]
        (S) edge [above] node {$(a + ba)^{\star}$} (F)
        ;
    \end{tikzpicture}

\end{multicols}
