%%% Ugly hack for no spacing around alignments, just a local thing
\setlength{\abovedisplayskip}{0pt}
\setlength{\belowdisplayskip}{0pt}
\setlength{\abovedisplayshortskip}{0pt}
\setlength{\belowdisplayshortskip}{0pt}
%%%

\section{Deváté cvičení}
\subsection{Návrh bezkontextové gramatiky}
Navrhněte bezkontextové gramatiky generující následující jazyky
\begin{enumerate}[a), noitemsep]
    \item $L_1 = \bc{0^{m+n} 1^n 0^m \mid 0 \leq n,m}$.
    \item $L_2 = \bc{0^i 1^j \mid 0 \leq i < j}$.
\end{enumerate}
Zdůvodněte, proč gramatika $\G$ jazyk $L$ generuje.

\begin{enumerate}[a)]
    \item 
        \begin{flalign*}
            P\text{: } & S \rightarrow 0S0 \mid A & \\
            & A \rightarrow 0A1 \mid \varepsilon &
        \end{flalign*}
        \begin{enumerate}[noitemsep]
            \item $\quad S \stackrel{S \rightarrow 0S0 }{\Longrightarrow}^{(m)} 0^m S 0^m \stackrel{S \rightarrow A}
            {\Longrightarrow} 0^m A 0^m \stackrel{A \rightarrow 0A1}{\Longrightarrow}^{(n)} 0^m 0^n A 1^n 0^m \stackrel
            {A \rightarrow \varepsilon}{\Longrightarrow} 0^m 1^m 0^n 1^n 0^m = 0^{m+n}1^n0^m$.
            \item $\quad S \Rightarrow^\star w, \, S \Rightarrow 0^i S 0^i, \, S \Rightarrow A, \, A \Rightarrow^\star 0^j 
            1^j, \, 0^i 0^j 1^j 0^i \in L(\G)$.
        \end{enumerate}

    \item 
        \begin{flalign*}
            P\text{: } & S \rightarrow 0S1 \mid S1 \mid 1 &
        \end{flalign*}
        \begin{enumerate}[noitemsep]
            \item $\quad S \stackrel{S \rightarrow S1 }{\Longrightarrow}^{(j-i-1)} S 0^{j-i} \stackrel{S \rightarrow 0S1}
            {\Longrightarrow}^{(i)} 0^i S 1^{(j-i+i-1)} \stackrel{S \rightarrow 1}{\Longrightarrow} 0^i S 1^j$.
            \item $\quad S \Rightarrow^\star w, S \stackrel{S \rightarrow 0S1 }{\Longrightarrow}^{(m)} 0^m S 1^m , \, 
            S \stackrel{S \rightarrow S1 }{\Longrightarrow}^{(n)} S1^n , \, 
            S \stackrel{S \rightarrow 1}{\Longrightarrow} 1 \text{. } 0^m  1^{m+1} \text{ nebo } 0^m 1^{m+n+1}$. \\
            Obě $\in L(\G)$.
        \end{enumerate}
\end{enumerate}

\subsection{Konstrukce nevypouštěcí gramatiky} % 9.2
Ke gramatice $\G$ zkonstruujte nevypouštěcí gramatiku $\G_1$, pro kterou 
$L(\G_1) = L(\G) - \bc{\varepsilon}$. Gramatiku $\G_1$ zredukujte.
\begin{align*}
    P\text{: } & S \rightarrow AB \mid \varepsilon \\
    & A \rightarrow aAAb \mid bS \mid CA \\
    & B \rightarrow BbA \mid CaC \mid \varepsilon \\
    & C \rightarrow aBB \mid bS
\end{align*}
$V_1 = \bc{A \mid A \rightarrow \varepsilon \in P} = \bc{S, B}$\\
$V_2 = V_1 \cup \bc{A \mid A \rightarrow \alpha \in P, \alpha \in V_1^{\star}} = V_1 \cup \emptyset = \bc{S, B}$
    \begin{flalign*}
        \G_1\text{: } & S \rightarrow AB \mid A & \\
        & A \rightarrow aAAb \mid bS \mid b \mid CA & \\
        & B \rightarrow BbA \mid bA \mid CaC & \\
        & C \rightarrow aBB \mid aB \mid a \mid bS \mid b &
    \end{flalign*}
Gramatika $\G_1$ už je redukovaná. 

\textbf{Obecný postup pro redukci}

$V = \bc{A \mid A \in N, A \Rightarrow^{\star}_{\G} w, w \in \Sigma^{\star}}$\\
$V_1 = \bc{A \mid A \Rightarrow_{\star} w \in P, w \in \Sigma^{\star}}$\\
$V_2 = V_1 \cup \bc{A \mid A \rightarrow \alpha \in P, \alpha \in (\Sigma \cup V_1)^{\star}}$\\
$U = \bc{A \mid A \in V, \exists \alpha, \beta\in (V \cup \Sigma)^{\star} \text{ tak, že } S \Rightarrow^{\star}_{\G} 
\alpha A \beta}$

Jazyk není prázdný právě tehdy, kdy $S \in V$. 

\subsection{Redukce gramatiky}

Zredukujte gramatiku $\G$, která je dána pravidly 
\begin{align*}
    P\text{: } & S \rightarrow SA \mid SB \mid \varepsilon \\
    & A \rightarrow bSA \mid baS \\
    & B \rightarrow aB \mid Ba \mid DA\\
    & C \rightarrow aCB \mid bA \\
    & D \rightarrow AB
\end{align*}
redukce tldr:\\
$V_1\dots$ to, co se promítne na $\varepsilon$ nebo na terminály \\ 
$V_2\dots$ to, co se promítne na terminály a na to, co už je ve $V_1$ \\ 
$U_0\dots$ $\bc{S}$ \\
$U_1\dots$ neterminály, do kterých se dostanu z $S$, pak z $U_1$ atd. 

$V_1 = \bc{S}$\\ 
$V_2 = V_1 \cup \bc{A} = \bc{S, A}$\\ 
$V_3 = V_2 \cup \bc{C} = \bc{S, A, C}$\\ 
$V_4 = V_3 \cup \emptyset = \bc{S, A, C}$ 

Zde nechám jenom neterminály z $V$ a z pravé strany vyškrtám pravidla obsahující neterminály $\not\in V$:
\begin{flalign*}
    P\text{: } & S \rightarrow SA \mid \varepsilon & \\
    & A \rightarrow bSA \mid baS & \\
    & C \rightarrow bA &
\end{flalign*}
Sem přidávám neterminály, do kterých se dostanu z počátečního stavu $S$, pak ze stavů v odpovídajícím~$U_i$: 

$U_0 = \bc{S}$\\ 
$U_{i+1} = U_i \cup \bc{A \mid \text{existují } B \in U_i, \alpha, \beta \in (V \cup \Sigma)^\star \text{ tak, že } B 
\rightarrow \alpha A \beta \in P^{'}}$\\
$U_1 = U_0 \cup \bc{A} = \bc{S, A}$ \\
$U_2 = U_1 \cup \emptyset = \bc{S, A}$ 

A ponechám jen pravidla, která nám zbyla v $U$: 
\begin{flalign*}
    P\text{: } & S \rightarrow SA \mid \varepsilon & \\
    & A \rightarrow bSA \mid baS &
\end{flalign*}

\subsection{Důkaz, že gramatika generuje alespoň jedno slovo} % 9.4

Rozhodněte, zda gramatika $\G$ generuje alespoň jedno slovo, tj. zda $L(\G) \neq \emptyset$, kde $\G$ je dána pravidly 
\begin{align*}
    P\text{: } & S \rightarrow aS \mid AB \mid CD \\
    & A \rightarrow aDb \mid AD \mid BC \\
    & B \rightarrow bSb \mid BB \\
    & C \rightarrow BA \mid ASb \\
    & D \rightarrow ABCD \mid \varepsilon
\end{align*}
$V_1 = \bc{D}$\\ 
$V_2 = V_1 \cup \bc{A} = \bc{D, A}$\\ 
$V_3 = V_2 \cup \emptyset = \bc{D, A} = V$\\
$S \notin V \quad \rightarrow \quad L(\G) = \emptyset$. Tedy gramatika $\G$ negeneruje ani jedno slovo.

\textbf{Chomského normální tvar polopatě}

Všechna pravidla na pravé straně mají buď přesně 2 neterminály nebo přesně 1 terminál bez $\varepsilon$.

\subsection{\href{https://youtu.be/jwrE-ez7S8I?list=PLQL6z4JeTTQkLuzI78OTnfYBclE1g0UjS&t=1846}{Příklad z přednášky} - Převod gramatiky do Chomského normálního tvaru}
Je dána gramatika $\G = (N, \Sigma, S, P)$, kde $N = \bc{S, A, C}$, $\Sigma = \bc{a, b}$ a P je 
\begin{align*}
    & S \rightarrow SCA \mid a \\
    & A \rightarrow aCb \mid \varepsilon \\
    & C \rightarrow AA \mid bb
\end{align*}
Převeďte $\G$ do Chomského normálního tvaru. 

\textbf{1. krok} Vytvořit nevypouštěcí gramatiku. 

$V_1 = \bc{A}$\\
$V_2 = \bc{A} \cup \bc{C} = \bc{A,C} = V$
\begin{flalign*}
    P^{'}\text{: } & S \rightarrow SCA \mid SA \mid SC \mid a & \\
    & A \rightarrow aCb \mid ab & \\
    & C \rightarrow AA \mid A \mid b &
\end{flalign*}
\textbf{2. krok} Nahrazení právě jednoho neterminálu pravých stran jeho pravidly. 
\begin{flalign*}
    & S \rightarrow SCA \mid SA \mid SC \mid a & \\
    & A \rightarrow aCb \mid ab & \\
    & C \rightarrow AA \mid aCb \mid ab \mid b &
\end{flalign*}
\textbf{3. krok} Vytvořit pomocná pravidla pro terminály, které "nezůstaly samy"
\begin{flalign*}
    & S \rightarrow SCA \mid SA \mid SC \mid a & \\
    & A \rightarrow X_aCX_b \mid X_aX_b & \\
    & C \rightarrow AA \mid X_aCX_b \mid X_aX_b \mid b & \\
    & X_a \rightarrow a & \\
    & X_b \rightarrow b &
\end{flalign*}
\textbf{4. krok} Rozbít dlouhá slova ($\geq 3$) (opět vytvořím pomocná pravidla).
\begin{flalign*}
    P^{''}\text{: } & S \rightarrow SB \mid SA \mid SC \mid a & \\
    & B \rightarrow CA & \\
    & A \rightarrow X_aD\mid X_aX_b & \\
    & D \rightarrow CX_b & \\ 
    & C \rightarrow AA \mid X_aD \mid X_aX_b \mid b & \\
    & X_a \rightarrow a & \\
    & X_b \rightarrow b &
\end{flalign*}

Gramatika $G_1 = (\bc{S,A,B,C,D,X_a,X_b}, \bc{a,b}, S, P^{''})$ je v Chomského normálním tvaru.

\subsection{Převod gramatiky do Chomského normálního tvaru} % 9.5 
Je dána gramatika $\G = (N, \Sigma, S, P)$, kde $N = \bc{S, A, B}$, $\Sigma = \bc{0, 1}$ a P je 
\begin{align*}
    P\text{: } & S \rightarrow A \mid 0SA \mid \varepsilon \\
    & A \rightarrow 1A \mid B1 \mid 1 \\
    & B \rightarrow 0B \mid 0
\end{align*}
Převeďte $\G$ do Chomského normálního tvaru. 

\textbf{1. krok} Vytvořit nevypouštěcí gramatiku. 

$V = \bc{S}$
\begin{flalign*}
    P\text{: } & S \rightarrow A \mid 0SA \mid 0A & \\
    & A \rightarrow 1A \mid B1 \mid 1 & \\
    & B \rightarrow 0B \mid 0 &
\end{flalign*}
\textbf{2. krok} Nahrazení právě jednoho neterminálu pravých stran jeho pravidly. 
\begin{flalign*}
    P\text{: } & S \rightarrow 1A \mid B1 \mid 1 \mid 0SA \mid 0A & \\
    & A \rightarrow 1A \mid B1 \mid 1 & \\
    & B \rightarrow 0B \mid 0 &
\end{flalign*}
\textbf{3. krok} Vytvořit pomocná pravidla pro terminály, které "nezůstaly samy" 
\begin{flalign*}
    P\text{: } & S \rightarrow X_1A \mid BX_1 \mid 1 \mid X_0SA \mid X_0A & \\
    & A \rightarrow X_1A \mid BX_1 \mid 1 & \\
    & B \rightarrow X_0B \mid 0 & \\
    & X_0 \rightarrow 0 & \\
    & X_1 \rightarrow 1 &
\end{flalign*}
\textbf{4. krok} Rozbít dlouhá slova ($\geq 3$) (opět vytvořím pomocná pravidla). 
\begin{flalign*}
    P\text{: } & S \rightarrow X_1A \mid BX_1 \mid 1 \mid X_0Y \mid X_0A  & \\
    & Y \rightarrow SA & \\
    & A \rightarrow X_1A \mid BX_1 \mid 1 & \\
    & B \rightarrow X_0B \mid 0 & \\
    & X_0 \rightarrow 0 & \\
    & X_1 \rightarrow 1 &
\end{flalign*}

\subsection{Práce s derivačním stromem bezkontextové gramatiky} % 9.6 
Je dán derivační strom v bezkontextové gramatice: 

\begin{center}
    \begin{forest}
    for tree={
        grow=south,                 % Tree grows downward
        edge={->},                   % Draw edges as lines
        align=center,               % Center align text inside nodes
        parent anchor=south,        % Parent anchor point
        child anchor=north        % Child anchor point
    }
    [$S$
        [$A$
            [$A$
                [$a$]
                [$S$
                    [$\varepsilon$]
                ]
                [$b$]
            ]
            [$A$
                [$\varepsilon$]
            ]
        ]
        [$a$]
        [$B$, text=red
            [$S$
                [$A$
                    [$\varepsilon$]
                ]
                [$a$]
                [$B$, text=red
                    [$a$]
                    [$c$]
                ]
            ]
            [$A$
                [$S$
                    [$\varepsilon$]
                ]
                [$B$
                    [$a$]
                    [$c$]
                ]
            ]
        ]
    ]
    \end{forest}
\end{center}
\begin{enumerate}[label=\alph*), noitemsep]
    \item Napiště pravidla minimální CF gramatiky, ve které je to derivační strom. 
    \item Napiště levou derivaci odpovídající tomuto derivačnímu stromu. 
    \item Rozlože výsledek derivačního stromu $w$ na pět částí $w = w_1 w_2 w_3 w_4 w_5$ tak, že $w_2 w_4 
    \neq \varepsilon$ a slovo $w_1 w_2^2 w_3 w_4^2 w_5$ je také generované gramatikou z bodu a). 
    \item Rozhodněte, zda je gramatika víceznačná. 
\end{enumerate}
\begin{flalign*}
    a) P\text{: } & S \rightarrow AaB \mid \varepsilon & \\
    & A \rightarrow AA \mid SB \mid aSb \mid \varepsilon & \\
    & B \rightarrow SA \mid ac &
\end{flalign*}
b) 
$\quad S \stackrel{S \rightarrow A a B}{\Longrightarrow} A a B \stackrel{A \rightarrow A A}{\Longrightarrow} 
A A a B \stackrel{A \rightarrow a S b}{\Longrightarrow} a S b A a B \stackrel{S \rightarrow \varepsilon}
{\Longrightarrow} a b A a B \stackrel{A \rightarrow \varepsilon}{\Longrightarrow} a b a B \stackrel
{B \rightarrow SA}{\Longrightarrow} a b a S A \stackrel{A\rightarrow Aab}{\Longrightarrow} abaAaBA \stackrel
{A\rightarrow \varepsilon}{\Longrightarrow} a b a aBA \stackrel{B \rightarrow ac}{\Longrightarrow} abaaacA 
\stackrel{A \rightarrow SB}{\Longrightarrow} abaaacSB \stackrel{S \rightarrow \varepsilon}{\Longrightarrow} 
abaaacB \stackrel{B \rightarrow ac}{\Longrightarrow} abaaacac $.

\vspace*{1mm}
c) basically pumping lemma pro gramatiky \\
$w_1 = aba$, $w_2 = a$, $w_3 = ac$, $w_4 = ac$, $w_5 = \varepsilon$. 

\begin{center}
    \begin{forest}
            for tree={
                grow=south,                 % Tree grows downward
                edge={->},                  % Draw edges as arrows
                align=center,               % Center the text inside nodes
            }
            [$S$
            [$A$
                [$A$
                    [$a$]
                    [$S$
                        [$\varepsilon$]
                    ]
                    [$b$]
                ]
                [$A$
                    [$\varepsilon$]
                ]
            ]
            [$a$]
            [$B$, text=red
                [$S$
                    [$A$
                        [$\varepsilon$]
                    ]
                    [$a$]
                    [$B$, text=red
                        [$S$
                            [$A$
                                [$\varepsilon$]
                            ]
                            [$a$]
                            [$B$
                                [$a$]
                                [$c$]
                            ]
                        ]
                        [$A$
                            [$A$
                                [$\varepsilon$]
                            ]
                            [$B$
                                [$a$]
                                [$c$]
                            ]
                        ]
                    ]
                ]
                [$A$
                    [$S$
                        [$\varepsilon$]
                    ]
                    [$B$
                        [$a$]
                        [$c$]
                    ]
                ]
            ]
        ]
    \end{forest}    
\end{center}
Jak toho docílím: jdu odspoda stromu a najdu dva stejné neterminály (různě v grafu) a v podstatě ten strom nafouknu 
(zkopíruju nějaký podstrom z vyššího do nižšího neterminálu, v ukázce červený node $B$ v nejvýš vpravo kopíruju do 
červeného $B$čka vlevo od něj).

d) je víceznačná, přepisujeme $A \rightarrow AA$. 

\subsection{\href{https://youtu.be/jwrE-ez7S8I?list=PLQL6z4JeTTQkLuzI78OTnfYBclE1g0UjS&t=3207}{Příklad z přednášky} - Pumping lemma pro bezkontextové gramatiky}
Ukážeme, že jazyk $L = \bc{0^n 1^n 2^n \mid n \geq 0}$ není bezkontextový.

Kdyby $L$ byl CF, tak existuje $m \geq 1$ s vlastnostmi 1,2,3.

Zvolme $z = 0^m 1^m 2^m \in L, |z| = 3m \geq m$.

Kdyby $z = uvwxy$, že $|vwx| \leq m$. 
\[
\text{Tak $vwx$ je podslovo}
\begin{cases}
    0^m 1^m:\quad 
      \begin{cases}
        v = 0^i,\quad x = 0^j,\quad i+j \leq 1, uv^2wx^2y \rightarrow 0^{m+i+j}1^m2^m \not\in L, \\[1ex]
        v = 0^i,\quad x = 1^j,\quad i+j \leq 1, uv^2wx^2y \rightarrow 0^{m+i}1^{m+j}2^m \not\in L, \\[1ex]
        v = 1^i,\quad x = 1^j,\quad i+j \leq 1, uv^2wx^2y \rightarrow 0^{m}1^{m+i+j}2^m \not\in L.
      \end{cases}
    \\[2ex]
    1^m 2^m:\quad 
      \begin{cases}
        v = 1^i,\quad x = 1^j,\quad i+j \leq 1, uv^2wx^2y \rightarrow 0^{m}1^{m+i+j}2^m \not\in L, \\[1ex]
        v = 1^i,\quad x = 2^j,\quad i+j \leq 1, uv^2wx^2y \rightarrow 0^{m}1^{m+i}2^{m+j} \not\in L, \\[1ex]
        v = 2^i,\quad x = 2^j,\quad i+j \leq 1, uv^2wx^2y \rightarrow 0^{m}1^{m}2^{m+i+j} \not\in L.
      \end{cases}
\end{cases}
\]
Tedy $L$ není bezkontextový.