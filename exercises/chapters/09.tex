\section{Deváté cvičení}
\subsection{Příklad}\noindent
Navrhněte bezkontextové gramatiky generující následující jazyky
\begin{enumerate}[a), noitemsep]
    \item $L_1 = \{0^{m+n} 1^n 0^m \mid 0 \leq n,m\}$.
    \item $L_2 = \{0^i 1^j \mid 0 \leq i < j\}$.
\end{enumerate}
Zdůvodněte, proč gramatika $\mathcal{G}$ jazyk $L$ generuje.

\begin{enumerate}[a)]
    \item 
\[
    \hspace*{-105mm}
    \begin{array}{l l}
        P: & S \rightarrow 0S0 \mid A \\
        & A \rightarrow 0A1 \mid \varepsilon\\
    \end{array}
\]

\begin{enumerate}[noitemsep]
    \item $\quad S \stackrel{S \rightarrow 0S0 (m)}{\Longrightarrow} 0^m S 0^m \stackrel{S \rightarrow A}
    {\Longrightarrow} 0^m A 0^m \stackrel{A \rightarrow 0A1(n)}{\Longrightarrow} 0^m 0^n A 1^n 0^m \stackrel
    {A \rightarrow \varepsilon}{\Longrightarrow} 0^m 1^m 0^n 1^n 0^m = 0^{m+n}1^n0^m$
    \item $\quad S \Rightarrow^* w, \, S \Rightarrow 0^i S 0^i, \, S \Rightarrow A, \, A \Rightarrow^* 0^j 
    1^j, \, 0^i 0^j 1^j 0^i \in L(\G)$
\end{enumerate}

\item 

\[
    \hspace*{-105mm}
    \begin{array}{l l}
        P: & S \rightarrow 0S1 \mid S1 \mid 1 \\
    \end{array}
\]

Důkazy mě těžce nebaví, všude jsou cca stejný. 
\end{enumerate}

\subsection{Příklad}\noindent % 9.2
Ke gramatice $\mathcal{G}$ zkonstruujte nevypouštěcí gramatiku $\mathcal{G}_1$, pro kterou 
$L(\mathcal{G}_1) = L(\mathcal{G}) - \{\varepsilon\}$. Gramatiku $\mathcal{G}_1$ zredukujte.

\[
    \begin{array}{l l}
        P: & S \rightarrow AB \mid \varepsilon \\
           & A \rightarrow aAAb \mid bS \mid CA \\
           & B \rightarrow BbA \mid CaC \mid \varepsilon\\
           & C \rightarrow aBB \mid bS \\
    \end{array}
\]
% TODO: neni zredukovana asi?? 
$V_1 = \{A \mid A \rightarrow \varepsilon \in P\}$\\
$V_1 = \{S, B\}$\\
$V_2 = V_1 \cup \{A \mid A \rightarrow \alpha \in P, \alpha \in V_1^{\star}\}$\\
$V_2 = V_1 \cup \emptyset = \{S, B\}$

\[
\hspace*{-115mm}
    \begin{array}{l l}
        \G_1: & S \rightarrow AB \mid A \\
           & A \rightarrow aAAb \mid bS \mid b \mid CA \\
           & B \rightarrow BbA \mid bA \mid CaC \\
           & C \rightarrow aBB \mid aB \mid a \mid bS \mid b \\
    \end{array}
\]

Gramatika $\G_1$ už je redukovaná. 

\section*{Redukce gramatiky}
obecně: \\
$V = \{A \mid A \in N, A \implies_{\G}^{\star} \, w, w \in \Sigma^{\star}\}$\\
$V_1 = \{A \mid A \implies^{\star} w \in P, w \in \Sigma^{\star}\}$\\
$V_2 = V_1 \cup \{A \mid A \rightarrow \alpha \in P, \alpha \in (\Sigma \cup V_1)^{\star}\}$\\
$U = \{A \mid A \in V, \exists \alpha, \beta\in (V \cup \Sigma)^{\star} \text{ tak, že } S 
\implies^{\star}_{\G} \quad \alpha A \beta\}$

Jazyk není prázdný právě tehdy, kdy $S \in V$. 

\subsection{Příklad - polopaticky vysvětlená redukce}

Zredukujte gramatiku $\G$, která je dána pravidly 
\[
    \begin{array}{l l}
        P: & S \rightarrow SA \mid SB \mid \varepsilon \\
           & A \rightarrow bSA \mid baS \\
           & B \rightarrow aB \mid Ba \mid DA\\
           & C \rightarrow aCB \mid bA \\
           & D \rightarrow AB \\
    \end{array}
\]

redukce tldr:\\
$V_1\dots$ to, co se promítne na $\varepsilon$ nebo na terminály \\ 
$V_2\dots$ to, co se promítne na terminály a na to, co už je ve $V_1$ \\ 
$U_0\dots$ $\{S\}$ \\
$U_1\dots$ neterminály, do kterých se dostanu z $S$, pak z $U_1$ atd. 

$V_1 = \{S\}$\\ 
$V_2 = V_1 \cup \{A\} = \{S, A\}$\\ 
$V_3 = V_2 \cup \{C\} = \{S, A, C\}$\\ 
$V_4 = V_3 \cup \emptyset = \{S, A, C\}$ 

a tady v tom nechám jenom neterminály z $V$, a z pravé strany vyškrtám pravidla obsahující neterminály $\notin V$. 

\[
    \begin{array}{l l}
        P: & S \rightarrow SA \mid \varepsilon \\
           & A \rightarrow bSA \mid baS \\
           & C \rightarrow \mid bA \\
    \end{array}
\]

sem přidávám neterminály, do kterých se dostanu z počátečního stavu $S$, pak ze stavů v odpovídajícím $U_i$. 

$U_0 = \{S\}$\\ 
$U_1 = U_0 \cup \{A\} = \{S, A\}$ \\
$U_2 = U_1 \cup \emptyset = \{S, A\}$. 

a ponechám jen pravidla, která nám zbyla v $U$. 

\[
    \begin{array}{l l}
        P: & S \rightarrow SA \mid \varepsilon \\
           & A \rightarrow bSA \mid baS \\
    \end{array}
\]

\subsection{Příklad} % 9.4

Rozhodněte, zda gramatika $\G$ generuje aspoň jedno slovo, tj. zda $L(\G) \neq \emptyset$, kde $\G$ je dána pravidly 

\[
\begin{array}{l l}
    P: & S \rightarrow aS \mid AB \mid CD \\
       & A \rightarrow aDb \mid AD \mid BC \\
       & B \rightarrow bSb \mid BB \\
       & C \rightarrow BA \mid ASb \\
       & D \rightarrow ABCD \mid \varepsilon \\
\end{array}
\]

$V_1 = \{D\}$\\ 
$V_2 = V_1 \cup \{A\} = \{D, A\}$\\ 
$V_3 = V_2 \cup \emptyset = \{D, A\} = V$

$S \notin V \quad \rightarrow \quad L(\G) = \emptyset$ 

\section*{Chomského normální tvar}

Chomského normální tvar: Všechna pravidla na pravé straně mají buď přesně 2 neterminály nebo přesně 1 terminál. 

\subsection{Příklad} % 9.5 

Je dána gramatika $\G = (N, \Sigma, S, P)$, kde $N = \{S, A, B\}$, $\Sigma = \{0, 1\}$ a P je 
\[
\begin{array}{l l}
    P: & S \rightarrow A \mid 0SA \mid \varepsilon \\
       & A \rightarrow 1A \mid B1 \mid 1 \\
       & B \rightarrow 0B \mid 0 \\
\end{array}
\]

Převeďte $\G$ do Chomského normálního tvaru. 

1. Uděláme z toho nevypouštěcí gramatiku. 

$V = \{S\}$\\ 
\[
\hspace*{-125mm}
\begin{array}{l l}
    P: & S \rightarrow A \mid 0SA \mid 0A \\
       & A \rightarrow 1A \mid B1 \mid 1 \\
       & B \rightarrow 0B \mid 0 \\
\end{array}
\]

2. Zbavíme se toho, kdy 1 neterminál přepisujeme na 1 neterminál. 

tady: $S \rightarrow A$, máme zde $A \rightarrow 1A \mid B1 \mid 1$ 

\[
\hspace*{-105mm}
\begin{array}{l l}
    P: & S \rightarrow 1A \mid B1 \mid 1 \mid 0SA \mid 0A \\
       & A \rightarrow 1A \mid B1 \mid 1 \\
       & B \rightarrow 0B \mid 0 \\
\end{array}
\]

vytvořím pomocná pravidla pro terminály, které "nezůstaly samy" 

\[
\hspace*{-95mm}
\begin{array}{l l}
    P: & S \rightarrow X_1A \mid BX_1 \mid 1 \mid X_0SA \mid X_0A \\
       & A \rightarrow X_1A \mid BX_1 \mid 1 \\
       & B \rightarrow X_0B \mid 0 \\
       & X_0 \rightarrow 0\\
       & X_1 \rightarrow 1\\
\end{array}
\]

3. Zbavím se dlouhých slov ($\geq 3$) (opět vytvořím pomocná pravidla). 
\[
\hspace*{-95mm}
\begin{array}{l l}
    P: & S \rightarrow X_1A \mid BX_1 \mid 1 \mid X_0Y \mid X_0A \\
       & Y \rightarrow SA \\
       & A \rightarrow X_1A \mid BX_1 \mid 1 \\
       & B \rightarrow X_0B \mid 0 \\
       & X_0 \rightarrow 0\\
       & X_1 \rightarrow 1\\
\end{array}
\]

\subsection{Příklad} % 9.6 

Je dán derivační strom v bezkontextové gramatice: 

% TODO: udelate ten strom symetrictejsi, tady i nahore 
\begin{center}
    \begin{forest}
    for tree={
        grow=south,                 % Tree grows downward
        edge={->},                   % Draw edges as lines
        align=center,               % Center align text inside nodes
        parent anchor=south,        % Parent anchor point
        child anchor=north        % Child anchor point
    }
    [$S$
        [$A$
            [$A$
                [$a$]
                [$S$
                    [$\varepsilon$]
                ]
                [$b$]
            ]
            [$A$
                [$\varepsilon$]
            ]
        ]
        [$a$]
        [$B$, text=red
            [$S$
                [$A$
                    [$\varepsilon$]
                ]
                [$a$]
                [$B$, text=red
                    [$a$]
                    [$c$]
                ]
            ]
            [$A$
                [$S$
                    [$\varepsilon$]
                ]
                [$B$
                    [$a$]
                    [$c$]
                ]
            ]
        ]
    ]
    \end{forest}
    \end{center}
    

    
\begin{enumerate}[label=\alph*), noitemsep]
    \item Napiště pravidla minimální CF gramatiky, ve které je to derivační strom. 
    \item Napiště levou derivaci odpovídající tomuto derivačnímu stromu. 
    \item Rozlože výsledek derivačního stromu $w$ na pět částí $w = w_1 w_2 w_3 w_4 w_5$ tak, že $w_2 w_4 
    \neq \varepsilon$ a slovo $w_1 w_2^2 w_3 w_4^2 w_5$ je také generované gramatikou z bodu a). 
    \item Rozhodněte, zda je gramatika víceznačná. 
\end{enumerate}
    
a) \vspace*{-4mm}\[
\hspace*{-95mm}
\begin{array}{l l}
    P: & S \rightarrow AaB \mid \varepsilon \\
       & A \rightarrow AA \mid SB \mid aSb \mid \varepsilon \\
       & B \rightarrow SA \mid ac \\
\end{array}
\]

b) 
$\quad S \stackrel{S \rightarrow A a B}{\Longrightarrow} A a B \stackrel{A \rightarrow A A}{\Longrightarrow} 
A A a B \stackrel{A \rightarrow a S b}{\Longrightarrow} a S b A a B \stackrel{S \rightarrow \varepsilon}
{\Longrightarrow} a b A a B \stackrel{A \rightarrow \varepsilon}{\Longrightarrow} a b a B \stackrel
{B \rightarrow SA}{\Longrightarrow} a b a S A \stackrel{A\rightarrow Aab}{\Longrightarrow} abaAaBA \stackrel
{A\rightarrow \varepsilon}{\Longrightarrow} a b a aBA \stackrel{B \rightarrow ac}{\Longrightarrow} abaaacA 
\stackrel{A \rightarrow SB}{\Longrightarrow} abaaacSB \stackrel{S \rightarrow \varepsilon}{\Longrightarrow} 
abaaacB \stackrel{B \rightarrow ac}{\Longrightarrow} abaaacac $.

\vspace*{1mm}
c) basically pumping lemma pro gramatiky, snad by to v testech ani zkoušce chtít neměla. \\
$w_1 = aba$, $w_2 = a$, $w_3 = ac$, $w_4 = ac$, $w5 = \varepsilon$. 

\begin{minipage}{0.7\textwidth}
    
\begin{center}
        
    \begin{forest}
        for tree={
            grow=south,                 % Tree grows downward
            edge={->},                  % Draw edges as arrows
            align=center,               % Center the text inside nodes
        }
        [$S$
        [$A$
            [$A$
                [$a$]
                [$S$
                    [$\varepsilon$]
                ]
                [$b$]
            ]
            [$A$
                [$\varepsilon$]
            ]
        ]
        [$a$]
        [$B$, text=red
            [$S$
                [$A$
                    [$\varepsilon$]
                ]
                [$a$]
                [$B$, text=red
                    [$S$
                        [$A$
                            [$\varepsilon$]
                        ]
                        [$a$]
                        [$B$
                            [$a$]
                            [$c$]
                        ]
                    ]
                    [$A$
                        [$A$
                            [$\varepsilon$]
                        ]
                        [$B$
                            [$a$]
                            [$c$]
                        ]
                    ]
                ]
            ]
            [$A$
                [$S$
                    [$\varepsilon$]
                ]
                [$B$
                    [$a$]
                    [$c$]
                ]
            ]
        ]
    ]
    \end{forest}    \end{center}
    
\end{minipage}
\begin{minipage}{0.3\textwidth}
    
    
    (jak toho docílím: jdu odspoda stromu a najdu dva stejné neterminály (různě v grafu) a v podstatě ten strom 
    nafouknu (zkopíruju nějaký podstrom z vyššího do nižšího neterminálu, v ukázce červený node $B$ v nejvýš 
    vpravo kopíruju do červeného $B$čka vlevo od něj).)
\end{minipage}

\vspace{7mm}

d) je víceznačná, přepisujeme $A \rightarrow AA$. 


\subsection{Příklad}  % samostatna prace 6 

Navrhněte bezkontextovou gramatiku generující následující jazyk $L = \{a^nb^ma^n \mid m, n \geq 0\}$. Zdůvodněte, 
proč zkonstruovaná gramatika jazyk $L$ generuje.


\[
    \hspace*{-125mm}
    \begin{array}{l l}
        P: & S \rightarrow aSa \mid b \mid \varepsilon
    \end{array}
\]

(1) $\quad    S \stackrel{S \rightarrow aSa (n)}{\Longrightarrow} a^nSa^n \stackrel{S \rightarrow Sb (m)}
{\Longrightarrow} a^nSb^ma^n\stackrel{S \rightarrow \varepsilon}{\Longrightarrow} a^n b^m a^n$
% \]

(2) \quad Když $w \in L(\mathcal{G})$, tak $w = a^nb^ma^n$. $S \implies \hspace*{-2.5mm}^\star \hspace*{1mm} w$, 
$S \implies \hspace*{-2.5mm}^\star \hspace*{1mm} a^lSa^l$, $S \implies \hspace*{-2.5mm}^\star \hspace*{1mm}  b^k$, 
$a^lb^ka^l \in L(\mathcal{G})$.


\subsection{Příklad} %9.10, samostatna prace 6 6 

Zredukujte gramatiku $\mathcal{G}$, kter je dána pravidly: 
\[
% \hspace*{-85mm}
\begin{array}{l l}
    \mathcal{G}: & S \rightarrow aA \mid bB \mid aSa \mid bSb \mid \varepsilon \\
    & A \rightarrow bCD \mid Dba \\
    & B \rightarrow Bb \mid AC \\
    & C \rightarrow aA \mid a \\
    & D \rightarrow DE \\ 
    & E \rightarrow \varepsilon
\end{array}
\]

(1)

% hned v prvnim kroku dostanu terminalni z neterminalniho: \\
$V_1 = \{A \mid A \rightarrow ^\star w \in P, w\in \Sigma ^ \star\}$\\
$V_1 = \{S, C, E\}$

% ty neterminaly, kde mam pravidlo, a $\alpha$ obsahuje jen neterminaly z $V_1$\\
$V_2 = V_1 \cup \{A \mid A \rightarrow \alpha \in P, \alpha \in (\Sigma \cup V_1)^\star\}$\\
$V_2 = V_1 \cup {\emptyset} = \{S, C, E\} \cup {\emptyset} = \{S, C, E\}$

\[
    \hspace*{-107mm}
    \begin{array}{l l}
        \mathcal{G'}: & S \rightarrow aSa \mid bSb \mid \varepsilon \\
        & C \rightarrow a \\
        & E \rightarrow \varepsilon
    \end{array}
\]
\vspace*{3mm}

(2) 


$U = \{X \in V \mid \exists \alpha, \beta \in (V \cup \Sigma)^\star, S \implies \alpha X \beta \}$

$U_0 = \{S\}$\\
$U_1 = U_0 \cup \{X \mid X \text{ se vyskytuje v }\alpha \text{ nebo } \beta \text{ na pravé straně pro pravidlo } 
Y \rightarrow \alpha \in P, Y \in U_0\}$, $S \implies ^\star S$\\
$U_1 = U_0 \cup {\emptyset}$

\[
    \hspace*{-107mm}
    \begin{array}{l l}
        \mathcal{G''}: & S \rightarrow  aSa \mid bSb \mid \varepsilon \\
    \end{array}
\]