\section{Dvanácté cvičení}

\subsection{Tvorba zásobníkového automatu podle jazyka}
Je dán jazyk $L$ nad abecedou $\Sigma = \{a,b\}$. Sestrojte zásobníkové automaty $A,B$ tak, že $L = N(A)$ a $L = L(B)$ 
(tj. $A$ přijímá $L$ prázdným zásobníkem, $B$ přijímá $L$ koncovým stavem), kde
\[L = \{(ab)^i b^j a^{j-i} \mid 0 < i < j\}\text{.}\]

$\implies L= \{(ab)^i b^{i+k} a^k \mid i>0, k>0\} \implies N(A)=\underbrace{\{(ab)^i b^i \mid k>0\}}_X,
L(B)=\underbrace{\{b^k a^k \mid k>0\}}_Y$.

Dva způsoby řešení:

a) Přímo.

\begin{tikzpicture}
    \node[state, initial] (0) {$q_0$};
    \node[state, right of=0] (1) {$q_1$};
    \node[state, right of=1] (2) {$q_2$};
    \node[state, right of=2] (3) {$q_3$};
    \node[state, right of=3] (4) {$q_4$};
    \node[state, right of=4] (5) {$q_5$};
    \node[state, accepting, below of=5] (f) {$q_f$};

    \draw
        (0) edge[bend left, above] node{$a, Z_0 / Z_0$} (1)
        (1) edge[bend left, above] node[align=center] {$b, Z_0 / X Z_0$ \\ $b, X / XX$} (2)
        (2) edge[bend left, above] node{$b, X / \varepsilon$} (3)
        (2) edge[bend left, below] node{$a, X / XX$} (1)
        (3) edge[loop above] node{$b, X / \varepsilon$} (3)
        (3) edge[bend left, above] node{$b, Z_0 / Y Z_0$} (4)
        (4) edge[loop above] node{$b, Y / YY$} (4)
        (4) edge[bend left, above] node{$a, Y / \varepsilon$} (5)
        (5) edge[loop above] node{$a, Y / \varepsilon$} (5)
        (5) edge[bend left, left] node{$\varepsilon, Z_0 / \varepsilon$} (f)

        ;
\end{tikzpicture}

b) Přes gramatiku.
\begin{multicols}{2}
    \begin{flalign*}
        \G: S &\rightarrow AB &\\
        A &\rightarrow abAb \mid abb &\\
        B &\rightarrow bBa \mid ba &
    \end{flalign*}
    \columnbreak

    \noindent
    1. $L \subseteq L(G)$\\
    $S \xRightarrow{S \rightarrow AB} AB \xRightarrow{A \rightarrow abAb}^{(i-1)} (ab)^{i-1} A b^{i-1} B \xRightarrow{A \rightarrow abb}
    (ab)^i b^i B \xRightarrow{B \rightarrow bBa}^{(k-1)} (ab)^i b^i b^{k-1} B a^{k-1} \xRightarrow{B \rightarrow ba} (ab)^i b^i b^k a^k$.

    \noindent
    2. $L(G) \subseteq L$\\
    $S \rightarrow^\star w$\\
    $S \rightarrow AB$\\
    $A \xRightarrow{A \rightarrow abAb}^j (ab)^j A b^j \xRightarrow{A \rightarrow abb} (ab)^j a b b^j$
    % TODO: complete proof 2.

\end{multicols}

\subsection{Tvorba zásobníkového automatu podle jazyka}
Je dán jazyk $L$ nad abecedou $\Sigma = \{a,b\}$. Sestrojte zásobníkové automaty $A,B$ tak, že $L = N(A)$ a $L = N(B)$ 
(tj. $A$ přijímá $L$ prázdným zásobníkem, $B$ přijímá $L$ koncovým stavem), kde
\[L = \{w \mid w \text{ začíná a končí symbolem } 1 \text{ a obsahuje o dvě } 1 \text{ více než } 0\}\text{.}\]
\begin{multicols}{2}
    $L = \{1u1 \mid |u|_0 = |u|_1\}, i = |u|_0$

    \begin{tikzpicture}
        \node[state, initial] (0) {$q_0$};
        \node[state, right of=0] (1) {$q_1$};
        \node[state, accepting, right of=1] (f) {$q_f$};
    
        \draw
            (0) edge[bend left, above] node{$1, Z_0 / Z_0$} (1)
            (1) edge[loop above] node[align=center]{$0, Z_0 / 0 Z_0$\\$1, Z_0 / 1 Z_0$} (1)
            (1) edge[loop below] node[align=center]{$0,0 / 00$\\ $1,1 / 11$\\ $0,1 / \varepsilon$\\ $1,0 / \varepsilon$} (1)
            (1) edge[bend left, above] node{$1, Z_0 / \varepsilon$} (f)
    
            ;
    \end{tikzpicture}

    \columnbreak

    1) $L \subseteq N(A)$\\
    $(q_0, 1u1, Z_0) \vdash (q_1, u1, Z_0) \vdash^\star (q_1, 1, Z_0) \vdash (q_f, \varepsilon, \varepsilon)$.

    2) $L(B) \subseteq L$\\

    $.$
    % TODO: complete proof 2.

\end{multicols}

\subsection{Důkaz bezkontextovosti jazyka}
Je dán jazyk $L = \{0^n 1^m; 0 \leq n \leq m \leq 2n\}$. Rozhodněte, zda jazyk $L$ je bezkontextový.\\
V případě, že je bezkontextový, najděte buď bezkontextovou gramatiku, která ho generuje, nebo zásobníkový automat, který 
ho přijímá.\\
V případě, že není bezkontextový, tvrzení dokažte.
\begin{multicols}{2}
    \begin{flalign*}
        \G_1: S \rightarrow 0SA1 \mid \varepsilon &\\
        A \rightarrow 1 \mid \varepsilon
    \end{flalign*}
    \columnbreak

    \begin{flalign*}
        \G_2: S \rightarrow 0S11 \mid 0S1 \mid \varepsilon &\\
        \\ \\
    \end{flalign*}
\end{multicols}

Důkaz $\G_2$.

1) $L \subseteq L(\G_2)$\\
$0 < n \leq m \leq 2n \rightarrow k=2n-m \geq 0$\\
$S \xRightarrow{S \rightarrow 0S1}^{(k)} 0^k S 1^k \xRightarrow{S \rightarrow 0S11}^{(n-k)} 0^k 0^{n-1} S 1^{n-1} 1^{n-1} 1^k
\xRightarrow{S \rightarrow \varepsilon} 0^n 1^{2n-k} = 0^n 1^{2n - 2n + m} = 0^n 1^m$.

2) $L(\G_2) \subseteq L$\\
% TODO: complete proof 2.