\section{Dvanácté cvičení}

\subsection{Příklad}
Je dán jazyk $L$ nad abecedou $\Sigma = \{a,b\}$. Sestrojte zásobníkové automaty $A,B$ tak, že $L = N(A)$ a $L = N(B)$ (tj. $A$ přijímá $L$ prázdným zásobníkem, $B$ přijímá $L$ koncovým stavem), kde
\[L = \{(ab)^i b^j a^{j-i} \mid 0 < i < j\}\text{.}\]

\subsection{Příklad}
Je dán jazyk $L$ nad abecedou $\Sigma = \{a,b\}$. Sestrojte zásobníkové automaty $A,B$ tak, že $L = N(A)$ a $L = N(B)$ (tj. $A$ přijímá $L$ prázdným zásobníkem, $B$ přijímá $L$ koncovým stavem), kde
\[L = \{w \mid w \text{ začíná a končí symbolem } 1 \text{ a obsahuje o dvě } 1 \text{ více než } 0\}\text{.}\]

\subsection{Příklad}
Je dán jazyk $L = \{0^n 1^m; 0 \leq n \leq m \leq 2n\}$. Rozhodněte, zda jazyk $L$ je bezkontextový.\\
V případě, že je bezkontextový, najděte buď bezkontextovou gramatiku, která ho generuje, nebo zásobníkový automat, který ho přijímá.\\
V případě, že není bezkontextový, tvrzení dokažte.