\section{Dvanácté cvičení}

\subsection{Tvorba zásobníkového automatu podle jazyka}
Je dán jazyk $L$ nad abecedou $\Sigma = \bc{a,b}$. Sestrojte zásobníkové automaty $A,B$ tak, že $L = N(A)$ a $L = L(B)$ 
(tj. $A$ přijímá $L$ prázdným zásobníkem, $B$ přijímá $L$ koncovým stavem), kde
\[L = \bc{(ab)^i b^j a^{j-i} \mid 0 < i < j}\text{.}\]

$\implies L= \bc{(ab)^i b^{i+k} a^k \mid i>0, k>0} \implies N(A)=\underbrace{\bc{(ab)^i b^i \mid i>0}}_X,
L(B)=\underbrace{\bc{b^k a^k \mid k>0}}_Y$.

Dva způsoby řešení:

a) Přímo.

\begin{tikzpicture}
    \node[state, initial] (0) {$q_0$};
    \node[state, right of=0] (1) {$q_1$};
    \node[state, right of=1] (2) {$q_2$};
    \node[state, right of=2] (3) {$q_3$};
    \node[state, right of=3] (4) {$q_4$};
    \node[state, right of=4] (5) {$q_5$};
    \node[state, accepting, below of=5] (f) {$q_f$};

    \draw
        (0) edge[bend left, above] node{$a, Z_0 / Z_0$} (1)
        (1) edge[bend left, above] node[align=center] {$b, Z_0 / X Z_0$ \\ $b, X / XX$} (2)
        (2) edge[bend left, above] node{$b, X / \varepsilon$} (3)
        (2) edge[bend left, below] node{$a, X / X$} (1)
        (3) edge[loop above] node{$b, X / \varepsilon$} (3)
        (3) edge[bend left, above] node{$b, Z_0 / Y Z_0$} (4)
        (4) edge[loop above] node{$b, Y / YY$} (4)
        (4) edge[bend left, above] node{$a, Y / \varepsilon$} (5)
        (5) edge[loop above] node{$a, Y / \varepsilon$} (5)
        (5) edge[bend left, left] node{$\varepsilon, Z_0 / \varepsilon$} (f)
        ;
\end{tikzpicture}

b) Přes gramatiku.
\begin{multicols}{3}
    \raggedcolumns
    \begin{flalign*}
        \G: S &\rightarrow AB &\\
        A &\rightarrow abAb \mid abb &\\
        B &\rightarrow bBa \mid ba &
    \end{flalign*}

\columnbreak

    Prázdným zásobníkem: 
    \[
    \begin{array}{l l}
        & \delta(q, \varepsilon, S) = \bc{(q, AB)} \\
        & \delta(q, \varepsilon, A) = \bc{(q, abAb), (q, abb)} \\
        & \delta(q, \varepsilon, B) = \bc{(q, bBa), (q, ba)} \\
        & \delta(q, a, a) = \bc{(q, \varepsilon)} \\
        & \delta(q, b, b) = \bc{(q, \varepsilon)} \\ 
    \end{array}
    \]
    

    Koncovým stavem: 
    \[
    \begin{array}{l l}
        & \delta(q_0, \varepsilon, Z_0) = \bc{(q, SZ_0)} \\
        & \delta(q, \varepsilon, S) = \bc{(q, AB)} \\
        & \delta(q, \varepsilon, A) = \bc{(q, abAb), (q, abb)} \\
        & \delta(q, \varepsilon, B) = \bc{(q, bBa), (q, ba)} \\
        & \delta(q, a, a) = \bc{(q, \varepsilon)} \\
        & \delta(q, b, b) = \bc{(q, \varepsilon)} \\ 
        & \delta(q, \varepsilon, Z_0) = \bc{(q_f, \varepsilon)} \\ 
    \end{array}
    \]
\end{multicols}

\noindent
1. $L \subseteq L(G)$\\
$S \xRightarrow{S \rightarrow AB} AB \xRightarrow{A \rightarrow abAb}^{(i-1)} (ab)^{i-1} A b^{i-1} B \xRightarrow{A \rightarrow abb}
(ab)^i b^i B \xRightarrow{B \rightarrow bBa}^{(k-1)} (ab)^i b^i b^{k-1} B a^{k-1} \xRightarrow{B \rightarrow ba} (ab)^i b^i b^k a^k$.

\noindent
2. $L(G) \subseteq L$\\
Nechť $w$ je libovolné slovo, které lze odvodit ze startovacího symbolu $S$. Pak postup derivace je vždy:
\begin{itemize}
    \item Musíme použít pravidlo $S \rightarrow AB$
    \item Nechť derivace z $A$ má právě $i > 0$ aplikací pravidla $A \rightarrow abAb$ (s konečným použitím pravidla 
    $A \rightarrow abb$). Tedy: $A \Rightarrow (ab)^i b^i$.
    \item Nechť derivace z $B$ má právě $k > 0$ aplikací pravidla $B \rightarrow bBa$ (s konečným použitím pravidla
    $B \rightarrow ba$). Tedy: $B \Rightarrow b^k a^k$.
\end{itemize}
Spojením (protože jsme museli použít pravidlo $S \rightarrow AB$) dostáváme $S \Rightarrow (ab)^i b^i b^k a^k \equiv 
(ab)^i b^{i+k} a^k$, $i > 0, k > 0 \implies 0 < i < i+k$. Což přesně odpovídá definici jazyka $L$, tedy $L(G) \subseteq L$.

\subsection{Tvorba zásobníkového automatu podle jazyka}
Je dán jazyk $L$ nad abecedou $\Sigma = \bc{a,b}$. Sestrojte zásobníkové automaty $A,B$ tak, že $L = N(A)$ a $L = L(B)$ 
(tj. $A$ přijímá $L$ prázdným zásobníkem, $B$ přijímá $L$ koncovým stavem), kde
\[L = \bc{w \mid w \text{ začíná a končí symbolem } 1 \text{ a obsahuje o dvě } 1 \text{ více než } 0}\text{.}\]

$L = \bc{1u1 \mid |u|_0 = |u|_1}, i = |u|_0$

\begin{tikzpicture}
    \node[state, initial] (0) {$q_0$};
    \node[state, right of=0] (1) {$q_1$};
    \node[state, accepting, right of=1] (f) {$q_f$};
    
    \draw
        (0) edge[bend left, above] node{$1, Z_0 / Z_0$} (1)
        (1) edge[loop above] node[align=center]{$0, Z_0 / 0 Z_0$\\$1, Z_0 / 1 Z_0$} (1)
        (1) edge[loop below] node[align=center]{$0,0 / 00$\\ $1,1 / 11$\\ $0,1 / \varepsilon$\\ $1,0 / \varepsilon$} (1)
        (1) edge[bend left, above] node{$1, Z_0 / \varepsilon$} (f)
        ;
\end{tikzpicture}

\textbf{1)} $L \subseteq N(A)$

$(q_0, 1u1, Z_0) \vdash (q_1, u1, Z_0) \vdash^\star (q_1, 1, Z_0) \vdash (q_f, \varepsilon, \varepsilon)$.

\textbf{2)} $L(B) \subseteq L$

Musíme ukázat, že pokud zásobníkový automat $B$ přijme slovo $w$, pak $w \in L$. To znamená, že $w$ začíná a končí 
symbolem $1$ a obsahuje o dvě více symbolů $1$ než $0$.

Vlastnosti přechodů:
\begin{enumerate}[noitemsep]
    \item Automat přechází z počátečního stavu $q_0$ do $q_1$, pokud první symbol je $1$. Tuto $1$ nezapočítáme.
    \item Ve stavu $q_1$ se zásobníkem automat udržuje informaci o rozdílu mezi počtem symbolů $1$ a $0$:
    \begin{itemize}[noitemsep]
        \item Při čtení symbolů $0$ a $1$ přidává, respektive odstraní (pokud existuje) odpovídající značku do zásobníku,
        což udržuje informaci o tom, zda je stejný počet obou znaků.
    \end{itemize}
    \item Pokud zpracování skončí symbolem $1$ a zásobník je prázdný, automat přejde do koncového stavu $q_f$, což 
    zajišťuje, že rozdíl mezi počtem symbolů $1$ a $0$ je přesně dvě.
\end{enumerate}
Každé slovo přijaté automatem $B$ splňuje podmínky:
\begin{itemize}[noitemsep]
    \item Začíná symbolem $1$ (nutný přechod z $q_0$ do $q_1$).
    \item Končí symbolem $1$ (nutný přechod z $q_1$ do $q_f$).
    \item Obsahuje přesně o dvě více symbolů $1$ než $0$ (zásobník na konci zajišťuje prázdný stav).
\end{itemize}
Z toho plyne, že $L(B) \subseteq L$.

\subsection{Tvorba zásobníkového automatu podle jazyka}
Je dán jazyk $L$. Sestrojte zásobníkové automaty $A$, $B$ tak, že $L = N(A)$ a $L = L(B)$ (tj. $A$ přijímá $L$ prázdným 
zásobníkem, $B$ přijímá $L$ koncovým stavem), kde $$L = \bc{a^i b^i c^{i+j}}$$
Ukažte práci nad slovem $w = abcc$. 

Konstrukce pomocí gramatiky: 

$\medskip S \to aSc \mid A$\\
$\medskip A \to bAc \mid \varepsilon$

\begin{multicols}{2}
    
    Prázdným zásobníkem: 
    \[
    \begin{array}{l l}
        & \delta(q, \varepsilon, S) = \bc{(q, aSc), (q, A)} \\
        & \delta(q, \varepsilon, A) = \bc{(q, bAc), (q, \varepsilon)} \\
        & \delta(q, a, a) = \bc{(q, \varepsilon)} \\
        & \delta(q, b, b) = \bc{(q, \varepsilon)} \\ 
        & \delta(q, c, c) = \bc{(q, \varepsilon)} \\ 
    \end{array}
    \]

    Práce nad slovem $w = abcc$:

    $
    {(q, abbc, S) \vdash (q, abcc, aSc) \vdash (q, bcc, Sc) \vdash}\\
    {(q, bcc, Ac) \vdash (q, bcc, bAcc) \vdash (q, cc, Acc) \vdash}\\ 
    {(q, cc, cc) \vdash (q, c, c) \vdash(q, \varepsilon, \varepsilon)}
    $

\columnbreak

    Koncovým stavem: 
    \[
    \begin{array}{l l}
        & \delta(q_0, \varepsilon, Z_0) = \bc{(q, SZ_0)} \\
        & \delta(q, \varepsilon, S) = \bc{(q, AB)} \\
        & \delta(q, \varepsilon, A) = \bc{(q, abAb), (q, abb)} \\
        & \delta(q, \varepsilon, B) = \bc{(q, bBa), (q, ba)} \\
        & \delta(q, \varepsilon, a) = \bc{(q, \varepsilon)} \\
        & \delta(q, \varepsilon, b) = \bc{(q, \varepsilon)} \\ 
        & \delta(q, \varepsilon, Z_0) = \bc{(q_f, \varepsilon)} \\ 
    \end{array}
    \]

    Práce nad slovem $w = abcc$:

    $
    {(q_0, abc, Z_0) \vdash (q, abbc, SZ_0) \vdash (q, abcc, aScZ_0) \vdash}\\
    {(q, bcc, SZ_0c) \vdash (q, bcc, AcZ_0) \vdash (q, bcc, bAccZ_0) \vdash}\\
    {(q, cc, AccZ_0) \vdash ( q, cc, ccZ_0) \vdash (q, c, cZ_0) \vdash}\\
    (q, \varepsilon, Z_0) \vdash (q_f, \varepsilon, \varepsilon)
    $
\end{multicols}

\subsection{Důkaz bezkontextovosti jazyka}
Je dán jazyk $L = \bc{0^n 1^m; 0 \leq n \leq m \leq 2n}$. Rozhodněte, zda jazyk $L$ je bezkontextový.\\
V případě, že je bezkontextový, najděte buď bezkontextovou gramatiku, která ho generuje, nebo zásobníkový automat, který 
ho přijímá.\\
V případě, že není bezkontextový, tvrzení dokažte.

Například $\G$ s pravidly $P$:
\begin{flalign*}
    S \rightarrow 0S11 \mid 0S1 \mid \varepsilon & \\
\end{flalign*}
Důkaz.

\textbf{1)} $L \subseteq L(\G)$ $-$ $\G$ generuje celé $L$
\begin{itemize}[leftmargin=*]
    \item $0 < n \leq m \leq 2n \rightarrow k=2n-m \geq 0$:\\
    $S \xRightarrow{S \rightarrow 0S1}^{(k)} 0^k S 1^k \xRightarrow{S \rightarrow 0S11}^{(n-k)} 0^k 0^{n-1} S 1^{n-1} 
    1^{n-1} 1^k \xRightarrow{S \rightarrow \varepsilon} 0^n 1^{2n-k} = 0^n 1^{2n - 2n + m} = 0^n 1^m$.
    \item $0 = n = m$: $ 0^0 1^0 = \varepsilon \in L$.
\end{itemize}
\textbf{2)} $L(\G) \subseteq L$ $-$ $\G$ negeneruje nic navíc

Nechť $w \in L(\G)$, kde $w = 0^n 1^m$. Sledujeme pravidla gramatiky $\G$:

(0) základní krok: Pokud $S \rightarrow \varepsilon$, pak $w = \varepsilon = 0^0 1^0$. Zřejmě $\varepsilon \in L$.

(1) indukční krok: 
\begin{itemize}
    \item Pokud $S \rightarrow 0S1$, přidáváme jeden symbol $0$ na začátek a jeden symbol $1$ na konec. Po aplikaci 
    tohoto pravidla se počet $0$ zvýší o jeden a počet $1$ o jedna, tedy platí $n \leq m \leq 2n$.
    \item Pokud $S \rightarrow 0S11$, přidáváme jeden symbol $0$ na začátek a dva symboly $1$ na konec. Po aplikaci 
    tohoto pravidla se počet $0$ zvýší o jeden a počet $1$ o dva, tedy stále platí $n \leq m \leq 2n$.
\end{itemize}

\newpage
\subsection{Tvorba zásobníkového automatu podle jazyka - \ii{Z}}
Je dán jazyk $L = \bc{(01)^j 1^i 0^{i+j}; 0 \leq i, 0 < j}$ nad abecedou $\Sigma = \bc{0,1}$. 

\begin{enumerate}[label=\alph*), noitemsep]
    \item Sestrojte zásobníkový automat $A_1$, který přijímá jazyk $L$ prázdným zásobníkem. Zdůvodněte.  
    \item Sestrojte zásobníkový automat $A_2$, který přijímá jazyk $L$ koncovým stavem. Zdůvodněte. 
    \item Ukažte práci zásobníkového automatu $A_2$ nad slovem $011100$ a nad slovem $010100$. 
\end{enumerate}

a), b) 

\begin{tikzpicture}[node distance=35mm]
    \node[state, initial] (0) {$q_0$};
    \node[state, below of=0] (1) {$q_1$};
    \node[state, right of=0] (2) {$q_2$};
    \node[state, right of=2] (3) {$q_3$};
    \node[state, accepting, right of=3] (f) {$q_f$};
    
    \draw
        (0) edge[bend right, left] node[align=center]{$0, Z_0 / Z_0$\\ $0, X / X$} (1)
        (1) edge[bend right, right] node[align=center]{$1, Z_0 / XZ_0$\\ $1, X / XX$} (0)
        (0) edge[bend left, above] node[align=center]{$\eps, X / X $} (2)
        (2) edge[loop below, below] node[align=center]{$1, X / XX $} (2)
        (2) edge[bend left, above] node[align=center]{$0, X / \eps $} (3)
        (3) edge[loop below, below] node[align=center]{$0, X / \eps $} (3)
        (3) edge[bend left, above] node[align=center]{$\eps, Z_0 / \eps $} (f)
        ;
\end{tikzpicture}

Zdůvodnění: 
\begin{enumerate}[noitemsep]
    \item Chceme přečíst $(01)^j, j > 0$. Se zásobníkem na $Z_0$ musíme tedy přečíst alespoň jednu smyčku $(01)$. 
    Každý průchod smyčkou započítáme přidáním $X$ na zásobník. Po prvním průchodu smyčkou ji můžeme projít kolikrát chceme. 
    \item Chceme načítat $1^i$. Nechceme to ale už střídat s načítáním smyčky $(01)$, posuneme se tedy do dalšího 
    stavu pomocí $\eps$. 
    \item Načítáme $1^i, i \geq 0$, tedy nemusíme přečíst znak $1$ vůbec a můžeme ho přečíst libovolně-krát. Udržujeme 
    si pojem o počtu načtení pomocí přidávání $X$ na zásobník. V budoucnu nás zajímá pouze součet $i+j$, nemusíme tedy 
    zavádět nový zásobníkový znak. 
    \item Načítáme $0$. Počet nul je $0^{i+j}$, kde $j > 0$, tedy alespoň jednu $0$ načíst musíme. Chceme načíst 
    $(i+j)$ nul, což nám udržují $X$ na zásobníku, a tedy umazáváme jedno $X$. 
    \item Čteme $(i + j - 1)$-krát znak $0$. Až dojdou $X$ ze zásobníku, máme jich načtených právě $(i+j)$. 
    \item Se $Z_0$ na vrcholu zásobníku víme, že jsme již přečetli vše, co jsme potřebovali, a můžeme se přesunout do 
    koncového stavu. Slovo je přijato jak koncovým stavem, tak prázdným zásobníkem. 
\end{enumerate}

c) 

\begin{multicols}{2} 

    Práce nad slovem $011100$: 

    $
    {(q_0, 011100, Z_0) \vdash (q_1, 11100, Z_0) \vdash }\\
    {\vdash (q_0, 1100, XZ_0) \vdash (q_2, 1100, XZ_0) \vdash }\\
    {\vdash (q_2, 100, XXZ_0) \vdash (q_2, 00, XXXZ_0) \vdash }\\ 
    {\vdash (q_3, 0, XXZ_0) \vdash (q_3, \eps, XZ_0) }\\
    \to $
    nepřijato. 

    Práce nad slovem $010100$: 

    $
    {(q_0, 010100, Z_0) \vdash (q_1, 10100, Z_0) \vdash }\\
    {\vdash (q_0, 0100, XZ_0) \vdash (q_1, 100, XZ_0) \vdash }\\
    {\vdash (q_0, 00, XXZ_0) \vdash (q_2, 00, XXZ_0) \vdash }\\
    {\vdash (q_3, 0, XZ_0) \vdash (q_3, \eps, Z_0) \vdash }\\
    {\vdash (q_f, \eps, \eps)}
    $. Přijato. 

\end{multicols}

\subsection{Tvorba zásobníkového automatu podle jazyka - \ii{Z}}
Je dán jazyk $L = \bc{w \mid |w|_0 = |w|_1 + 2 \text{ a } w \text{ začíná a končí } 0}$ nad abecedou $\Sigma = \bc{0,1}$.

\begin{enumerate}[a), noitemsep]
    \item Sestrojte zásobníkový automat $A_1$, který přijímá jazyk $L$ prázdným zásobníkem. Zdůvodněte.  
    \item Sestrojte zásobníkový automat $A_2$, který přijímá jazyk $L$ koncovým stavem. Zdůvodněte. 
    \item Ukažte práci zásobníkového automatu $A_2$ nad slovem $011001$ a nad slovem $011000$. 
    \item Je jazyk $L$ přijímán deterministickým zásobníkovým automatem prázdným zásobníkem? DZA definujte.
\end{enumerate}

a), b)

\begin{tikzpicture}[node distance=35mm]
    \node[state, initial] (0) {$q_0$};
    \node[state, right of=0] (1) {$q_1$};
    \node[state, right of=1] (2) {$q_2$};
    \node[state, accepting, right of=2] (f) {$q_f$};
    
    \draw
        (0) edge[bend left, above] node{$0, Z_0 / 0 Z_0$} (1)
        (1) edge[loop below] node[align=center]{$0, Z_0 / 0 Z_0$ \\ $1, Z_0 / 1 Z_0$ \\ $0,0 / 00$ \\ $1,1 / 11$ \\ $0,1/\eps$ \\ $1,0/\eps$} (1)
        (1) edge[bend left, above] node{$0,0/\eps$} (2)
        (2) edge[bend left, above] node{$\eps, Z_0 / \eps$} (f)
        ;
\end{tikzpicture}

Zdůvodnění: 
\begin{enumerate}[noitemsep]
    \item Prvním přechodem zařídíme, že slovo nutně začíná $0$. Přidáme si informaci o přijetí na zásobník, tu následně 
    využijeme v 3. přechodu, abychom zajistili, že slovo bude mít přesně o dvě $0$ víc než $1$.
    \item Počítáme rozdíl mezi počty $0$ a $1$. Protože slovo musí mít o dvě $0$ víc a zároveň začínat a končit $0$, tak
    zbytek slova musí mít stejný počet $0$ a $1$.
    \item Jako poslední znak bude $0$, zároveň ji zkrátíme s $0$, kterou jsme si poznamenali na zásobník z 1. přechodu. 
    Což by také měl být poslední znak před $Z_0$ na zásobníku.
    \item Pokud ná zásobníku skutečně nic krom $Z_0$ nezbylo, tak vymažeme $Z_0$ a skončíme s prázdným zásobníkem a 
    přijatým slovem. Zásobník tedy přijímá prázdným zásobníkem i koncovým stavem.
\end{enumerate}

c) 

\begin{multicols}{2} 

    Práce nad slovem $011001$: 

    $
    {(q_0, 011001, Z_0) \vdash (q_1, 11001, 0Z_0) \vdash^{(2)} }\\
    {(q_1, 001, 110Z_0) \vdash^{(2)} (q_1, 1, 0Z_0)} \xmark\\
    $Konec, neúspěch.

    Práce nad slovem $011000$: 

    $
    {(q_0, 011000, Z_0) \vdash (q_1, 11000, 0Z_0) \vdash^{(2)}}\\
    {(q_1, 000, 110Z_0) \vdash^{(2)} (q_1, 0, 0Z_0) \vdash}\\
    {(q_2, \eps, Z_0) \vdash (q_f, \eps, \eps)} \checkmark\\
    $Konec, úspěch.

\end{multicols}

d) Automat je deterministický, pokud pro každou kombinaci $(q, a, X)$ (se stavem $q$, vstupním symbolem $a$ (nebo $\eps$)
a vrcholem zásobníku $X$) existuje nejvýše jeden možný přechod. Konkrétně, pokud je v konfiguraci definován přechod na 
čtený symbol, nesmí být definován žádný $\eps$-přechod, a naopak. Tato vlastnost zaručuje, že výpočet je 
jednoznačný=deterministický.

Jazyk $L$ \textbf{není} přijímán žádným deterministickým zásobníkovým automatem (DZA) pracujícím na principu prázdného zásobníku, 
jelikož by to implikovalo, že i jazyk $L$ je deterministický, což není pravda. Pro $L$ totiž při zpracování vstupu nelze 
jednoznačně určit, kdy by měl automat přestat „pushovat“ symboly (typicky 0) a začít je „popovat“ (typicky při čtení 1). 
Neexistuje žádný deterministický způsob, jak tuto hranici přesně určit, protože vstupní řetězec může mít 0 a 1 v 
libovolném uspořádání.
