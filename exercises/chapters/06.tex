\addtocontents{toc}{\protect\newpage}
\section{Šesté cvičení}

\subsection{Návrh NFA a redukovaný DFA}
Navrhněte NFA, který přijímá jazyk $L$ nad abecedou $\{a,b\}$, kde $L$ obsahuje právě všechna slova $w$ taková, že
\begin{itemize}[noitemsep]
    \item druhý znak slova $w$ je $a$,
    \item předposlední znak slova $w$ je b.
\end{itemize}
K danému NFA (není-li již DFA) sestrojte podmnožinovou konstrukcí DFA přijímající stejný jazyk. Výsledný DFA redukujte.

NFA:

\begin{tikzpicture}

    \node[state, initial] (0) {$0$};
    \node[state] (1) [right of=0] {$1$};
    \node[state, accepting] (2) [right of=1] {$2$};

    \path[->]
    (0) edge node {$b$} (1)
    (1) edge node {$a$} (2);

\end{tikzpicture}

\begin{tikzpicture}

    \node[state, initial] (3) {$3$};
    \node[state] (4) [right of=3] {$4$};
    \node[state] (5) [right of=4] {$5$};
    \node[state] (6) [right of=5] {$6$};
    \node[state, accepting] (7) [right of=6] {$7$};

    \path[->]
    (3) edge node {$a, b$} (4)
    (4) edge node {$a$} (5)
    (5) edge [loop below] node {$a, b$} ()
        edge node {$b$} (6)
    (6) edge node {$a, b$} (7);

\end{tikzpicture}

Podmnožinová konstrukce: 

\begin{tabular}{|r r|c c||c|c c||c|c c||c|c c||c|c c||c|}
    \hline
    & & $a$ & $b$ & $ \sim_0 $ & $a$ & $b$ & $\sim_1$ & $a$ & $b$ & $\sim_2$ & $a$ & $b$ & $\sim_3$ & $a$ & $b$ & $\sim_4$\\ \hline \hline
      $\to$ & $\{0,3\}$ & $\{4\}$ & $\{1,4\}$ & $O$ & $O$ & $O$ & $O$ & $O$ & $A$ & $D$ & $O$ & $A$ & $D$ & $F$ & $A$ & $D$ \\
            & $\{4\}$ & $\{5\}$ & $\emptyset$ & $O$ & $O$ & $O$ & $O$ & $O$ & $O$ & $O$ & $E$ & $O$ & $F$ & $E$ & $O$ & $F$ \\
        & $\{1,4\}$ & $\{2,5\}$ & $\emptyset$ & $O$ & $K$ & $O$ & $A$ & $B$ & $O$ & $A$ & $B$ & $O$ & $A$ & $B$ & $O$ & $A$ \\
              & $\{5\}$ & $\{5\}$ & $\{5,6\}$ & $O$ & $O$ & $O$ & $O$ & $O$ & $C$ & $E$ & $E$ & $C$ & $E$ & $E$ & $C$ & $E$ \\
    & $\emptyset$ & $\emptyset$ & $\emptyset$ & $O$ & $O$ & $O$ & $O$ & $O$ & $O$ & $O$ & $O$ & $O$ & $O$ & $O$ & $O$ & $O$ \\
      $\gets$&$\{2,5\}$ & $\{5\}$ & $\{5,6\}$ & $K$ & $O$ & $O$ & $B$ & $O$ & $C$ & $B$ & $E$ & $C$ & $B$ & $E$ & $C$ & $B$ \\
        & $\{5,6\}$ & $\{5,7\}$ & $\{5,6,7\}$ & $O$ & $K$ & $K$ & $C$ & $B$ & $K$ & $C$ & $B$ & $K$ & $C$ & $B$ & $K$ & $C$ \\
      $\gets$&$\{5,7\}$ & $\{5\}$ & $\{5,6\}$ & $K$ & $O$ & $O$ & $B$ & $O$ & $C$ & $B$ & $E$ & $C$ & $B$ & $E$ & $C$ & $B$ \\  
$\gets$&$\{5,6,7\}$ & $\{5,7\}$ & $\{5,6,7\}$ & $K$ & $K$ & $K$ & $K$ & $B$ & $K$ & $K$ & $B$ & $K$ & $K$ & $B$ & $K$ & $K$ \\
    \hline
\end{tabular}

DFA: 

\begin{tikzpicture} 

    \node[state, initial] (D) {$D$};
    \node[state] (F) [right of=D] {$F$};
    \node[state] (A) [below of=D] {$A$};
    \node[state] (O) [below of=F] {$O$};
    \node[state, accepting] (B) [right of=O] {$B$};
    \node[state] (E) [right of=F] {$E$};
    \node[state] (C) [right of=E] {$C$};
    \node[state, accepting] (K) [right of=B] {$K$};

    \path[->]
    (D) edge [bend left] node {$a$} (F)
        edge [bend right] node {$b$} (A)
    (F) edge [bend left] node {$a$} (E)
        edge [bend left] node {$b$} (O)
    (A) edge [bend right] node {$b$} (O)
        edge [bend right] node [swap] {$a$} (B)
    (O) edge [loop right] node {$a, b$} ()
    (E) edge [loop above] node {$a$} ()
        edge [bend left] node {$b$} (C)
    (C) edge [bend left] node {$a$} (B)
        edge [bend left] node {$b$} (K)
    (B) edge [bend left] node {$a$} (E)
        edge [bend left] node {$b$} (C)
    (K) edge [loop right] node {$b$} ()
        edge [bend left] node {$a$} (B)
    ;

\end{tikzpicture}


\subsection{Návrh NFA a redukovaný DFA}
Je dán jazyk $L$ nad abecedou $\{a,b\}$ takto: 
\[L = \{w \mid w = ubabv; u,v\in\{a,b\}^\star\}\text{,}\] 
tj. $L$ se skládá ze všech slov, které obsahují slovo $bab$ jako podslovo. Zkonstruujte nejprve NFA $N$, který přijímá 
$L$. Podmnožinovou konstrukcí k $N$ zkonstruujte DFA a ten pak zredukujte.

NFA: 

\begin{tikzpicture}[node distance=25mm]

    \node[state, initial] (1) {$1$};
    \node[state] (2) [right of=1] {$2$};
    \node[state] (3) [right of=2] {$3$};
    \node[state, accepting] (4) [right of=3] {$4$};

    \path[->]
    (1) edge [loop above] node {$a,b$} ()
    (1) edge [bend left] node {$b$} (2)
    (2) edge [bend left] node {$a$} (3)
    (3) edge [bend left] node {$b$} (4)
    (4) edge [loop above] node {$a,b$} ();

\end{tikzpicture}

Podmnožinová konstrukce: 

\begin{tabular}{|r r|c c||c|c c||c|c c||c|c c||c|}
    \hline
    & & $a$ & $b$ & $ \sim_0 $ & $a$ & $b$ & $\sim_1$ & $a$ & $b$ & $\sim_2$ & $a$ & $b$ & $\sim_3$ \\ \hline \hline
              $\to$ & $\{1\}$ & $\{1\}$ & $\{1,2\}$ & $O$ & $O$ & $O$ & $O$ & $O$ & $O$ & $O$ & $O$ & $B$ & $O$ \\
                & $\{1,2\}$ & $\{1,3\}$ & $\{1,2\}$ & $O$ & $O$ & $O$ & $O$ & $A$ & $O$ & $B$ & $A$ & $B$ & $B$ \\
                & $\{1,3\}$ & $\{1\}$ & $\{1,2,4\}$ & $O$ & $O$ & $K$ & $A$ & $O$ & $K$ & $A$ & $O$ & $K$ & $A$ \\
    $\gets$&$\{1,2,4\}$ & $\{1,3,4\}$ & $\{1,2,4\}$ & $K$ & $K$ & $K$ & $K$ & $K$ & $K$ & $K$ & $K$ & $K$ & $K$ \\ 
    $\gets$ & $\{1,3,4\}$ & $\{1,4\}$ & $\{1,2,4\}$ & $K$ & $K$ & $K$ & $K$ & $K$ & $K$ & $K$ & $K$ & $K$ & $K$ \\ 
    $\gets$   & $\{1,4\}$ & $\{1,4\}$ & $\{1,2,4\}$ & $K$ & $K$ & $K$ & $K$ & $K$ & $K$ & $K$ & $K$ & $K$ & $K$ \\ 
    \hline
\end{tabular}

DFA: 

\begin{tikzpicture}[node distance=25mm]

    \node[state, initial] (O) {$O$};
    \node[state] (B) [right of=O] {$B$};
    \node[state] (A) [right of=B] {$A$};
    \node[state, accepting] (K) [right of=A] {$K$};

    \path[->]
    (O) edge [loop above] node {$a$} ()
    (O) edge [bend left] node {$b$} (B)
    (B) edge [bend left] node {$a$} (A)
    (B) edge [bend left] node {$b$} (O)
    (A) edge [bend left] node {$b$} (K)
    (A) edge [bend left] node {$a$} (O)
    (K) edge [loop above] node {$a,b$} ();

\end{tikzpicture}


% TODO: doplnit chybejici reseni

\subsection{Důkaz minimálního počtu stavů DFA pro jazyk}
Zjistěte, jaký je minimální počet stavů DFA, který přijímá jazyk $L_n = \{u1v \mid |v|=n-1\}$ nad abecedou 
$\Sigma = \{0,1\}$. Zdůvodněte. Jak by se změnil výsledek, kdyby bylo $\Sigma = \{0,1,2\}$?

\subsection{Pumping lemma pro doplněk}
Dokažte nebo vyvraťte toto tvrzení (Pumping lemma pro doplněk):

Pro každý regulární jazyk $L$ nad abecedou $\Sigma$ (tj. jazyk, který je přijímán nějakým DFA) existuje přirozené číslo 
$n$ s touto vlastností:

Každé slovo $u \not\in L$, které je delší než $n$ (tj. $|u| > n$) lze rozdělit na tři slova $u =xwy$, tak, že
\begin{enumerate}[noitemsep]
    \item $|xw| \leq n,$
    \item $w \not= \varepsilon$,
    \item pro každé přirozené $i = 0, 1, ...$ platí $xw^iy \not\in L$.
\end{enumerate}

\subsection{Návrh DFA dle jazyka}
Navrhněte deterministický konečný automat (DFA), který přijímá jazyk $L$ nad abecedou $\{a,b\}$, kde $L$ obsahuje právě 
všechna slova $w$ taková, že $|w|_a$ je dělitelné $5$, $w$ začíná $b$ a končí $a$.

\noindent
O navrženém automatu ukažte, že opravdu přijímá daný jazyk. 

\begin{multicols}{2} 

\begin{tikzpicture}[shorten >=1pt,node distance=25mm,on grid,auto]        
    
    \def\radius{18mm}
    \node[state, initial] (s) at (-5, 1.8) {$q_s$}; 
    \node[state] (m) at (-3, 1.8) {$q_m$};
    \node[state] (f) at (-4, -0.5) {$q_f$};  
    \node[state] (1) at (90:\radius) {$q_1$}; 
    \node[state] (2) at (18:\radius) {$q_2$}; 
    \node[state, accepting] (0) at (162:\radius) {$q_0$};
    \node[state] (3) at (306:\radius) {$q_3$};
    \node[state] (4) at (234:\radius) {$q_4$}; 

   


    \path[->]
    (s) edge [bend left=20] node {$b$} (m)
        edge [bend right=20] node [below left] {$a$} (f)
    (f) edge [loop below] node {$a,b$} ()
    (m) edge [loop above] node {$b$} ()
        edge [bend left=20] node {$a$} (1)
    (1) edge [loop above] node {$b$} ()
        edge [bend left=20] node {$a$} (2)
    (2) edge [loop above] node {$b$} ()
        edge [bend left=20] node {$a$} (3)
    (3) edge [loop right] node {$b$} ()
        edge [bend left=20] node {$a$} (4) 
    (4) edge [loop left] node {$b$} () 
        edge [bend left=20] node {$a$} (0) 
    (0) edge [bend left=20] node {$a$} (1) 
        edge [bend left=20] node {$b$} (m)
    ;

\end{tikzpicture}

Důkaz pomocí invariantů: 
\begin{itemize}[noitemsep]
    \item $q_s$: \texttt{in},
    \item $q_m$: $|w|_a = 5k, k \in \mathbb{Z}$, začíná $b$, končí $b$,
    \item $q_1$: $|w|_a = 5k + 1$, začíná $b$,
    \item $q_2$: $|w|_a = 5k + 2$, začíná $b$,
    \item $q_3$: $|w|_a = 5k + 3$, začíná $b$,
    \item $q_4$: $|w|_a = 5k + 4$, začíná $b$,
    \item $q_0$: $|w|_a = 5k$, začíná $b$, končí $a$, \texttt{out},
    \item $q_f$: začíná $a$.
\end{itemize}

\end{multicols}

\newpage % TODO: zkontrolovat az se doplni ostatni reseni
\subsection{Návrh DFA součinovou konstrukcí}
Navrhněte redukovaný DFA $M$, který přijímá jazyk $L$ nad $\Sigma = \{0,1\}$, kde
\[L = \{w \mid |w|_0 \text{ je sudé a za každým symbolem } 1 \text{ je symbol } 0\}\text{.}\]
Postupujte buď součinovou konstrukcí nebo přímo. V druhém případě řádně zdůvodněte, proč $M$ opravdu přijímá jazyk $L$.

Postup součinovou konstrukcí: Vytvoříme dva automaty, jeden pro pravidlo $|w|_0$ je sudé, a druhý pro 
"za každým symbolem $1$ je symbol $0$". 

\begin{multicols}{2} 
    1. 

    \begin{tikzpicture}[shorten >=1pt,node distance=30mm,on grid,auto]

        \node[state, initial, accepting] (s) {$s$};
        \node[state] (l) [right of=s] {$l$};

        \path[->]
        (s) edge [bend left] node {$0$} (l)
        (s) edge [loop above] node {$1$} ()
        (l) edge [bend left] node {$0$} (s)
        (l) edge [loop above] node {$1$} ();  

    \end{tikzpicture}

    2. 

    \begin{tikzpicture}[shorten >=1pt,node distance=30mm,on grid,auto]
        \node[state, initial, accepting] (0) {$0$};
        \node[state] (1) [right of=0] {$1$};
        \node[state] (2) [right of=1] {$2$};

        \path[->]
        (0) edge [loop above] node {$0$} ()
        (0) edge [bend left] node {$1$} (1)
        (1) edge [bend left] node {$1$} (2)
        (1) edge [bend left] node {$0$} (0)
        (2) edge [loop above] node {$0,1$} ()
        ;

    \end{tikzpicture}

\end{multicols}

DFA sestavený součinovou konstrukcí dvou výše nakreslených automatů: 

\begin{tikzpicture}[shorten >=1pt,node distance=30mm,on grid,auto]

    \node[state, initial, accepting] (s0) {$s_0$};
    \node[state] (s1) [right of=s0] {$s_1$};
    \node[state] (s2) [right of=s1] {$s_2$};
    \node[state] (l0) [below of=s0] {$l_0$};
    \node[state] (l1) [right of=l0] {$l_1$};
    \node[state] (l2) [right of=l1] {$l_2$};

    \path[->]
    (s0) edge [bend left] node {$1$} (s1)
    (s0) edge [bend right] node [below left] {$0$} (l0)
    (s1) edge [bend left] node {$1$} (s2)
    (s1) edge [bend left] node [left] {$0$} (l0)
    (s2) edge [loop right] node {$1$} ()
    (s2) edge [bend right] node [right] {$0$} (l2)
    (l0) edge [bend right] node [right] {$0$} (s0)
    (l0) edge [bend right] node [above] {$1$} (l1)
    (l1) edge [bend right] node [above] {$1$} (l2)
    (l1) edge [bend right] node [below] {$0$} (s0)
    (l2) edge [loop right] node {$1$} ()
    (l2) edge [bend right] node [below right] {$0$} (s2);

\end{tikzpicture}


\subsection{Návrh DFA}
Navrhněte redukovaný DFA $M$, který přijímá jazyk $L$ nad $\Sigma = \{0,1\}$, kde
\[L = \{w \mid w \text{ začíná } 10 \text{ nebo končí } 01\}\text{.}\]
Zdůvodněte, proč $M$ přijímá jazyk $L$.
