\section*{Úvod}

\hspace{0.8cm} Tento text není psán jako učebnice, nýbrž jako soubor řešených příkladů, u kterých je vždy uveden celý korektní postup
a případné poznámky řešitelů, které často nebývají formální, a tedy by neměly být používány při oficálním řešení problémů, 
například při zkoušce. Jedná se pouze o pokus předat probíranou látku z různých úhlů pohledu, pokud by korektní matematický
nebyl dostatečně výřečný.

\hspace{0.8cm} Autoři velmi ocení, pokud čtenáři zašlou své podněty, úpravy anebo připomínky k textu. Budeme rádi za všechnu konstruktivní
kritiku a nápady na změny. Dejte nám také prosím vědět, pokud v textu objevíte překlepy, chyby a jiné.

Errata a aktuální verse textu bude na stránce \url{https://github.com/knedl1k/B4B01JAG}.

\textbf{Poděkování.} Rádi bychom poděkovali profesorce Marii Demlové nejen za zadání, okolo kterých je postavena celá sbírka,
ale také za celý předmět Jazyků, automatů a gramatik.

\hspace{0.8cm} Text je vysázen makrem \LaTeX{} Leslieho Lamporta s využitím balíků \texttt{hypperref} \\ 
Sebastiana Rahtze a Heiko Oberdiek. Automaty byly nakresleny pomocí maker Ti\textit{k}Z Tilla Tantaua a derivační
stromy pomocí maker \texttt{forest} Sašo Živanoviće.

\subsection*{Stručné informace o textu}
Všechny růžové texty jsou zároveň hypertextové odkazy. Často jsou použity u  
\href{https://youtube.com/playlist?list=PLQL6z4JeTTQkLuzI78OTnfYBclE1g0UjS&si=7fIuogtxj7mi1HZ-}{přednáškových} 
příkladů, pomocí nichž lze vidět ukázkové řešení příkladu na přednášce.

U každého příkladu je pro ušetření místa a zjednodušení visuálnosti sbírky řešení jednotlivých příkladů ihned pod zadáním.

